
Boost.Test 是最古老,也是最受欢迎的 C++ 测试框架,提供了一组易于使用的 API,用于编写测试并将其组织成测试用例和测试套件。Boost.Test 对断言、异常处理、固件等测试框架所需的重要特性支持良好。

接下来的几个示例中将展示其特性,这些特性能够编写单元测试。本示例中,将介绍如何安装框架并创建一个简单的测试项目。

\mySubsubsection{}{Getting ready}

Boost.Test 框架有一个基于宏的 API。只需要使用提供的宏来编写测试,但想很好地使用这个框架,建议对宏有一个很好的理解。

\mySubsubsection{}{How to do it...}

为了设置环境以使用 Boost.Test,请按照以下步骤操作:

\begin{itemize}
\item
从 \url{http://www.boost.org/} 下载最新版本的 Boost 库。

\item
解压存档内容。

\item
使用提供的工具和脚本来构建库,以便使用静态或动态库版本。如果计划使用纯头文件版本的库,则此步不是必需的。
\end{itemize}

Linux 系统上,也可以使用包管理工具来安装库。例如,在 Ubuntu 上,可以使用 app-get 安装包含 Boost.Test 库的 libboost-test-dev 包,命令如下:

\begin{shell}
sudo apt-get install libboost-test-dev
\end{shell}

\begin{myNotic}
建议查阅库的在线文档以获取各种系统的安装步骤。
\end{myNotic}

要使用 Boost.Test 库的纯头文件版本创建的第一个测试程序,请按照以下步骤操作:

\begin{enumerate}
\item
创建一个新的空白 C++ 项目。

\item
进行特定于开发环境的必要设置,以便项目的 Boost 主文件夹可用于包含头文件。

\item
向项目中添加一个新源文件:

\begin{cpp}
#define BOOST_TEST_MODULE My first test module
#include <boost/test/included/unit_test.hpp>
BOOST_AUTO_TEST_CASE(first_test_function)
{
    int a = 42;
    BOOST_TEST(a > 0);
}
\end{cpp}

\item
如果想链接到动态库,还需要定义 \verb|BOOST_TEST_DYN_LINK| 宏。

\item
构建并运行项目。
\end{enumerate}

\mySubsubsection{}{How it works...}

Boost.Test 库可以从 \url{http://www.boost.org/} 下载,随同其他 Boost 库一起提供。本书中使用的是 1.83 版本,这些示例中讨论的功能可能会在未来的许多版本中可用。测试库有三种版本:

\begin{itemize}
\item
单头文件:可以在不构建库的情况下编写测试程序,只需要包含一个头文件。其局限性在于模块只能有一个翻译单元,可以将模块拆分为多个头文件,以便将不同的测试套件分开存放。

\item
静态库:可以在不同的翻译单元之间分割模块,但首先需要将库构建为静态库。

\item
动态库:与静态库相同,但优点在于,对于具有许多测试模块的程序,这个库只链接一次,而不是每个模块链接一次,从而可减小二进制文件大小。但在这种情况下,运行时必须有动态库可用。
\end{itemize}

简单起见,本书将使用单头文件版本。静态和动态库版本,需要构建库。下载的存档文件中包含了构建库的脚本。但具体的步骤根据平台和编译器的不同而有所变化;这里不会介绍这些步骤,可以在网上去查找。

为了使用库,有几个术语和概念需要理解:

\begin{itemize}
\item
测试模块是一个执行测试的程序。有两种类型的模块:单文件(使用单头文件版本时)和多文件(使用静态或动态版本时)。

\item
测试断言检查测试模块的条件。

\item
测试用例是一组一个或多个测试断言,由测试模块独立执行和监控,如果失败或泄露未捕获的异常,其他测试的执行就不会停止。

\item
测试套件是一个或多个测试用例或测试套件的集合。

\item
测试单元是测试用例或测试套件。

\item
测试树是测试单元的层次结构。在此结构中,测试用例是叶子节点,测试套件是非叶子节点。

\item
测试运行器是给定测试树后执行必要初始化、测试执行和结果报告的组件。

\item
测试报告是测试运行器执行测试后产生的报告。

\item
测试日志是在测试模块执行期间记录的所有事件。

\item
测试设置是负责框架初始化、测试树构建和个人测试用例设置的测试模块部分。

\item
测试清理是负责清理操作的测试模块部分。

\item
测试固件是一对设置和清理操作,为多个测试单元调用,以避免重复代码。
\end{itemize}

定义了这些概念后,就可以解释上面列出的示例代码了:

\begin{enumerate}
\item
\verb|#define BOOST_TEST_MODULE|在 My first test module 定义了一个模块初始化的存根,并为主测试套件设置了一个名称。这必须在包含任何库头文件之前定义。

\item
\verb|#include <boost/test/included/unit_test.hpp>| 包含单头文件库,其中包含了所有其他必要的头文件。

\item
\verb|BOOST_AUTO_TEST_CASE(first_test_function)| 声明了一个无参数的测试用例 (\verb|first_test_function|),并自动注册将其包含在测试树中作为封装测试套件的一部分。这个例子中,测试套件是主测试套件,由 \verb|BOOST_TEST_MODULE| 定义。

\item
\verb|BOOST_TEST(a > 0);| 执行测试断言。
\end{enumerate}

执行此测试模块的输出如下:

\begin{shell}
Running 1 test case...
*** No errors detected
\end{shell}

\mySubsubsection{}{There's more...}

如果不想让库生成 main() 函数,而希望自己编写,则需要在包含库头文件之前定义另外两个宏 - \verb|BOOST_TEST_NO_MAIN| 和 \verb|BOOST_TEST_ALTERNATIVE_INIT_API|。然后,在自己的 main() 函数中,通过提供默认初始化函数 \verb|init_unit_test()| 作为参数来调用默认的测试运行器 \verb|unit_test_main()|:

\begin{cpp}
#define BOOST_TEST_MODULE My first test module
#define BOOST_TEST_NO_MAIN
#define BOOST_TEST_ALTERNATIVE_INIT_API
#include <boost/test/included/unit_test.hpp>
BOOST_AUTO_TEST_CASE(first_test_function)
{
    int a = 42;
    BOOST_TEST(a > 0);
}
int main(int argc, char* argv[])
{
    return boost::unit_test::unit_test_main(init_unit_test, argc, argv);
}
\end{cpp}

还可以自定义测试运行器的初始化函数,必须移除 \verb|BOOST_TEST_MODULE| 宏的定义,并改写一个不带参数并返回布尔值的初始化函数:

\begin{cpp}
#define BOOST_TEST_NO_MAIN
#define BOOST_TEST_ALTERNATIVE_INIT_API
#include <boost/test/included/unit_test.hpp>
#include <iostream>
BOOST_AUTO_TEST_CASE(first_test_function)
{
    int a = 42;
    BOOST_TEST(a > 0);
}
bool custom_init_unit_test()
{
    std::cout << "test runner custom init\n";
    return true;
}
int main(int argc, char* argv[])
{
    return boost::unit_test::unit_test_main(
        custom_init_unit_test, argc, argv);
}
\end{cpp}

\begin{myTip}
可在自己不定义 main() 函数的情况下,自定义初始化函数。这时,不应该定义 \verb|BOOST_TEST_NO_MAIN| 宏,初始化函数应命名为 \verb|init_unit_test()|。
\end{myTip}

