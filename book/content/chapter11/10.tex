
默认情况下,Google Test 程序的输出会发送到标准流,并以人类可读的格式打印。框架提供了几种自定义输出的选项,包括以基于 JUnit 的格式将 XML 打印到磁盘文件中。

本示例将介绍可用于控制输出的选项。

\mySubsubsection{}{Getting ready}

为了本示例的目的,看看以下测试程序:

\begin{cpp}
#include <gtest/gtest.h>
TEST(Sample, Test)
{
    auto a = 42;
    ASSERT_EQ(a, 0);
}
int main(int argc, char **argv)
{
    testing::InitGoogleTest(&argc, argv);
    return RUN_ALL_TESTS();
}
\end{cpp}

其输出如下:

\begin{shell}
[==========] Running 1 test from 1 test suite.
[----------] Global test environment set-up.
[----------] 1 test from Sample
[ RUN      ] Sample.Test
f:\chapter11gt_05\main.cpp(6): error: Expected equality of these values:
    a
        Which is: 42
    0
[  FAILED  ] Sample.Test (1 ms)
[----------] 1 test from Sample (1 ms total)
[----------] Global test environment tear-down
[==========] 1 test from 1 test suite ran. (3 ms total)
[  PASSED  ] 0 tests.
[  FAILED  ] 1 test, listed below:
[  FAILED  ] Sample.Test
1 FAILED TEST
\end{shell}

使用这个简单的测试程序,来演示可以用来控制程序输出的各种选项,这些选项在接下来的部分中举例说明。

\mySubsubsection{}{How to do it...}

要控制测试程序的输出:

\begin{itemize}
\item
使用 \verb|--gtest_output| 命令行选项或 \verb|GTEST_OUTPUT| 环境变量与 \verb|xml:filepath| 字符串一起指定 XML 报告要写入的文件位置:

\begin{shell}
chapter11gt_05.exe --gtest_output=xml:report.xml
<?xml version="1.0" encoding="UTF-8"?>
<testsuites tests="1" failures="1" disabled="0" errors="0"
time="0.007" timestamp="2020-05-18T19:00:17"
name="AllTests">
<testsuite name="Sample" tests="1" failures="1" disabled="0"
errors="0" time="0.002"
timestamp="2020-05-18T19:00:17">
<testcase name="Test" status="run" result="completed" time="0"
timestamp="2020-05-18T19:00:17" classname="Sample">
<failure message="f:\chapter11gt_05\main.cpp:6&#x0A;Expected equality of these values:&#x0A;  a&#x0A;    Which is: 42&#x0A;  0&#x0A;" type=""><![CDATA[f:\chapter11gt_05\main.cpp:6
Expected equality of these values:
a
Which is: 42
0
]]></failure>
</testcase>
</testsuite>
</testsuites>
\end{shell}

\item
使用 \verb|--gtest_color| 命令行选项或 \verb|GTEST_COLOR| 环境变量,并指定 \verb|auto|, \verb|yes|, 或 \verb|no| 来指示报告是否应使用颜色输出到终端:

\begin{shell}
chapter11gt_05.exe --gtest_color=no
\end{shell}

\item
使用 \verb|--gtest_print_time| 命令行选项或 \verb|GTEST_PRINT_TIME| 环境变量并设置值为 0 来抑制打印每个测试执行所需的时间:

\begin{shell}
chapter11gt_05.exe --gtest_print_time=0
[==========] Running 1 test from 1 test suite.
[----------] Global test environment set-up.
[----------] 1 test from Sample
[ RUN      ] Sample.Test
f:\chapter11gt_05\main.cpp(6): error: Expected equality of these values:
    a
        Which is: 42
    0
[  FAILED  ] Sample.Test
[----------] Global test environment tear-down
[==========] 1 test from 1 test suite ran.
[  PASSED  ] 0 tests.
[  FAILED  ] 1 test, listed below:
[  FAILED  ] Sample.Test
1 FAILED TEST
\end{shell}
\end{itemize}

\mySubsubsection{}{How it works...}

生成 XML 格式的报告不会影响打印到终端的人类可读报告。输出路径可以指向文件、目录(将创建一个与可执行文件同名的文件——如果已存在,则通过附加数字后缀来创建一个新名称的文件),或者为空,报告将写入当前目录下的名为 \verb|test_detail.xml| 的文件中。

XML 报告格式基于 JUnitReport Ant 任务,包含以下元素:

\begin{itemize}
\item
<testsuites>: 根元素,对应整个测试程序。

\item
<testsuite>: 对应一个测试套件。

\item
<testcase>: 对应一个测试函数,正如 Google Test 函数相当于其他框架中的测试用例一样。
\end{itemize}

默认情况下,框架报告每个测试执行所需的时间。此功能可以通过使用 \verb|--gtest_print_time| 命令行选项或 \verb|GTEST_PRINT_TIME| 环境变量来抑制,如前所述。



