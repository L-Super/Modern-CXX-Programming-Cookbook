测试代码是软件开发的重要组成部分。虽然 C++ 标准中没有支持测试的功能,但有许多框架可用于单元测试 C++ 代码。本章的目的在于让你开始接触几个现代且广泛使用的测试框架,这些框架使你能够编写可移植的测试代码。本章将涵盖的框架有 Boost.Test、Google Test 和 Catch2。

本章包含以下示例:

\begin{itemize}
\item
开始使用 Boost.Test

\item
使用 Boost.Test 编写和调用测试

\item
使用 Boost.Test 断言

\item
使用 Boost.Test 的测试固件

\item
使用 Boost.Test 控制输出

\item
开始使用 Google Test

\item
使用 Google Test 编写和调用测试

\item
使用 Google Test 断言

\item
使用 Google Test 的测试固件

\item
使用 Google Test 控制输出

\item
开始使用 Catch2

\item
使用 Catch2 编写和调用测试

\item
使用 Catch2 断言

\item
使用 Catch2 控制输出
\end{itemize}

选择这三个框架是因为它们广泛的使用、丰富的功能、易于编写和执行、可扩展性和可定制性。下表简要比较了这三个库的特点:

% Please add the following required packages to your document preamble:
% \usepackage{longtable}
% Note: It may be necessary to compile the document several times to get a multi-page table to line up properly
\begin{longtable}{|l|l|l|l|}
\hline
\textbf{特性}            & \textbf{Boost.Test}  & \textbf{Google Test} & \textbf{Catch2 (v3)}      \\ \hline
\endfirsthead
%
\endhead
%
易于安装             & Yes                  & Yes                  & Yes                       \\ \hline
仅头文件库                  & Yes                  & No                   & No                        \\ \hline
编译库            & Yes                  & Yes                  & Yes                       \\ \hline
易于编写测试         & Yes                  & Yes                  & Yes                       \\ \hline
自动测试注册 & Yes                  & Yes                  & Yes                       \\ \hline
支持测试套件        & Yes                  & Yes                  & No (通过标签间接支持) \\ \hline
支持固件           & Yes (设置/清理) & Yes (设置/清理) & Yes (多种方式)       \\ \hline
丰富的断言集         & Yes                  & Yes                  & Yes                       \\ \hline
非致命断言           & Yes                  & Yes                  & Yes                       \\ \hline
多种输出格式  & Yes (包括 HRF, XML) & Yes (包括 HRF, XML) & Yes (包括 HRF, XML) \\ \hline
测试执行过滤n & Yes                  & Yes                  & Yes                       \\ \hline
许可证                      & Boost                & Apache 2.0           & Boost                     \\ \hline
\end{longtable}

\begin{center}
表 11.1: Boost.Test、Google Test 和 Catch2 的特性对比
\end{center}

每个框架的所有这些特点都将在本章中详细讨论。本章结构对称,每个测试框架有 4 到 5 个示例。首先介绍的框架是 Boost.Test。