文件和目录的操作,例如复制、移动和删除文件,或者创建、重命名和删除目录,都由文件系统库支持。文件和目录通过路径来标识(可以是绝对路径、规范路径或相对路径)。本示例中,将介绍上述操作的标准函数是什么,以及其工作原理。

\mySubsubsection{}{Getting ready}

对于以下所有示例,将使用以下变量,并假设当前路径在 Windows 上是 \verb|C:\Users\Marius\Documents|,而在 POSIX 系统上是 /home/marius/docs:

\begin{cpp}
auto err = std::error_code{};
auto basepath = fs::current_path();
auto path = basepath / "temp";
auto filepath = path / "sample.txt";
\end{cpp}

假设当前路径的 temp 子目录中存在一个名为 sample.txt 的文件(例如 \verb|C:\Users\Marius\Documents\temp\sample.txt| 或 \verb|/home/marius/docs/temp/sample.txt|)。

\mySubsubsection{}{How to do it...}

使用以下库函数执行目录操作:

\begin{itemize}
\item
要创建新目录,请使用 \verb|create_directory()|。如果目录已存在,此方法不会执行任何操作,不会递归地创建目录:

\begin{cpp}
auto success = fs::create_directory(path, err);
\end{cpp}

\item
要递归地创建新目录,请使用 \verb|create_directories()|:

\begin{cpp}
auto temp = path / "tmp1" / "tmp2" / "tmp3";
auto success = fs::create_directories(temp, err);
\end{cpp}

\item
要移动现有目录,请使用 rename():

\begin{cpp}
auto temp = path / "tmp1" / "tmp2" / "tmp3";
auto newtemp = path / "tmp1" / "tmp3";
fs::rename(temp, newtemp, err);
if (err) std::cout << err.message() << '\n';
\end{cpp}

\item
要重命名现有目录,也请使用 rename():

\begin{cpp}
auto temp = path / "tmp1" / "tmp3";
auto newtemp = path / "tmp1" / "tmp4";
fs::rename(temp, newtemp, err);
if (err) std::cout << err.message() << '\n';
\end{cpp}

\item
要复制现有目录,请使用 copy()。要递归地复制整个目录的内容,请使用 \verb|copy_options::recursive| 标志:

\begin{cpp}
fs::copy(path, basepath / "temp2",
         fs::copy_options::recursive, err);
if (err) std::cout << err.message() << '\n';
\end{cpp}

\item
要创建指向目录的符号链接,请使用 \verb|create_directory_symlink()|:

\begin{cpp}
auto linkdir = basepath / "templink";
fs::create_directory_symlink(path, linkdir, err);
if (err) std::cout << err.message() << '\n';
\end{cpp}

\item
要删除空目录,请使用 remove():

\begin{cpp}
auto temp = path / "tmp1" / "tmp4";
auto success = fs::remove(temp, err);
\end{cpp}

\item
要递归地删除目录及其全部内容,请使用 \verb|remove_all()|:

\begin{cpp}
auto success = fs::remove_all(path, err) !=
               static_cast<std::uintmax_t>(-1);
\end{cpp}

\item
要更改目录或文件的权限,请使用 permissions(),指定来自 perms 枚举的权限选项。除非从 \verb|perm_options| 枚举中指定操作类型(替换、添加或移除),否则默认行为是用指定的权限替换所有现有权限:

\begin{cpp}
// replace permissions with specified ones
fs::permissions(temp, fs::perms::owner_all |
                      fs::perms::group_all, err);
if (err) std::cout << err.message() << '\n';
// remove specified permissions
fs::permissions(temp, fs::perms::group_exec,
                      fs::perm_options::remove, err);
if (err) std::cout << err.message() << '\n';
\end{cpp}
\end{itemize}

使用以下库函数执行文件操作:

\begin{itemize}
\item
要复制文件,请使用 \verb|copy()| 或 \verb|copy_file()|。下一节会解释这两个函数之间的区别:

\begin{cpp}
auto success = fs::copy_file(filepath, path / "sample.bak", err);
if (!success) std::cout << err.message() << '\n';
fs::copy(filepath, path / "sample.cpy", err);
if (err) std::cout << err.message() << '\n';
\end{cpp}

\item
要重命名文件,请使用 rename():

\begin{cpp}
auto newpath = path / "sample.log";
fs::rename(filepath, newpath, err);
if (err) std::cout << err.message() << '\n';
\end{cpp}

\item
要移动文件,请使用 rename():

\begin{cpp}
auto newpath = path / "sample.log";
fs::rename(newpath, path / "tmp1" / "sample.log", err);
if (err) std::cout << err.message() << '\n';
\end{cpp}

\item
要创建指向文件的符号链接,请使用 \verb|create_symlink()|:

\begin{cpp}
auto linkpath = path / "sample.txt.link";
fs::create_symlink(filepath, linkpath, err);
if (err) std::cout << err.message() << '\n';
\end{cpp}

\item
要删除文件,请使用 remove():

\begin{cpp}
auto success = fs::remove(path / "sample.cpy", err);
if (!success) std::cout << err.message() << '\n';
\end{cpp}
\end{itemize}

\mySubsubsection{}{Hot it works...}

本示例中提到的所有函数,都有多个重载版本,可以分为两类:

\begin{itemize}
\item
最后一个参数接受 \verb|std::error_code| 引用的重载:这些重载不会抛出异常(定义为 noexcept)。相反,如果发生操作系统错误,会将 \verb|error_code| 值设置为操作系统错误代码。如果没有此类错误,则调用 \verb|error_code| 的 clear() 方法以重置先前设置的错误码。

\item
不接受最后一个参数为 \verb|std::error_code| 类型的重载:如果发生错误,这些重载会抛出异常。如果发生操作系统错误,会抛出 \verb|std::filesystem::filesystem_error| 异常。另一方面,如果内存分配失败,这些函数会抛出 \verb|std::bad_alloc| 异常。
\end{itemize}

上一节中的所有示例都使用了不抛出异常,而是在发生错误时设置代码的重载。一些函数返回布尔值以指示成功或失败。可以通过检查方法 value() 返回的错误代码值是否不同于 0,或者使用转换操作符 bool(对于同样的情况返回 true,否则返回 false)来判断 \verb|error_code| 是否持有错误代码。要获取错误码的说明字符串,请使用 message() 方法。

某些文件系统库函数同时适用于文件和目录。这适用于 rename()、remove() 和 copy()。每个函数的可能很复杂,特别是 copy()。

关于复制文件,有两种函数可以使用:\verb|copy()| 和 \verb|copy_file()|。这两种函数具有相同的签名的等效重载,表面上看起来工作方式相同。但有一个重要的区别(除了 \verb|copy()| 同样适用于目录之外):\verb|copy_file()| 会跟随符号链接。

为了避免这样,而是复制实际的符号链接,必须使用 \verb|copy_symlink()| 或带有 \verb|copy_options::copy_symlinks| 标志的 \verb|copy()|。无论是 \verb|copy()| 还是 \verb|copy_file()| 函数都接受 \verb|std::filesystem::copy_options| 类型参数的重载,该参数定义了操作应该如何执行。\verb|copy_options| 是一个作用域枚举:

\begin{cpp}
enum class copy_options
{
    none               = 0,
    skip_existing      = 1,
    overwrite_existing = 2,
    update_existing    = 4,
    recursive          = 8,
    copy_symlinks      = 16,
    skip_symlinks      = 32,
    directories_only   = 64,
    create_symlinks    = 128,
    create_hard_links  = 256
};
\end{cpp}

下表定义了每个标志如何影响 \verb|copy()| 或 \verb|copy_file()| 的复制操作。表格摘自 C++ 标准 N4917 版本的第 31.12.8.3 段落:

% Please add the following required packages to your document preamble:
% \usepackage{longtable}
% Note: It may be necessary to compile the document several times to get a multi-page table to line up properly
\begin{longtable}{|ll|}
\hline
\multicolumn{2}{|l|}{\textbf{控制现有目标文件 copy\_file 函数效果的选项组}}             \\ \hline
\endfirsthead
%
\endhead
%
\multicolumn{1}{|l|}{none}                & (默认)错误;文件已存在                                 \\ \hline
\multicolumn{1}{|l|}{skip\_existing}      & 不覆盖现有文件;不报告错误                \\ \hline
\multicolumn{1}{|l|}{overwrite\_existing} & 覆盖现有文件                                          \\ \hline
\multicolumn{1}{|l|}{update\_existing}    & 如果现有文件比替换文件旧,则覆盖现有文件 \\ \hline
\multicolumn{2}{|l|}{\textbf{控制 copy 函数子目录效果的选项组}}                          \\ \hline
\multicolumn{1}{|l|}{none}                & 默认)不复制子目录                                  \\ \hline
\multicolumn{1}{|l|}{recursive}           & 递归复制子目录及其内容                   \\ \hline
\multicolumn{2}{|l|}{\textbf{控制 copy 函数符号链接效果的选项组}}                          \\ \hline
\multicolumn{1}{|l|}{none}                & (默认)跟随符号链接                                       \\ \hline
\multicolumn{1}{|l|}{copy\_symlinks} &
复制符号链接本身,而非其所指向的目标 \\ \hline
\multicolumn{1}{|l|}{skip\_symlinks}      &忽略符号链接,不复制                                                \\ \hline
\multicolumn{2}{|l|}{\textbf{控制复制功能效果的选项组,用于选择复制的形式}}            \\ \hline
\multicolumn{1}{|l|}{none}                & (默认)复制内容                                              \\ \hline
\multicolumn{1}{|l|}{directories\_only}   & 仅复制目录,忽略文件   \\ \hline
\multicolumn{1}{|l|}{create\_symlinks} &
\begin{tabular}[c]{@{}l@{}}创建符号链接而不是文件副本;\\源路径将是绝对路径,除非目标路径在当前目录中\end{tabular}  \\ \hline
\multicolumn{1}{|l|}{create\_hard\_links} & 制作硬链接,而非文件副本                           \\ \hline
\end{longtable}

\begin{center}
表 7.3: copy\_operation 标志如何影响复制操作
\end{center}

另一个需要提及的方面与符号链接有关:\verb|create_directory_symlink()| 创建指向目录的符号链接,而 \verb|create_symlink()| 则创建指向文件或目录的符号链接。在 POSIX 系统上,当涉及到目录时,两者是相同的。但其他系统(如 Windows)中,创建指向目录的符号链接与创建指向文件的符号链接的方式不同,建议为了编写在所有系统上都能正确运行的代码,对于目录使用 \verb|create_directory_symlink()|。

\begin{myTip}
进行文件和目录的操作,如本示例中描述的那些操作,并且使用可能会抛出异常的重载时,请在调用时使用 try-catch。无论使用哪种类型的重载,都应检查操作的成功与否,并在失败的情况下采取适当的措施。
\end{myTip}

如果需要更改文件或目录的权限,可以使用 permissions() 函数。其有几个重载版本,允许指定一系列权限选项。这些选项定义在 \verb|std::filesystem::perms| 枚举中。如果没有指定特定的更改操作,将会执行现有权限的完整替换。但可以使用 \verb|std::filesystem::perm_options| 枚举中提供的选项来指定添加或删除权限。除了替换、添加和移除外,还有一个第四个选项,即 nofollow。这适用于符号链接,这样权限就会更改在符号链接本身上,而不是解析到原文件上。

