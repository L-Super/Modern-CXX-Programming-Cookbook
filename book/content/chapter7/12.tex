

本章中,已经探讨了文件系统库提供的许多功能,例如:处理路径、对文件和目录进行操作(创建、移动、重命名、删除等),以及查询或修改属性。当处理文件系统时,另一个有用的功能是遍历目录的内容。文件系统库提供了两种目录迭代器,一个为 \verb|directory_iterator| 的迭代器,迭代目录的内容,另一个是 \verb|recursive_directory_iterator| 的迭代器,递归地迭代目录及其子目录的内容。本示例中,将介绍如何使用。

\mySubsubsection{}{Getting ready}

本示例中,将使用以下目录结构:

\begin{shell}
test/
|--data/
|   |--input.dat
|   \--output.dat
|--file_1.txt
|--file_2.txt
\--file_3.log
\end{shell}

接下来的代码中,将引用以下函数:

\begin{cpp}
void print_line(std::string_view prefix,
std::filesystem::path const& path)
{
    std::cout << prefix << path << '\n';
}
\end{cpp}

本示例中,将处理文件系统路径并检查文件系统实例的属性。

\mySubsubsection{}{How to do it...}

使用以下模式来枚举目录的内容:

\begin{itemize}
\item
要仅迭代目录的内容而不递归访问其子目录,使用 \verb|directory_iterator|:

\begin{cpp}
void visit_directory(fs::path const & dir)
{
    if (fs::exists(dir) && fs::is_directory(dir))
    {
        for (auto const & entry : fs::directory_iterator(dir))
        {
            auto filename = entry.path().filename();
            if (fs::is_directory(entry.status()))
                print_line("[+]", filename);
            else if (fs::is_symlink(entry.status()))
                print_line("[>]", filename);
            else if (fs::is_regular_file(entry.status()))
                print_line(" ", filename);
            else
                print_line("[?]", filename);
        }
    }
}
\end{cpp}

\item
要迭代目录的所有内容,包括其子目录,当处理条目的顺序无关紧要时,使用 \verb|recursive_directory_iterator|:

\begin{cpp}
void visit_directory_rec(fs::path const & dir)
{
    if (fs::exists(dir) && fs::is_directory(dir))
    {
        for (auto const & entry :
        fs::recursive_directory_iterator(dir))
        {
            auto filename = entry.path().filename();
            if (fs::is_directory(entry.status()))
                print_line("[+]", filename);
            else if (fs::is_symlink(entry.status()))
                print_line("[>]",filename);
            else if (fs::is_regular_file(entry.status()))
                print_line(" ",filename);
            else
                print_line("[?]",filename);
        }
    }
}
\end{cpp}

\item
以结构化的方式迭代目录的所有内容,包括其子目录,例如:遍历树结构。可以使用与第一个示例类似的函数,该函数使用 \verb|directory_iterator| 迭代目录的内容。但对于每个子目录,可递归地调用:

\begin{cpp}
void visit_directory_rec_ordered(
fs::path const & dir,
bool const recursive = false,
unsigned int const level = 0)
{
    if (fs::exists(dir) && fs::is_directory(dir))
    {
        auto lead = std::string(level*3, ' ');
        for (auto const & entry : fs::directory_iterator(dir))
        {
            auto filename = entry.path().filename();
            if (fs::is_directory(entry.status()))
            {
                print_line(lead + "[+]", filename);
                if(recursive)
                    visit_directory_rec_ordered(entry, recursive,
                level+1);
            }
            else if (fs::is_symlink(entry.status()))
                print_line(lead + "[>]", filename);
            else if (fs::is_regular_file(entry.status()))
                print_line(lead + " ", filename);
            else
                print_line(lead + "[?]", filename);
        }
    }
}
\end{cpp}
\end{itemize}

\mySubsubsection{}{Hot it works...}

\verb|directory_iterator| 和 \verb|recursive_directory_iterator| 都是输入迭代器,用于迭代目录。不同之处在于前者不会递归访问子目录,而后者正如其名称所暗示的那样,会递归访问。其行为有相似之处:

\begin{itemize}
\item
迭代顺序未指定。

\item
每个目录条目只访问一次。

\item
特殊的路径点(.)和双点(..)会跳过。默认构造的迭代器是结束迭代器,两个结束迭代器总相等。

\item
当迭代超过最后一个目录条目后,会等于结束迭代器。

\item
标准没有规定在迭代器创建后,向迭代目录中添加或删除目录会发生什么。

\item
标准为 \verb|directory_iterator| 和 \verb|recursive_directory_iterator| 定义了非成员函数 begin() 和 end(),就可以在基于范围的 for 循环中使用这些迭代器。
\end{itemize}

两个迭代器都有重载的构造函数。其中,\verb|recursive_directory_iterator| 的某些构造函数重载接受 \verb|std::filesystem::directory_options| 类型的参数,该参数指定了迭代的选项:

\begin{itemize}
\item
\verb|none|: 这是默认值,不指定内容。

\item
\verb|follow_directory_symlink|: 指定迭代应跟随符号链接,而非提供链接本身。

\item
\verb|skip_permission_denied|: 指定应忽略并跳过可能导致访问拒绝错误的目录。
\end{itemize}

两个目录迭代器指向的元素都是 \verb|directory_entry| 类型的。成员函数 \verb|path()| 返回此对象表示的文件系统对象的路径。可以通过成员函数 \verb|status()| 和 \verb|symlink_status()| (针对符号链接)获取文件系统实例的状态。

前面的例子遵循了一个共同的模式:

\begin{enumerate}
\item
验证要迭代的路径确实存在。

\item
使用基于范围的 for 循环迭代目录中的所有目录条目。

\item
根据迭代方式选择文件系统库中可用的两种目录迭代器之一。

\item
根据需求处理每个目录条目。
\end{enumerate}

例子中,只是将目录条目的名称打印到控制台。需要注意的是,目录内容以未指定的顺序迭代。如果想以结构化的方式处理内容,比如:显示缩进的子目录及其条目(这种特定情况下)或树形结构(其他类型的应用程序中),那么不合适使用 \verb|recursive_directory_iterator|。应该像上一节最后一个例子那样,使用 \verb|directory_iterator| 在迭代时对每个子目录递归调用一个函数。

使用本示例开头呈现的目录结构(相对于当前路径),当使用递归迭代器时,会得到如下输出:

\begin{cpp}
visit_directory_rec(fs::current_path() / "test");
\end{cpp}

\begin{shell}
[+]data
    input.dat
    output.dat
    file_1.txt
    file_2.txt
    file_3.log
\end{shell}

另一方面,当使用第三个例子中的递归函数时,输出会按照子层级的顺序显示:

\begin{cpp}
visit_directory_rec_ordered(fs::current_path() / "test", true);
\end{cpp}

\begin{shell}
[+]data
        input.dat
        output.dat
    file_1.txt
    file_2.txt
    file_3.log
\end{shell}

\verb|visit_directory_rec()| 函数是非递归函数,使用 \verb|recursive_directory_iterator|,而 \verb|visit_directory_rec_ordered()| 函数是递归函数,使用 \verb|directory_iterator|。这个例子应该能区别这两种迭代器。

\mySubsubsection{}{There's more...}

在上一个示例中,还讨论了 \verb|file_size()| 函数,该函数返回文件的字节数。然而,如果指定的路径是一个目录,该函数将失败。为了确定目录的大小,需要递归地遍历目录的内容,检索常规文件或符号链接的大小,并将它们加在一起。

\begin{cpp}
std::uintmax_t dir_size(fs::path const & path)
{
    if (fs::exists(path) && fs::is_directory(path))
    {
        auto size = static_cast<uintmax_t>(0);
        for (auto const & entry :
        fs::recursive_directory_iterator(path))
        {
            if (fs::is_regular_file(entry.status()) ||
            fs::is_symlink(entry.status()))
            {
                auto err = std::error_code{};
                auto filesize = fs::file_size(entry, err);
                if (!err)
                    size += filesize;
            }
        }
        return size;
    }
    return static_cast<uintmax_t>(-1);
}
\end{cpp}

\verb|dir_size()| 函数返回目录中所有文件(递归地)的大小总和,或者在错误的情况下(路径不存在或不表示目录)返回 -1(作为 \verb|uintmax_t| 类型的值)。

