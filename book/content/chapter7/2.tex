上一个示例中,介绍了如何将原始数据(即非结构化数据)写入和从文件中读取。但在很多时候,必须保存和加载对象而非原始数据。按照前一个示例中展示的方式进行写入和读取仅适用于POD类型(Plain Old Data,简单旧数据类型)。对于其他任何类型,必须明确决定实际上写入或读取什么,因为写入或读取指针(包括指向虚表的指针),以及任何形式的元数据不仅无关紧要,而且在语义上也是错误的。这些操作通常称为序列化和反序列化。这个示例中,将介绍如何将POD和非POD类型序列化和反序列化到二进制文件中。

\mySubsubsection{}{Getting ready}

为了这个示例中的例子,将使用foo和foopod类型:

\begin{cpp}
class foo
{
    int         i;
    char        c;
    std::string s;
public:
    foo(int const i = 0, char const c = 0, std::string const & s = {}):
        i(i), c(c), s(s)
    {}
    foo(foo const &) = default;
    foo& operator=(foo const &) = default;
    bool operator==(foo const & rhv) const
    {
        return i == rhv.i &&
               c == rhv.c &&
               s == rhv.s;
    }
    bool operator!=(foo const & rhv) const
    {
        return !(*this == rhv);
    }
};
struct foopod
{
    bool a;
    char b;
    int  c[2];
};
bool operator==(foopod const & f1, foopod const & f2)
{
    return f1.a == f2.a && f1.b == f2.b &&
           f1.c[0] == f2.c[0] && f1.c[1] == f2.c[1];
}
\end{cpp}


\mySubsubsection{}{How to do it...}

对于不包含指针的POD类型,可以使用ofstream::write()和ifstream::read()来序列化/反序列化:

\begin{itemize}
\item
使用ofstream和write()方法将实例序列化到二进制文件:

\begin{cpp}
std::vector<foopod> output {
    {true, '1', {1, 2}},
    {true, '2', {3, 4}},
    {false, '3', {4, 5}}
};
std::ofstream ofile("sample.bin", std::ios::binary);
if(ofile.is_open())
{
    for(auto const & value : output)
    {
        ofile.write(reinterpret_cast<const char*>(&value),
        sizeof(value));
    }
    ofile.close();
}
\end{cpp}

\item
使用ifstream和read()方法从二进制文件反序列化实例:

\begin{cpp}
std::vector<foopod> input;
std::ifstream ifile("sample.bin", std::ios::binary);
if(ifile.is_open())
{
    while(true)
    {
        foopod value;
        ifile.read(reinterpret_cast<char*>(&value), sizeof(value));
        if(ifile.fail() || ifile.eof()) break;
        input.push_back(value);
    }
    ifile.close();
}
\end{cpp}

\end{itemize}

对于非POD类型(或包含指针的POD类型),必须显式地将数据成员的值写入文件。为了反序列化,必须按相同的顺序从文件读取到数据成员:

\begin{itemize}
\item
添加一个名为write()的成员函数来序列化此类型实例。该方法接受一个ofstream的引用并返回一个布尔值,指示操作是否成功:

\begin{cpp}
bool write(std::ofstream& ofile) const
{
    ofile.write(reinterpret_cast<const char*>(&i), sizeof(i));
    ofile.write(&c, sizeof(c));
    auto size = static_cast<int>(s.size());
    ofile.write(reinterpret_cast<char*>(&size), sizeof(size));
    ofile.write(s.data(), s.size());
    return !ofile.fail();
}
\end{cpp}

\item
添加一个名为read()的成员函数来反序列化此类型实例。该方法接受一个ifstream的引用并返回一个布尔值,指示操作是否成功:

\begin{cpp}
bool read(std::ifstream& ifile)
{
    ifile.read(reinterpret_cast<char*>(&i), sizeof(i));
    ifile.read(&c, sizeof(c));
    auto size {0};
    ifile.read(reinterpret_cast<char*>(&size), sizeof(size));
    s.resize(size);
    ifile.read(reinterpret_cast<char*>(&s.front()), size);
    return !ifile.fail();
}
\end{cpp}
\end{itemize}

除了前面展示的write()和read()成员函数之外,还可以重载\verb|operator<<|和\verb|operator>>|,应执行以下步骤:

\begin{enumerate}
\item
向要序列化/反序列化的类(为foo类)添加非成员\verb|operator<<|和\verb|operator>>|的友元声明:

\begin{cpp}
friend std::ofstream& operator<<(std::ofstream& ofile, foo const& f);
friend std::ifstream& operator>>(std::ifstream& ifile, foo& f);
\end{cpp}

\item
重载\verb|operator<<|:

\begin{cpp}
std::ofstream& operator<<(std::ofstream& ofile, foo const& f)
{
    ofile.write(reinterpret_cast<const char*>(&f.i),
                sizeof(f.i));
    ofile.write(&f.c, sizeof(f.c));
    auto size = static_cast<int>(f.s.size());
    ofile.write(reinterpret_cast<char*>(&size), sizeof(size));
    ofile.write(f.s.data(), f.s.size());
    return ofile;
}
\end{cpp}

\item
重载\verb|operator>>|:

\begin{cpp}
std::ifstream& operator>>(std::ifstream& ifile, foo& f)
{
    ifile.read(reinterpret_cast<char*>(&f.i), sizeof(f.i));
    ifile.read(&f.c, sizeof(f.c));
    auto size {0};
    ifile.read(reinterpret_cast<char*>(&size), sizeof(size));
    f.s.resize(size);
    ifile.read(reinterpret_cast<char*>(&f.s.front()), size);
    return ifile;
}
\end{cpp}
\end{enumerate}

\mySubsubsection{}{Hot it works...}

无论是序列化整个对象(对于POD类型)还是只序列化其部分,都使用了在前一个示例中讨论的相同流类型:ofstream用于输出文件流,ifstream用于输入文件流。

将对象序列化和反序列化到文件中时,应避免将指针的值写入文件。此外,不应该从文件中读取指针值,这些值代表的是内存地址,在不同的进程中,甚至在同一进程的不同时刻都没有意义。相反,应该写入指针所指向的数据,并将数据读入指针所指向的对象中。

这是一般原则,实践中可能会遇到源中有多个指针指向同一个对象的情况;这时,可以只写入一份副本,并以相应的方式处理读取过程。

如果想要序列化的对象是POD类型,可以像讨论原始数据时所做的那样进行。这个食谱的例子中,序列化了一系列foo类型的实例。当反序列化时,从文件流中循环读取,直到读取到文件末尾或发生失败。这时,读取方式可能看起来有些违反直觉,但如果以其他方式操作可能会导致最后一次读取重复的值:

\begin{enumerate}
\item
读取在一个无限循环中完成。

\item
在循环中执行一次读取操作。

\item
检查是否发生失败或到达文件末尾,如果其中任意一种情况发生,则退出无限循环。

\item
将值添加到输入序列中,继续循环。
\end{enumerate}

使用\verb|while(!ifile.eof())|这样的循环退出条件来读取文件末尾标记,最后一个值将添加到输入序列两次。读取最后一个值时,文件末尾尚未遇到(文件末尾是一个位于文件最后一个字节之后的标记)。只有在下一次读取尝试时才会达到文件末尾,这将设置流的EOF位。但输入变量仍然持有最后一个值,因为其没有包含,因此这个值会第二次添加到输入vector中。

如果想要序列化和反序列化的对象是非POD类型,不可能将这些对象作为原始数据写入/读取,例如:这样的对象可能有一个虚表。虽然将虚表写入文件不会造成问题,即使没有任何价值,但从文件中读取,从而覆盖实例的虚表,将对实例和程序产生灾难性的影响。

当序列化/反序列化非POD类型时,有各种各样的选择,其中一些已经在前一节中讨论过,都提供了显式的方法进行写入和读取或重载标准的\verb|<<|和\verb|>>|操作符。第二种方法的优点在于,允许在通用代码中使用类型,通过这些操作符将对象写入和从流文件中读出。

\begin{myNotic}
当计划序列化和反序列化对象时,从一开始就考虑数据版本化,以避免随着时间推移数据结构变化而引起的问题。如何进行版本化超出了本示例的范畴。
\end{myNotic}

