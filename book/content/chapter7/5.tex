除了基于流的输入输出库之外,标准库还提供了一系列称为操纵符的辅助函数,这些函数通过使用操作符 <{}< 和 >{}> 控制输入和输出流。本示例中,将介绍一些这些操纵符,并通过一些格式化控制台输出的例子展示其用法。将在接下来的示例中,继续介绍更多的操纵符。

\mySubsubsection{}{Getting ready}

I/O操纵符位于 <ios>、<istream>、<ostream> 和 <iomanip> 头文件中的 std 命名空间下。这个示例中,将只讨论来自 <ios> 和 <iomanip> 的一些操纵符。

\mySubsubsection{}{How to do it...}

以下操纵符可用于控制流的输入或输出:

\begin{itemize}
\item
boolalpha 和 noboolalpha 启用和禁用布尔值的文字表示形式:

\begin{cpp}
std::cout << std::boolalpha << true << '\n';    // true
std::cout << false << '\n';                     // false
std::cout << std::noboolalpha << false << '\n'; // 0
\end{cpp}

\item
left、right 和 internal 影响填充字符的对齐方式;left 和 right 影响所有文本,但 internal 只影响整数、浮点数和货币输出:

\begin{cpp}
std::cout << std::right << std::setw(10) << "right\n";
std::cout << std::setw(10) << "text\n";
std::cout << std::left << std::setw(10) << "left\n";
\end{cpp}

\item
fixed、scientific、hexfloat 和 defaultfloat 改变用于浮点类型(对于输入和输出流)的格式化方式。后两种自C++11起可用:

\begin{cpp}
std::cout << std::fixed << 0.25 << '\n';         // 0.250000
std::cout << std::scientific << 0.25 << '\n';    // 2.500000e-01
std::cout << std::hexfloat << 0.25 << '\n';      // 0x1p-2
std::cout << std::defaultfloat << 0.25 << '\n';  // 0.25
\end{cpp}

\item
dec、hex 和 oct 控制整数类型(对于输入和输出流)使用的基数:

\begin{cpp}
std::cout << std::oct << 42 << '\n'; // 52
std::cout << std::hex << 42 << '\n'; // 2a
std::cout << std::dec << 42 << '\n'; // 42
\end{cpp}

\item
setw 改变下一个输入或输出字段的宽度。默认宽度为0。

\item
setfill 改变输出流的填充字符;这是用来填充直到达到指定宽度为止的下一个字段的字符。默认填充字符为空格:

\begin{cpp}
std::cout << std::right
          << std::setfill('.') << std::setw(10)
          << "right" << '\n';
// .....right
\end{cpp}

\item
setprecision 改变输入和输出流中浮点类型的十进制精度(生成多少位数字)。默认精度为6:

\begin{cpp}
std::cout << std::fixed << std::setprecision(2) << 12.345 << '\n';
// 12.35
\end{cpp}
\end{itemize}

\mySubsubsection{}{Hot it works...}

上述列出的所有I/O操纵符(除了 setw,仅指下一个输出字段)都会影响流,所有连续的读写操作都会使用最后一次指定的格式,直到再次使用另一个操纵符。

一些操纵符是无参调用的。例如 boolalpha/noboolalpha 或 dec/hex/oct。这些操纵符是接受单个参数(即,字符串的引用),并返回相同流的引用函数:

\begin{cpp}
std::ios_base& hex(std::ios_base& str);
\end{cpp}

\verb|std::cout << std::hex| 这样的表达式之所以可能,是因为 \verb|basic_ostream::operator<<| 和 \verb|basic_istream::operator>>| 都有重载来接受这些函数的指针。

其他操纵符会带有参数,这些操纵符接受一个或多个参数,并返回一个未指定类型实例的函数:

\begin{cpp}
template<class CharT>
/*unspecified*/ setfill(CharT c);
\end{cpp}

为了更好地展示这些操纵符的使用,来看两个格式化控制台输出的例子。

第一个例子中,将列出一本书的目录,有如下要求:

\begin{itemize}
\item
章节编号右对齐并用罗马数字表示。

\item
章节标题左对齐,直到页码之间的空白用点填充。

\item
章节的页码右对齐。
\end{itemize}

例子中,将使用以下类型和辅助函数:

\begin{cpp}
struct Chapter
{
    int Number;
    std::string Title;
    int Page;
};
struct BookPart
{
    std::string Title;
    std::vector<Chapter> Chapters;
};
struct Book
{
    std::string Title;
    std::vector<BookPart> Parts;
};
std::string to_roman(unsigned int value)
{
    struct roman_t { unsigned int value; char const* numeral; };
    const static roman_t rarr[13] =
    {
        {1000, "M"}, {900, "CM"}, {500, "D"}, {400, "CD"},
        { 100, "C"}, { 90, "XC"}, { 50, "L"}, { 40, "XL"},
        {  10, "X"}, {  9, "IX"}, {  5, "V"}, {  4, "IV"},
        {   1, "I"}
    };
    std::string result;
    for (auto const & number : rarr)
    {
        while (value >= number.value)
        {
            result += number.numeral;
            value -= number.value;
        }
    }
    return result;
}
\end{cpp}

如下面的代码所示,\verb|print_toc()| 函数接收一个 Book 对象作为参数,并根据指定的要求将其内容输出到控制台上:

\begin{itemize}
\item
std::left 和 std::right 指定文本对齐方式

\item
std::setw 指定每个输出字段的宽度

\item
std::fill 指定填充字符(章节编号为空格,章节标题为点)
\end{itemize}

\verb|print_toc()|函数的实现如下:

\begin{cpp}
void print_toc(Book const & book)
{
    std::cout << book.Title << '\n';
    for(auto const & part : book.Parts)
    {
        std::cout << std::left << std::setw(15) << std::setfill(' ')
                  << part.Title << '\n';
        std::cout << std::left << std::setw(15) << std::setfill('-')
                  << '-' << '\n';
        for(auto const & chapter : part.Chapters)
        {
            std::cout << std::right << std::setw(4) << std::setfill(' ')
                      << to_roman(chapter.Number) << ' ';
            std::cout << std::left << std::setw(35) << std::setfill('.')
                      << chapter.Title;
            std::cout << std::right << std::setw(3) << std::setfill('.')
                      << chapter.Page << '\n';
        }
    }
}
\end{cpp}

下面的例子使用了《魔戒同盟》目录的 Book 类型实例调用了此方法:

\begin{cpp}
auto book = Book
{
    "THE FELLOWSHIP OF THE RING"s,
    {
        {
            "BOOK ONE"s,
            {
                {1, "A Long-expected Party"s, 21},
                {2, "The Shadow of the Past"s, 42},
                {3, "Three Is Company"s, 65},
                {4, "A Short Cut to Mushrooms"s, 86},
                {5, "A Conspiracy Unmasked"s, 98},
                {6, "The Old Forest"s, 109},
                {7, "In the House of Tom Bombadil"s, 123},
                {8, "Fog on the Barrow-downs"s, 135},
                {9, "At the Sign of The Prancing Pony"s, 149},
                {10, "Strider"s, 163},
                {11, "A Knife in the Dark"s, 176},
                {12, "Flight to the Ford"s, 197},
            },
        },
        {
            "BOOK TWO"s,
            {
                {1, "Many Meetings"s, 219},
                {2, "The Council of Elrond"s, 239},
                {3, "The Ring Goes South"s, 272},
                {4, "A Journey in the Dark"s, 295},
                {5, "The Bridge of Khazad-dum"s, 321},
                {6, "Lothlorien"s, 333},
                {7, "The Mirror of Galadriel"s, 353},
                {8, "Farewell to Lorien"s, 367},
                {9, "The Great River"s, 380},
                {10, "The Breaking of the Fellowship"s, 390},
            },
        },
    }
};
print_toc(book);
\end{cpp}

输出如下所示:

\begin{shell}
THE FELLOWSHIP OF THE RING
BOOK ONE
---------------
   I A Long-expected Party...............21
  II The Shadow of the Past..............42
 III Three Is Company....................65
  IV A Short Cut to Mushrooms............86
   V A Conspiracy Unmasked...............98
  VI The Old Forest.....................109
 VII In the House of Tom Bombadil.......123
VIII Fog on the Barrow-downs............135
  IX At the Sign of The Prancing Pony...149
   X Strider............................163
  XI A Knife in the Dark................176
 XII Flight to the Ford.................197
BOOK TWO
---------------
   I Many Meetings......................219
  II The Council of Elrond..............239
 III The Ring Goes South................272
  IV A Journey in the Dark..............295
   V The Bridge of Khazad-dum...........321
  VI Lothlorien.........................333
 VII The Mirror of Galadriel............353
VIII Farewell to Lorien.................367
  IX The Great River....................380
   X The Breaking of the Fellowship.....390
\end{shell}

第二个例子中,目标是输出一个表格,列出世界上按收入最大的公司。该表将包括公司名称、行业、收入(以数十亿美元计)、收入增长的增减情况、增长率、员工人数和原籍国家等。这个例子中,将使用以下类型:

\begin{cpp}
struct Company
{
    std::string Name;
    std::string Industry;
    double      Revenue;
    bool        RevenueIncrease;
    double      Growth;
    int         Employees;
    std::string Country;
};
\end{cpp}

下面代码中的 \verb|print_companies()| 函数使用了比上一个例子更多的操纵符:

\begin{itemize}
\item
std::boolalpha 以 true 和 false 而非 1 和 0 显示布尔值。

\item
std::fixed 指示固定的小数点表示形式,std::defaultfloat 恢复默认小数点表示形式。

\item
std::setprecision 指定输出中显示的小数位数。与 std::fixed 结合使用,可以指示增长率字段的固定表示形式,并保留一位小数。
\end{itemize}

\verb|print_companies()| 函数的实现:

\begin{cpp}
void print_companies(std::vector<Company> const & companies)
{
    for(auto const & company : companies)
    {
        std::cout << std::left << std::setw(26) << std::setfill(' ')
                  << company.Name;
        std::cout << std::left << std::setw(18) << std::setfill(' ')
                  << company.Industry;
        std::cout << std::left << std::setw(5) << std::setfill(' ')
                  << company.Revenue;
        std::cout << std::left << std::setw(5) << std::setfill(' ')
                  << std::boolalpha << company.RevenueIncrease
                  << std::noboolalpha;
        std::cout << std::right << std::setw(5) << std::setfill(' ')
                  << std::fixed << std::setprecision(1) << company.Growth
                  << std::defaultfloat << std::setprecision(6) << ' ';
        std::cout << std::right << std::setw(8) << std::setfill(' ')
                  << company.Employees << ' ';
        std::cout << std::left << std::setw(2) << std::setfill(' ')
                  << company.Country
                  << '\n';
    }
}
\end{cpp}

下面是调用此方法的例子。这里显示的数据来源是维基百科(\url{https://en.wikipedia.org/wiki/List_of_largest_companies_by_revenue},截至2016年):

\begin{cpp}
std::vector<Company> companies
{
    {"Walmart"s, "Retail"s, 482, false, 0.71,
        2300000, "US"s},
    {"State Grid"s, "Electric utility"s, 330, false, 2.91,
        927839, "China"s},
    {"Saudi Aramco"s, "Oil and gas"s, 311, true, 40.11,
        65266, "SA"s},
    {"China National Petroleum"s, "Oil and gas"s, 299,
        false, 30.21, 1589508, "China"s},
    {"Sinopec Group"s, "Oil and gas"s, 294, false, 34.11,
        810538, "China"s},
};
print_companies(companies);
\end{cpp}

输出具有基于表格的格式:

\begin{shell}
Walmart                   Retail            482  false  0.7  2300000 US
State Grid                Electric utility  330  false  2.9   927839 China
Saudi Aramco              Oil and gas       311  true  40.1    65266 SA
China National Petroleum  Oil and gas       299  false 30.2  1589508 China
Sinopec Group             Oil and gas       294  false 34.1   810538 China
\end{shell}

作为一个练习,可以尝试添加表头或甚至是一条网格线来前置这些行,以便更好地排列数据。

