文件系统库提供了函数和类型,使开发人员能够检查文件系统对象(如文件或目录)的存在性、属性(如类型、最后写入时间、权限等)。本示例中,将介绍这些类型和函数,及如何使用。

\mySubsubsection{}{Getting ready}

下面的代码示例,可使用 \verb|fs| 作为 \verb|std::filesystem| 命名空间的别名。文件系统库在名为 \verb|<filesystem>| 的头文件中可用,使用变量 \verb|path| 表示文件路径,\verb|err| 用于接收来自文件系统API的错误代码:

\begin{cpp}
auto path = fs::current_path() / "main.cpp";
auto err = std::error_code{};
\end{cpp}

另外,这个示例中将引用以下所示的 \verb|to_time_t| 函数:

\begin{cpp}
template <typename TP>
std::time_t to_time_t(TP tp)
{
    using namespace std::chrono;
    auto sctp = time_point_cast<system_clock::duration>(
        tp - TP::clock::now() + system_clock::now());
    return system_clock::to_time_t(sctp);
}
\end{cpp}

\mySubsubsection{}{How to do it...}

使用以下库函数检索文件系统实例的信息:

\begin{itemize}
\item
要检查路径是否指代现有的文件系统实例,可使用 \verb|exists()|:

\begin{cpp}
auto exists = fs::exists(path, err);
std::cout << "file exists: " << std::boolalpha
          << exists << '\n';
\end{cpp}

\item
要检查两个不同的路径是否指代同一个文件系统实例,可使用 \verb|equivalent()|:

\begin{cpp}
auto same = fs::equivalent(path,
                fs::current_path() / "." / "main.cpp", err);
std::cout << "equivalent: " << same << '\n';
\end{cpp}

\item
要检索文件的字节大小,使用 \verb|file_size()|。这不需要打开文件,所以应该优先于先打开文件再使用 \verb|seekg()|/\verb|tellg()| 函数的方法:

\begin{cpp}
auto size = fs::file_size(path, err);
std::cout << "file size: " << size << '\n';
\end{cpp}

\item
要检索指向文件系统对象的硬链接数量,可使用 \verb|hard_link_count()|:

\begin{cpp}
auto links = fs::hard_link_count(path, err);
if(links != static_cast<uintmax_t>(-1))
    std::cout << "hard links: " << links << '\n';
else
    std::cout << "hard links: error" << '\n';
\end{cpp}

\item
要检索或设置文件系统对象的最后修改时间,可使用 \verb|last_write_time()|:

\begin{cpp}
auto lwt = fs::last_write_time(path, err);
auto time = to_time_t(lwt);
auto localtime = std::localtime(&time);
std::cout << "last write time: "
          << std::put_time(localtime, "%c") << '\n';
\end{cpp}

\item
要检索文件属性,如类型和权限(类似于POSIX的stat函数返回的结果),可使用 \verb|status()| 函数。此函数会跟随符号链接。要检索符号链接本身的文件属性而不跟随,可使用 \verb|symlink_status()|:

\begin{cpp}
auto print_perm = [](fs::perms p)
{
    std::cout
        << ((p & fs::perms::owner_read) != fs::perms::none ?
            "r" : "-")
        << ((p & fs::perms::owner_write) != fs::perms::none ?
            "w" : "-")
        << ((p & fs::perms::owner_exec) != fs::perms::none ?
            "x" : "-")
        << ((p & fs::perms::group_read) != fs::perms::none ?
            "r" : "-")
        << ((p & fs::perms::group_write) != fs::perms::none ?
            "w" : "-")
        << ((p & fs::perms::group_exec) != fs::perms::none ?
            "x" : "-")
        << ((p & fs::perms::others_read) != fs::perms::none ?
            "r" : "-")
        << ((p & fs::perms::others_write) != fs::perms::none ?
            "w" : "-")
        << ((p & fs::perms::others_exec) != fs::perms::none ?
            "x" : "-")
        << '\n';
};
auto status = fs::status(path, err);
std::cout << "type: " << static_cast<int>(status.type()) << '\n';
std::cout << "permissions: ";
print_perm(status.permissions());
\end{cpp}

\item
要检查路径是否指代特定类型的文件系统实例,如文件、目录、符号链接等,使用 \verb|is_regular_file()|、\verb|is_directory()|、\verb|is_symlink()| 等函数:

\begin{cpp}
std::cout << "regular file? " <<
          fs::is_regular_file(path, err) << '\n';
std::cout << "directory? " <<
          fs::is_directory(path, err) << '\n';
std::cout << "char file? " <<
          fs::is_character_file(path, err) << '\n';
std::cout << "symlink? " <<
          fs::is_symlink(path, err) << '\n';
\end{cpp}

\item
要检查文件或目录是否为空,使用 \verb|is_empty()| 函数:

\begin{cpp}
bool empty = fs::is_empty(path, err);
if (!err)
{
    std::cout << std::boolalpha
    << "is_empty(): " << empty << '\n';
}
\end{cpp}
\end{itemize}

\mySubsubsection{}{Hot it works...}

这些用于检索文件系统文件和目录信息的函数一般来说简单直接,也需要注意一些事项:

\begin{itemize}
\item
检查文件系统对象是否存在可以使用 \verb|exists()| 函数,通过传递路径或先前使用 \verb|status()| 函数检索的 \verb|std::filesystem::file_status| 实例来完成。

\item
\verb|equivalent()| 函数确定两个文件系统对象的状态是否相同,状态由 \verb|status()| 函数检索。如果两个路径都不存在,或者都存在但都不是文件、目录或符号链接,则该函数将返回错误。指向同一文件对象的硬链接等价,符号链接及其目标也等价。

\item
\verb|file_size()| 函数只能用来确定常规文件和指向常规文件的符号链接的大小。对于其他类型的文件实例,如目录,此函数会失败。该函数返回文件的字节数,如果发生错误则返回 -1。如果想确定文件是否为空,可以使用 \verb|is_empty()| 函数。这对所有类型的文件系统对象有效,包括目录。

\item
\verb|last_write_time()| 函数有两种重载:一种用于检索文件系统对象的最后修改时间,另一种用于设置最后修改时间。时间由 \verb|std::filesystem::file_time_type| 实例表示,基本上是 \verb|std::chrono::time_point| 类型的别名。以下示例将文件的最后写入时间改为比其先前值早30分钟:

\begin{cpp}
using namespace std::chrono_literals;
auto lwt = fs::last_write_time(path, err);
fs::last_write_time(path, lwt - 30min);
\end{cpp}

\item
\verb|status()| 函数确定文件系统对象的类型和权限。如果文件是一个符号链接,返回的信息是关于符号链接的目标。要检索符号链接本身的信息,必须使用 \verb|symlink_status()| 函数,这些函数返回一个 \verb|std::filesystem::file_status| 实例。

此类有一个 \verb|type()| 成员函数用于检索文件类型,以及一个 \verb|permissions()| 成员函数用于检索文件权限。文件的类型由 \verb|std::filesystem::file_type| 枚举定义。文件的权限由 \verb|std::filesystem::perms| 枚举定义。枚举中的不是所有枚举器都代表权限;有些代表控制位,如 \verb|add_perms| 表示应添加权限,或 \verb|remove_perms| 表示应删除权限。可以使用 \verb|permissions()| 函数修改文件或目录的权限。以下示例向文件的所有者和用户组添加所有权限:

\begin{cpp}
fs::permissions(
    path,
    fs::perms::add_perms |
    fs::perms::owner_all | fs::perms::group_all,
    err);
\end{cpp}

\item
要确定文件系统对象的类型,如文件、目录或符号链接,有两个选项:检索文件状态然后检查类型属性,或者使用可用的文件系统函数,如 \verb|is_regular_file()|、\verb|is_symlink()| 或 \verb|is_directory()|。以下检查路径是否指代普通文件的示例等效:

\begin{cpp}
auto s = fs::status(path, err);
auto isfile = s.type() == std::filesystem::file_type::regular;
auto isfile = fs::is_regular_file(path, err);
\end{cpp}
\end{itemize}

本示例中讨论的所有函数,都有一个重载会在发生错误时抛出异常,以及不会抛出异常但通过函数参数返回错误代码的重载。虽然本示例中的代码未显示(为了简化),但要检查这些函数返回的 \verb|error_code| 值。每个值的实际含义取决于返回调用和所属的值型类别(如系统、I/O流或通用),在所有值型类别中,值0认为表示成功:

\begin{cpp}
auto lwt = fs::last_write_time(path, err);
if (!err) // success
{
    auto time = to_time_t(lwt);
    auto localtime = std::localtime(&time);
    std::cout << "last write time: "
              << std::put_time(localtime, "%c") << '\n';
}
\end{cpp}

如果使用的是不返回错误代码但抛出异常的重载,就需要捕获可能发生的异常:

\begin{cpp}
try
{
    auto exists = fs::exists(path);
    std::cout << "file exists: " << std::boolalpha << exists << '\n';
}
catch (std::filesystem::filesystem_error const& ex)
{
    std::cerr << ex.what() << '\n';
}
\end{cpp}


