

当涉及到使能数据类型反射或内省的功能,或者标准机制来定义语言扩展时,C++ 曾经欠缺这方面的能力。因此,编译器为这些目的定义了自己的特定扩展,例如:VC++ 的 \verb|__declspec()| 说明符和 GCC 的 \verb|__attribute__((...))|。然而,C++11 引入了属性的概念,使得编译器能够以标准方式实现扩展,甚至嵌入领域特定语言。更新的 C++ 标准定义了所有编译器都应该实现的若干属性,而这将是本示例的主题。


\mySubsubsection{}{How to do it...}

使用标准属性来为编译器提供关于各种设计目标的提示,如以下场景所示,但不限于这些:

\begin{itemize}
\item
为了确保不能忽略函数的返回值,用 \verb|[[nodiscard]]| 属性声明该函数。C++20 中,可以指定一个形如 \verb|[[nodiscard(text)]]| 的字符串字面量来解释为什么结果不应该被丢弃:

\begin{cpp}
[[nodiscard]] int get_value1()
{
    return 42;
}
get_value1();
// warning: ignoring return value of function
//          declared with 'nodiscard' attribute get_value1();
\end{cpp}

\item
或者,可以用 \verb|[[nodiscard]]| 属性声明作为函数返回类型的枚举和类型;返回此类类型的任何函数的返回值都不能忽略:

\begin{cpp}
enum class[[nodiscard]] ReturnCodes{ OK, NoData, Error };
ReturnCodes get_value2()
{
    return ReturnCodes::OK;
}
struct[[nodiscard]] Item{};
Item get_value3()
{
    return Item{};
}
// warning: ignoring return value of function
//          declared with 'nodiscard' attribute
get_value2();
get_value3();
\end{cpp}

\item
为了确保编译器以警告标志认为是过时的函数或类型的使用,用 \verb|[[deprecated]]| 属性声明:

\begin{cpp}
[[deprecated("Use func2()")]] void func()
{
}
// warning: 'func' is deprecated : Use func2()
func();
class [[deprecated]] foo
{
};
// warning: 'foo' is deprecated
foo f;
\end{cpp}

\item
为了确保编译器不对未使用的变量发出警告,使用 \verb|[[maybe_unused]]| 属性:

\begin{cpp}
double run([[maybe_unused]] int a, double b)
{
    return 2 * b;
}
[[maybe_unused]] auto i = get_value1();
\end{cpp}

\item
为了确保编译器不对 switch 语句中故意的穿透情况标签发出警告,使用 \verb|[[fallthrough]]| 属性:

\begin{cpp}
void option1() {}
void option2() {}
int alternative = get_value1();
switch (alternative)
{
    case 1:
    option1();
    [[fallthrough]]; // this is intentional
    case 2:
    option2();
}
\end{cpp}

\item
为了帮助编译器优化更有可能或不太可能执行的执行路径,使用 C++20 的 \verb|[[likely]]| 和 \verb|[[unlikely]]| 属性:

\begin{cpp}
void execute_command(char cmd)
{
    switch(cmd)
    {
        [[likely]]
        case 'a': /* add */ break;
        [[unlikely]]
        case 'd': /* delete */ break;
        case 'p': /* print */ break;
        default:  /* do something else */ break;
    }
}
\end{cpp}

\item
为了帮助编译器根据用户给定的假设优化代码,使用 C++23 的 \verb|[[assume]]| 属性:

\begin{cpp}
void process(int* data, size_t len)
{
    [[assume(len > 0)]];
    for(size_t i = 0; i < len; ++i)
    {
        // do something with data[i]
    }
}
\end{cpp}
\end{itemize}

\mySubsubsection{}{How it works...}

属性是 C++ 的一个非常灵活的特性;几乎可以在任何地方使用,但每个特定属性的实际使用都是具体定义的。可以用于类型、函数、变量、名称、代码块或整个转换单元。

属性是在双方括号之间指定的(例如,\verb|[[attr1]]|),并且可以在一个声明中指定多个属性(例如,\verb|[[attr1, attr2, attr3]]|)。

属性可以有参数,例如 \verb|[[mode(greedy)]]|;也可以完全限定,例如 \verb|[[sys::hidden]]| 或 \verb|[[using sys: visibility(hidden), debug]]|。

属性可以出现在它们应用的实体名称之前或之后,或者两者都有,这种情况下它们会进行合并。以下是几个示例,举例说明了这一点:

\begin{cpp}
// attr1 applies to a, attr2 applies to b
int a [[attr1]], b [[attr2]];
// attr1 applies to a and b
int [[attr1]] a, b;
// attr1 applies to a and b, attr2 applies to a
int [[attr1]] a [[attr2]], b;
\end{cpp}

属性不能出现在命名空间声明中,但可以在命名空间中的任何地方作为单行声明出现。这时,每个属性是否应用于随后的声明、命名空间或转换单元是特定的:

\begin{cpp}
namespace test
{
    [[debug]];
}
\end{cpp}

标准确实定义了几种所有编译器都必须实现的属性,可以写出更好的代码。前一节给出的例子中已经看到了其中的一些。这些属性是在不同版本的标准中定义的:

\begin{itemize}
\item
C++11:

\begin{itemize}
\item
\verb|[[noreturn]]| 属性表明一个函数不会返回。

\item
\verb|[[carries_dependency]]| 属性表明内存序 \verb|std::memory_order| 中的依赖链会在函数内外传播,允许编译器跳过不必要的内存栅栏指令。
\end{itemize}

\item
C++14:

\begin{itemize}
\item
\verb|[[deprecated]]| 和 \verb|[[deprecated("原因")]]| 属性表明用这些属性声明的实体是过时的,不应使用。这些属性可以用于类、非静态数据成员、typedef、函数、枚举和模板特化。“原因”字符串是一个可选参数。
\end{itemize}

\item
C++17:

\begin{itemize}
\item
\verb|[[fallthrough]]| 属性表明 switch 语句中标签之间的穿透是有意为之。该属性必须单独出现在 case 标签之前的一行上。

\item
\verb|[[nodiscard]]| 属性表明不能忽略函数的返回值。

\item
\verb|[[maybe_unused]]| 属性表明一个实体可能未使用,但编译器不应为此发出警告。此属性可以应用于变量、类、非静态数据成员、枚举、枚举值和 typedef。
\end{itemize}

\item
C++20:

\begin{itemize}
\item
\verb|[[nodiscard(text)]]| 属性是 C++17 \verb|[[nodiscard]]| 属性的扩展,提供了描述为什么结果不应该丢弃的文本。

\item
\verb|[[likely]]| 和 \verb|[[unlikely]]| 属性向编译器提供提示,表明一条执行路径更有可能或不太可能执行,从而允许编译器相应地优化。可以应用于语句(但不是声明)和标签,但两者互斥,只能应用其中一个。

\item
\verb|[[no_unique_address]]| 属性可以应用于非静态数据成员(不包括位字段),告诉编译器该成员不必具有唯一地址。当应用于具有空类型成员时,编译器可以优化它,使其不占用空间,就像它是空基类一样。另一方面,如果成员的类型不为空,编译器可能会重用来存储其他数据成员。
\end{itemize}

\item
C++23:

\begin{itemize}
\item
\verb|[[assume(expr)]]| 表明一个表达式将始终评估为真,其目的是使编译器能够进行代码优化,而不是记录函数的前提条件。但表达式永远不会进行评估,一个具有未定义行为或抛出异常的表达式将评估为假。一个不成立的假设会导致未定义行为,所以应该谨慎使用assume。另一方面,编译器也可能根本不使用assume。
\end{itemize}

\end{itemize}

属性通常现代 C++ 编程书籍和教程忽视或简要提及,原因可能是开发者实际上并不能编写属性,因为该语言特性是为了编译器实现而设计的。不过,对于某些编译器来说,可以定义用户提供的属性;一个这样的编译器是 GCC,支持添加功能到编译器的插件,这些插件也可以用来定义新的属性。

