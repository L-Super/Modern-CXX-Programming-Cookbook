
前面的示例中,介绍了如何使用 \verb|static_assert| 和 \verb|std::enable_if| 对类型和函数施加限制,以及这两个方法的不同之处。在定义函数重载或编写变参函数模板时使用 SFINAE 和 \verb|std::enable_if|,模板元编程可能会变得复杂和混乱。C++17 的新特性旨在简化这样的代码,称为 constexpr if,其定义了一个在编译时评估条件的 if 语句,导致编译器在转换单元中选择分支的一个体或另一个体。constexpr if 的典型用途是简化变参模板和基于 \verb|std::enable_if| 的代码。

\mySubsubsection{}{Getting ready}

这个示例中,会参考并简化前两个示例中的代码。继续这个示例之前,应该花点时间回顾一下前几个示例:

\begin{itemize}
\item
使用 \verb|enable_if| 条件编译类和函数示例中的 \verb|compute()| 函数重载,针对整型和非整型类型。

\item
第2章中创建原始用户定义字面量示例中的用户定义 8 位、16 位和 32 位二进制字面量。
\end{itemize}

这些实现有几个问题:

\begin{itemize}
\item
难以阅读。很多注意力都集中在模板声明上,而函数体却非常简单。最大的问题是需要开发人员更多的注意力,因为其充斥着复杂的声明,比如 \verb|typename = std::enable_if<std::is_integral<T>::value, T>::type|。

\item
代码太多。第一个例子的最终目的是拥有一个对不同类型行为不同的通用函数,但需要为这个函数编写两个重载;此外,为了区分这两个重载,需要使用一个额外的未使用的模板参数。第二个例子中,目的是从字符 '0' 和 '1' 构建一个整数值,但需要编写一个类模板和三个特化来实现。

\item
要求具备高级的模板元编程技能,而这些技能对于完成如此简单的任务来说完全没必要。
\end{itemize}

constexpr if 的语法与常规的 if 语句非常相似,需要在条件前加上 constexpr 关键字:

\begin{cpp}
if constexpr (init-statement condition) statement-true
else statement-false
\end{cpp}

注意,\verb|初始化语句| 可选。

接下来的部分中,将探讨使用 constexpr if 进行条件编译的几种用例。

\mySubsubsection{}{How to do it...}

使用 constexpr if 语句来做到以下几点:

\begin{itemize}
\item
避免使用 \verb|std::enable_if| 并依赖 SFINAE 来对函数模板类型施加限制和有条件地编译代码:

\begin{cpp}
template <typename T>
auto value_of(T value)
{
    if constexpr (std::is_pointer_v<T>)
        return *value;
    else
        return value;
}
\end{cpp}

\item
简化变参模板的编写并实现编译时递归的元编程:

\begin{cpp}
namespace binary
{
    using byte8 = unsigned char;
    namespace binary_literals
    {
        namespace binary_literals_internals
        {
            template <typename CharT, char d, char... bits>
            constexpr CharT binary_eval()
            {
                if constexpr(sizeof...(bits) == 0)
                return static_cast<CharT>(d-'0');
                else if constexpr(d == '0')
                return binary_eval<CharT, bits...>();
                else if constexpr(d == '1')
                return static_cast<CharT>(
                (1 << sizeof...(bits)) |
                binary_eval<CharT, bits...>());
            }
        }
        template<char... bits>
        constexpr byte8 operator""_b8()
        {
            static_assert(
            sizeof...(bits) <= 8,
            "binary literal b8 must be up to 8 digits long");
            return binary_literals_internals::
            binary_eval<byte8, bits...>();
        }
    }
}
\end{cpp}
\end{itemize}

\mySubsubsection{}{How it works...}

constexpr if 的工作方式相对简单:if 语句中的条件必须是一个编译时常量表达式,该表达式可以计算为布尔值或可以转换为布尔值。如果条件为真,则选择 if 语句的主体部分,则将进入转换单元进行编译。如果条件为假,则定义的 else 分支(如果有)。丢弃的 constexpr if 分支中的返回语句不会影响函数返回类型的推导。

“How to do it...”的第一个例子中,\verb|value_of()| 函数模板有一个干净的签名。函数体也非常简单;如果替换模板参数的类型是指针类型,编译器将选择第一个分支(即 \verb|return *value;|)进行代码生成,并丢弃 else 分支。对于非指针类型,因为条件评估为假,编译器将选择 else 分支(即 \verb|return value;|)进行代码生成,并丢弃其余部分。这个函数可以这样使用:

\begin{cpp}
auto v1 = value_of(42);
auto p = std::make_unique<int>(42);
auto v2 = value_of(p.get());
\end{cpp}

如果没有 constexpr if 的帮助,只能使用 \verb|std::enable_if| 来实现。以下是一个杂乱的替代实现:

\begin{cpp}
template <typename T,
          typename = typename std::enable_if_t<std::is_pointer_v<T>, T>>
auto value_of(T value)
{
    return *value;
}
template <typename T,
          typename = typename std::enable_if_t<!std::is_pointer_v<T>, T>>
T value_of(T value)
{
    return value;
}
\end{cpp}

constexpr if 的版本不仅更短,而且更具表现力,更容易阅读和理解。

“How to do it...”的第二个例子中,内部的 \verb|binary_eval()| 辅助函数是一个没有参数的变参模板函数,只有模板参数。该函数评估第一个参数,然后以递归的方式(不是运行时递归)处理其余参数。当只剩下单个字符并且剩余包的大小为 0 时,返回由该字符表示的十进制值(对于 '0' 返回 0,对于 '1' 返回 1)。如果当前第一个元素是 '0',返回由评估其余参数包确定的值,这涉及到一次递归调用。如果当前第一个元素是 '1',可将 1 左移剩余包大小给出的位置数,或由评估其余参数包确定的值来返回值,这同样涉及到一次递归调用。






