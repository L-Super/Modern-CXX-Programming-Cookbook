数字和字符串是编程语言中基本的数据类型,其他类型要么基于这些基本类型,要么由它们组成。开发者经会面对决诸如在数字和字符串之间转换、解析和格式化字符串、生成随机数的任务。本章将介绍如何使用现代C++和库来完成这些任务。

本章包含如下内容:

\begin{itemize}
\item
数值类型

\item
数值类型的极值和属性

\item
数值类型和字符串类型互相转换

\item
字符和字符串类型

\item
将Unicode字符输出到控制台

\item
生成伪随机数

\item
初始化伪随机数生成器

\item
创建预处理自定义字面量

\item
创建原自定义字面量

\item
原字符串字面量避免转义

\item
创建字符串辅助库

\item
正则表达式验证字符串格式

\item
正则表达式解析字符串内容

\item
正则表达式替换字符串内容

\item
使用 std::string\_view 而非常量字符串引用

\item
使用 std::format 和 std::print 格式化和输出文本

\item
使用 std::format 处理自定义的类型
\end{itemize}


