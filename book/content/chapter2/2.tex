
有时候,需要知道并使用数值类型(如 char、int 或 double)能够表示的最小值和最大值。许多开发人员为此目的使用标准 C 宏,例如 CHAR\_MIN/CHAR\_MAX、INT\_MIN/INT\_MAX 和 DBL\_MIN/DBL\_MAX。C++ 提供名为 numeric\_limits 的类模板,为每种数值类型提供了特化版本,可以查询该类型的最小值和最大值。然而,numeric\_limits 的功能不仅限于此,还提供了常量来查询类型属性,比如该类型是否有符号、表示值所需多少位、是否能表示浮点类型的无穷大等。

C++11前,由于 numeric\_limits<T> 不能在需要常量的地方使用(例如:数组的大小或 switch 语句的 case 标签),开发者通常更倾向于在整个代码中使用 C 宏。这种情况在 C++11 中发生了变化,因为 numeric\_limits<T> 的所有静态成员现在都是 constexpr 类型,所以可以在需要常量表达式的地方使用。这一变化增强了 numeric\_limits 的灵活性和实用性,使其成为现代 C++ 编程中处理数值类型属性的首选。

\mySubsubsection{}{Getting ready}

numeric\_limits<T> 类模板在 <limits> 头文件中,并位于 std 命名空间下。

\mySubsubsection{}{How to do it...}

使用 std::numeric\_limits<T> 查询数值类型 T 的各种属性:

\begin{itemize}
\item
min() 和 max() 静态方法来获取类型的最小值和最大值:

\begin{cpp}
// example 1
template<typename T, typename Iter>
T minimum(Iter const start, Iter const end) // finds the
// minimum value
// in a range
{
    T minval = std::numeric_limits<T>::max();
    for (auto i = start; i < end; ++i)
    {
        if (*i < minval)
        minval = *i;
    }
    return minval;
}
// example 2
int range[std::numeric_limits<char>::max() + 1] = { 0 };
// example 3
switch(get_value())
{
    case std::numeric_limits<int>::min():
    // do something
    break;
}
\end{cpp}

\item
使用其他静态方法和静态常量来获取数值类型的其他属性。以下是一个例子,其中 bits 变量是一个 std::bitset 实例,包含表示由变量 n(这是一个整数)所代表的数值所需的一系列位:

\begin{cpp}
auto n = 42;
std::bitset<std::numeric_limits<decltype(n)>::digits>
bits { static_cast<unsigned long long>(n) };
\end{cpp}
\end{itemize}

\begin{myNotic}
C++11 起,std::numeric\_limits<T> 的使用没有限制;现代 C++ 代码中,建议优先使用,而非 C 中定义的宏。
\end{myNotic}

\mySubsubsection{}{How it work...}

std::numeric\_limits<T> 类模板允许开发者查询数值类型的属性。实际值通过特化获得,标准库为所有内置数值类型(如 char、short、int、long、float、double 等)提供了特化。此外,第三方库也可能为其他类型提供实现,例如:实现了 bigint 类型和 decimal 类型的数值库,可能会为这些类型提供 numeric\_limits 的特化。

<limits> 头文件中提供了以下数值类型的特化。char16\_t 和 char32\_t 类型的特化是在 C++11 中新增的;其他特化在此之前就已经存在。除了前面列出的特化外,库还为这些数值类型的每一个 cv 限定版本提供了特化,这些特化与未限定的特化相同。对于类型 int,有四个实际的特化(并且它们是相同的):numeric\_limits<int>、numeric\_limits<const int>、numeric\_limits<volatile int> 和 numeric\_limits<const volatile int>。 可以在 \url{https://en.cppreference.com/w/cpp/types/numeric_limits} 上找到特化的完整列表.

自 C++11 起,std::numeric\_limits 的所有静态成员都是 constexpr 类型,可以在需要常量表达式的地方使用。这相对于 C++ 宏有几个显著的优势:

\begin{itemize}
\item
更容易记忆,只需要知道类型的名称。

\item
支持 C 中不可用的类型,如 char16\_t 和 char32\_t。

\item
模板中未知类型的唯一解决方案。

\item
min 和 max 只是提供的多种类型属性中的两个,其实际用途超出了显示的数值限制。因此,这个类应称为 numeric\_properties,而不是 numeric\_limits。
\end{itemize}

以下是函数模板 print\_type\_properties(),用于输出类型的最小值和最大值,以及其他信息:

\begin{cpp}
emplate <typename T>
void print_type_properties()
{
    std::cout
    << "min="
    << std::numeric_limits<T>::min()        << '\n'
    << "max="
    << std::numeric_limits<T>::max()        << '\n'
    << "bits="
    << std::numeric_limits<T>::digits       << '\n'
    << "decdigits="
    << std::numeric_limits<T>::digits10     << '\n'
    << "integral="
    << std::numeric_limits<T>::is_integer   << '\n'
    << "signed="
    << std::numeric_limits<T>::is_signed    << '\n'
    << "exact="
    << std::numeric_limits<T>::is_exact     << '\n'
    << "infinity="
    << std::numeric_limits<T>::has_infinity << '\n';
}
\end{cpp}

如果对 unsigned short、int 和 double 调用 print\_type\_properties() 函数,将得到以下输出:

print\_type\_properties() 函数对 unsigned short、int 和 double 的输出如下:

\begin{itemize}
\item
unsigned short

\begin{shell}
min=0
max=65535
bits=16
decdigits=4
integral=1
signed=0
exact=1
infinity=0
\end{shell}

\item
int

\begin{shell}
min=-2147483648
max=2147483647
bits=31
decdigits=9
integral=1
signed=1
exact=1
infinity=0
\end{shell}

\item
double

\begin{shell}
min=2.22507e-308
max=1.79769e+308
bits=53
decdigits=15
integral=0
signed=1
exact=0
infinity=1
\end{shell}
\end{itemize}

注意,digits 和 digits10 常量之间存在差异:

\begin{itemize}
\item
digits 表示整数类型的位数(不包括符号位,如果有)和填充位(如果有),以及浮点类型的尾数位数。

\item
digits10 表示一个类型可以无损地表示的十进制数字的数量。为了更好地理解,看一下 unsigned short 的情况。这是一种 16 位的整数类型,可以表示介于 0 和 65,536 之间的数字,可以表示最多五位的十进制数字,即从 10,000 到 65,536,但不能表示所有五位的十进制数字,因为从 65,537 到 99,999 的数字需要更多的位。因此,无需更多位即可表示的最大数字具有四位十进制数字(从 1,000 到 9,999),这就是 digits10 所指示的值。对于整数类型 T,digits10 的值与 digits 常量直接相关,即 std::numeric\_limits<T>::digits * std::log10(2)。
\end{itemize}

标准库中算术类型的别名(如 std::size\_t),也可以使用 std::numeric\_limits 来检查。然而,其他非算术类型的标准类型,如 std::complex<T> 或 std::nullptr\_t,则没有 std::numeric\_limits 的特化。

