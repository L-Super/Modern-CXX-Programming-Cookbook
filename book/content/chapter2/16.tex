
C++ 语言提供了两种格式化文本的方式:\verb|printf| 函数族和 I/O 流库,\verb|printf| 函数从 C 继承而来,提供了将格式文本与参数分离的方式。流库则提供了安全性和可扩展性,通常推荐使用流库,而非 \verb|printf| 函数。一般来说,流的速度较慢。C++20 标准提出了一种新的输出格式化库,其形式类似于 \verb|printf|,但更安全且具有可扩展性,旨在补充现有的流。本示例中,将介绍如何使用新功能来代替 \verb|printf| 函数或流库。

\mySubsubsection{}{Getting ready}

新格式化库可以在 <format> 头文件中找到,需要包含此头文件以便以下示例能够正常工作。

\mySubsubsection{}{How to do it...}

\verb|std::format()| 函数根据提供的格式字符串格式化其参数,可以按照以下方式使用:

\begin{itemize}
\item
每个参数对应的格式字符串中提供空的替换字段,用 \verb|{}| 表示:

\begin{cpp}
auto text = std::format("{} is {}", "John", 42);
\end{cpp}

\item
替换字段内指定参数列表中每个参数的基于零的索引,例如 \verb|{0}, {1}| 等。参数的顺序不重要,但索引必须有效:

\begin{cpp}
auto text = std::format("{0} is {1}", "John", 42);
\end{cpp}

\item
使用替换字段中的冒号 (\verb|:|) 后跟的格式说明符来控制输出文本。对于基本类型和字符串类型,这是一个标准格式说明符;对于时间类型,这是一个时间格式说明符:

\begin{cpp}
auto text = std::format("{0} hex is {0:08X}", 42);
auto now = std::chrono::system_clock::now();
auto date = std::format("Today is {:%Y-%m-%d}", now);
std::cout << date << '\n';
\end{cpp}
\end{itemize}

也可以使用迭代器以输出格式写入参数,可以使用 \verb|std::format_to()| 或 \verb|std::format_to_n()|:

\begin{itemize}
\item
使用 \verb|std::format_to()| 和辅助函数 \verb|std::back_inserter()| 将数据写入缓冲区,如 \verb|std::string| 或 \verb|std::vector<char>|:

\begin{cpp}
std::vector<char> buf;
std::format_to(std::back_inserter(buf), "{} is {}", "John", 42);
\end{cpp}

\item
使用 \verb|std::formatted_size()| 获取存储参数的格式化表示所需的字符数:

\begin{cpp}
auto size = std::formatted_size("{} is {}", "John", 42);
std::vector<char> buf(size);
std::format_to(buf.data(), "{} is {}", "John", 42);
\end{cpp}

\item
若要限制写入输出缓冲区的字符数,可以使用 \verb|std::format_to_n()|,其功能与 \verb|std::format_to()| 类似,但最多写入 n 个字符:

\begin{cpp}
char buf[100];
auto result = std::format_to_n(buf, sizeof(buf), "{} is {}", "John", 42);
\end{cpp}
\end{itemize}

从 C++23 开始,可以直接将格式化的文本写入文件流,比如标准输出控制台,使用新 <print> 头文件的函数:

\begin{itemize}
\item
\verb|std::print| 按照格式字符串写入参数:

\begin{cpp}
std::print("The answer is {}", 42);
\end{cpp}

\item
使用 \verb|std::println| 按照格式字符串写入参数,后面跟随一个换行符 (\verb|'\n'|):

\begin{cpp}
std::println("The answer is {}", 42);
std::FILE* stream = std::fopen("demo.txt", "w");
if (stream)
{
    std::println(stream, "The answer is {}", 42);
    std::fclose(stream);
}
\end{cpp}
\end{itemize}

\mySubsubsection{}{How it work...}

\verb|std::format()| 函数有多个重载版本,可以指定格式字符串为字符串视图或宽字符串视图,函数返回相应的 \verb|std::string| 或 \verb|std::wstring|。此外,还可以指定第一个参数为 \verb|std::locale|,用于本地格式化。所有函数重载都是变长函数模板,可以在格式后指定任意数量的参数。

格式字符串由普通字符、替换字段和转义序列组成。转义序列是 \verb|{{| 和 \verb|}}|,输出中会替换成 \verb|{| 和 \verb|}|,替换字段位于花括号 \verb|{}| 内部。可以选择包含一个非负整数,表示要格式化的参数的基于零的索引,以及一个冒号 (\verb|:|),后跟格式说明符。如果格式说明符无效,则抛出类型为 \verb|std::format_error| 的异常。

同样,\verb|std::format_to()| 也有多个重载版本,像 \verb|std::format()| 一样。两者的区别在于 \verb|std::format_to()| 总是以输出缓冲区的迭代器作为第一个参数,并返回一个指向输出范围末尾的迭代器(而不是像 \verb|std::format()| 返回字符串)。而 \verb|std::format_to_n()| 比 \verb|std::format_to()| 多一个参数,即第二个参数表示要写入缓冲区的最大字符数。

下面列出了这三个函数模板最简单的重载:

\begin{cpp}
template<class... Args>
std::string format(std::string_view fmt, const Args&... args);
template<class OutputIt, class... Args>
OutputIt format_to(OutputIt out,
                   std::string_view fmt, const Args&... args);
template<class OutputIt, class... Args>
std::format_to_n_result<OutputIt>
format_to_n(OutputIt out, std::iter_difference_t<OutputIt> n,
            std::string_view fmt, const Args&... args);
\end{cpp}

当提供格式字符串时,可以提供参数标识符(基于零的索引)或省略,但同时使用两者是非法的。如果省略了索引,参数将按提供的顺序处理,替换字段的数量不得大于提供的参数数量。如果提供了索引,必须有效,才能使格式字符串有效。

当使用格式说明符时:

\begin{itemize}
\item
对于基本类型和字符串类型,可视为标准格式说明符。

\item
对于时间类型,可视为时间格式说明符。

\item
对于用户定义的类型,由用户定义的 \verb|std::formatter| 类型的特化定义。
\end{itemize}

标准格式说明符基于 Python 的格式说明符,语法如下:

\begin{shell}
fill-and-align(optional) sign(optional) #(optional) 0(optional) width(optional) precision(optional) L(optional) type(optional)
\end{shell}

这些语法部分简要描述如下:

\verb|fill-and-align| 是一个可选的填充字符,后跟着一个对齐选项:

\begin{itemize}
\item
\verb|<|: 强制字段左对齐可用空间。

\item
\verb|>|: 强制字段右对齐可用空间。

\item
\verb|^|: 强制字段居中对齐可用空间。将在左边插入 n/2 个字符,在右边也插入 n/2 个字符:

\begin{cpp}
auto t1 = std::format("{:5}", 42);    // "   42"
auto t2 = std::format("{:5}", 'x');   // "x    "
auto t3 = std::format("{:*<5}", 'x'); // "x****"
auto t4 = std::format("{:*>5}", 'x'); // "****x"
auto t5 = std::format("{:*^5}", 'x'); // "**x**"
auto t6 = std::format("{:5}", true);  // "true "
\end{cpp}

\end{itemize}

\verb|sign|, \verb|#| 和 0 只有在使用数字(无论是整数还是浮点数)时才有效。符号可以是:

\begin{itemize}
\item
+: 指定正数和负数都要显示符号。

\item
-: 指定只有负数显示符号(这是默认行为)。

\item
空格: 指定负数显示符号,非负数前显示空格:
\end{itemize}

\verb|#| 符号导致使用替代形式,可以是以下之一:

\begin{itemize}
\item
对于整数类型,当指定二进制、八进制或十六进制表示时,替代形式会在输出中添加前缀 0b、0 或 0x。

\item
对于浮点数类型,替代形式会导致格式化值中始终存在小数点,即使后面没有数字也是如此。此外,当使用 \verb|g| 或 \verb|G| 时,不会从输出中删除尾随的零。
\end{itemize}

数字 0 指定应在字段宽度内输出前导零,但当浮点数类型的值为无穷大或 \verb|NaN| 时不适用。当与对齐选项一起出现时,将忽略0说明符:

\begin{cpp}
auto t9  = std::format("{:+05d}", 42); // "+0042"
auto t10 = std::format("{:#05x}", 42); // "0x02a"
auto t11 = std::format("{:<05}", -42); // "-42  "
\end{cpp}

\verb|width| 指定最小字段宽度,可以是一个正十进制数或嵌套的替换字段。\verb|precision| 字段指示浮点数类型的精度,或对于字符串类型,从字符串中使用的字符数。由点 (\verb|.|) 后跟一个非负十进制数或嵌套的替换字段指定。

特定于本地化的格式化用大写的 L 指定,导致使用特定于本地化的形式。此选项仅适用于算术类型。

可选的 \verb|type| 确定了数据如何在输出中呈现。可用的字符串呈现类型如下表所示:

% Please add the following required packages to your document preamble:
% \usepackage{multirow}
% \usepackage{longtable}
% Note: It may be necessary to compile the document several times to get a multi-page table to line up properly
\begin{longtable}{|l|l|l|}
\hline
\textbf{类型} &
\textbf{呈现类型} &
\textbf{描述} \\ \hline
\endfirsthead
%
\endhead
%
字符串 &
无, s &
将字符串复制到输出。 \\ \hline
\multirow{7}{*}{整数类型} &
b &
二进制格式,前缀为 0b \\ \cline{2-3}
&
B &
二进制格式,前缀为 0B \\ \cline{2-3}
&
C &
字符格式。将值作为字符类型复制到输出。 \\ \cline{2-3}
&
无 或 d &
十进制格式。 \\ \cline{2-3}
&
O &
八进制格式,前缀为 0(除非值为 0)。 \\ \cline{2-3}
&
x &
十六进制格式,前缀为 0x。 \\ \cline{2-3}
&
X &
十六进制格式,前缀为 0X。 \\ \hline
\multirow{2}{*}{char 和 wchar\_t} &
无 或 c &
将字符复制到输出。\\ \cline{2-3}
&
b, B, c, d, o, x, X &
整数呈现类型。 \\ \hline
\multirow{2}{*}{bool} &
无 或 s  &
\begin{tabular}[c]{@{}l@{}}将 true 或 false 作为文本表示形式\\复制到输出(或其本地化形式)。\end{tabular} \\ \cline{2-3}
&
b, B, c, d, o, x, X &
整数呈现类型。 \\ \hline
\multirow{7}{*}{浮点数} &
a &
\begin{tabular}[c]{@{}l@{}}十六进制表示。如同调用了\\std::to\_chars(\\first, \\last, \\value, \\std::chars\_format::hex, \\precision) \\或 \\std::to\_chars(\\first, \\last, \\value, \\std::chars\_format::hex),\\具体取决于是否指定了精度。\end{tabular} \\ \cline{2-3}
&
A &
\begin{tabular}[c]{@{}l@{}}与 a 相同,只是对于大于 9 的\\数字使用大写字母,并使用 P 表示指数。 \end{tabular} \\ \cline{2-3}
&
e &
\begin{tabular}[c]{@{}l@{}}科学记数法表示。如同调用了 \\std::to\_chars(\\first, \\last, \\value, \\std::chars\_format::scientific, \\precision)。\end{tabular} \\ \cline{2-3}
&
E &
与 e 类似,只是使用 E 表示指数。 \\ \cline{2-3}
&
f, F &
\begin{tabular}[c]{@{}l@{}}定点表示。如同调用了 \\std::to\_chars(\\first, \\last, value, \\std::chars\_format::fixed, \\precision)。\\当未指定精度时,默认值为 6。\end{tabular} \\ \cline{2-3}
&
g &
\begin{tabular}[c]{@{}l@{}}一般浮点表示。如同调用了 \\std::to\_chars(\\first, last, \\value, \\std::chars\_format::general, \\precision)。\\当未指定精度时,默认值为 6。\end{tabular}  \\ \cline{2-3}
&
G &
与 g 相同,只是使用 E 表示指数。\\ \hline
指针 &
无 或 p &
\begin{tabular}[c]{@{}l@{}}指针表示。如同调用了 \\std::to\_chars(\\first, \\last, \\reinterpret\_cast\\\textless{}std::uintptr\_t\textgreater{}(value), \\16),\\并在输出前加上前缀 0x。仅当定义了 \\std::uintptr\_t 时可用;\\否则,输出是实现定义的。\end{tabular} \\ \hline
\end{longtable}

\begin{center}
表 2.16: 可用的呈现类型列表
\end{center}

时间格式说明符的形式如下:

\begin{shell}
fill-and-align(optional) width(optional) precision(optional) chrono-spec(optional)
\end{shell}

\verb|fill-and-align|, \verb|width| 和 \verb|precision| 字段的含义与之前描述的标准格式说明符相同。精度仅对 \verb|std::chrono::duration| 类型有效,当表示类型为浮点数时,其他情况下使用会抛出 \verb|std::format_error| 异常。

时间格式说明符可以为空,参数将如同通过流式传输到 \verb|std::stringstream| 并复制结果字符串一样进行格式化,也可以由一系列转换说明符和普通字符组成。以下是一些常用的格式说明符:

% Please add the following required packages to your document preamble:
% \usepackage{longtable}
% Note: It may be necessary to compile the document several times to get a multi-page table to line up properly
\begin{longtable}{|l|l|}
\hline
\textbf{转换说明符} & \textbf{描述}                                                                                         \\ \hline
\endfirsthead
%
\endhead
%
\%\%                          & 写入字面上的 \% 字符。                                                                                 \\ \hline
\%n                           & 写入换行符。                                                                                          \\ \hline
\%t                           & 写入水平制表符。                                                                                      \\ \hline
\%Y & 写入十进制数形式的年份。如果结果少于四位数字,则用 0 左填充至四位数字。                             \\ \hline
\%m & 写入十进制数形式的月份(一月为 01)。如果结果是一位数字,则前面加 0。                                  \\ \hline
\%d & 写入十进制数形式的日。如果结果是一位数字,则前面加 0。                                              \\ \hline
\%w                           & 写入十进制数形式的星期几(0-6),其中周日为 0。                                                        \\ \hline
\%D                           & 等效于 \%m/\%d/\%y。                                                                                  \\ \hline
\%F                           & 等效于 \%Y-\%m-\%d。                                                                                  \\ \hline
\%H                           & 写入十进制数形式的小时(24 小时制)。如果结果是一位数字,则前面加 0。                                 \\ \hline
\%I                           & 写入十进制数形式的小时(12 小时制)。如果结果是一位数字,则前面加 0。                                 \\ \hline
\%M                           & 写入十进制数形式的分钟。如果结果是一位数字,则前面加 0。                                                \\ \hline
\%S & 写入十进制数形式的秒。如果秒数小于 10,则结果前面加 0。                                         \\ \hline
\%R                           & 等效于 \%H:\%M。                                                                                      \\ \hline
\%T                           & 等效于 \%H:\%M:\%S。                                                                                  \\ \hline
\%X                           & 写入本地化的时间表示形式。                                                                             \\ \hline
\end{longtable}

\begin{center}
表 2.17: 常见的时间格式说明符列表
\end{center}

时间库的所有格式说明符的完整列表可以在 \url{https://en.cppreference.com/w/cpp/chrono/system_clock/formatter} 查看。

向控制台或文件流写入格式化文本需要两个步骤(将文本格式化为字符串或字符数组,然后将该缓冲区写入输出流),C++23 标准引入了几种新函数来简化这个过程。

新 \verb|std::print| 和 \verb|std::println| 函数非常相似,唯一的区别是 \verb|std::println| 在格式化文本后附加了一个 \verb|\n| 字符(换行符)。这两个函数各有两个重载版本:

\begin{itemize}
\item
接受 \verb|std::FILE*| 作为第一个参数,代表输出文件流

\item
没有这样的参数,隐式使用 C 输出流 stdout
\end{itemize}

以下两种调用 \verb|std::println| 的方式等价:

\begin{cpp}
std::println("The answer is {}", 42);
std::println(stdout, "The answer is {}", 42);
\end{cpp}

以下两个调用 \verb|std::print| 和 \verb|std::println| 对标准输出流产生的结果相同:

\begin{cpp}
std::println("The answer is {}", 42);
std::print("The answer is {}\n", 42);
\end{cpp}

格式字符串的规格说明与 \verb|std::format| 相同,已在前面介绍过。


