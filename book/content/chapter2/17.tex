
C++20 格式化库是对使用类似 printf 的函数或 I/O 流库的一种现代替代方案,实际上是补充了这些传统方法。标准为基本类型提供了默认格式化,如整型和浮点类型、布尔类型、字符类型、字符串和时间类型,也可以为自定义类型创建特定的格式化规则。本示例中,将介绍如何为自定义类型创建格式化规则。

\mySubsubsection{}{Getting ready}

阅读上一节的示例,以熟悉格式化库。接下来的例子中,将使用以下类型:

\begin{cpp}
struct employee
{
    int         id;
    std::string firstName;
    std::string lastName;
};
\end{cpp}

\mySubsubsection{}{How to do it...}

为了使用户定义的类型能够使用新格式化库进行格式化,必须执行以下步骤:

\begin{itemize}
\item
std 命名空间中定义 \verb|std::formatter<T, CharT>| 类型的特化。

\item
实现 \verb|parse()| 方法来解析格式字符串中对应当前参数的部分。如果类型继承自另一个格式化器,则此方法可以省略。

\item
实现 \verb|format()| 方法来格式化参数并通过 \verb|format_context| 写入输出。
\end{itemize}

对于上述的 employee 类型,将 employee 格式化为  \verb|[42]| \verb|John Doe| 形式的格式化器(即 \verb|[id] firstName lastName|) 可以有如下实现:

\begin{cpp}
template <>
struct std::formatter<employee>
{
    constexpr auto parse(format_parse_context& ctx)
    {
        return ctx.begin();
    }
    auto format(employee const & e, format_context& ctx) const
    {
        return std::format_to(ctx.out(),
        "[{}] {} {}",
        e.id, e.firstName, e.lastName);
    }
};
\end{cpp}

\mySubsubsection{}{How it work...}

格式化库使用 \verb|std::formatter<T, CharT>| 类型模板定义给定类型的格式化规则。内置类型、字符串类型和时间类型都由库提供格式化器,这些格式化器为 \verb|std::formatter<T, CharT>| 类型模板的特化实现。

此类型有两个方法:

\begin{itemize}
\item
\verb|parse()|,接受一个 \verb|std::basic_format_parse_context<CharT>| 类型的参数,并解析由解析上下文提供的类型 T 的格式规范,解析结果应该存储在类型的成员中。如果解析成功,此函数应返回表示格式规范结束的 \verb|std::basic_format_parse_context<CharT>::iterator| 类型值。如果解析失败,函数应抛出一个 \verb|std::format_error| 类型的异常来提供错误详情。

\item
\verb|format()|,接受两个参数,第一个是要格式化的类型 T 的对象,第二个是一个类型为 \verb|std::basic_format_context<OutputIt, CharT>| 的格式化上下文类型实例。此函数应该根据所需的规范(隐式的或解析格式规范的结果)将输出写入 \verb|ctx.out()|。函数必须返回一个类型为 \verb|std::basic_format_context<OutputIt, CharT>::iterator| 的值,表示输出的结束。
\end{itemize}

前一节的实现中,\verb|parse()| 函数除了返回一个表示格式规范开始的迭代器之外不做其他事情。格式化总是通过在方括号内打印员工标识符,后面跟着名字和姓氏来进行,例如 \verb|[42] John Doe|。会尝试使用格式规范会导致编译时错误:

\begin{cpp}
employee e{ 42, "John", "Doe" };
auto s1 = std::format("{}", e);   // [42] John Doe
auto s2 = std::format("{:L}", e); // error
\end{cpp}

如果想让自定义类型支持格式规范,则必须正确地实现 \verb|parse()| 方法。为了展示如何做到这一点,将支持 employee 类型的多个规范符:

% Please add the following required packages to your document preamble:
% \usepackage{longtable}
% Note: It may be necessary to compile the document several times to get a multi-page table to line up properly
\begin{longtable}{|l|l|l|}
\hline
\textbf{规范符} & \textbf{字典序} & \textbf{示例}   \\ \hline
\endfirsthead
%
\endhead
%
L                  & 字典序  & {[}42{]} Doe, John \\ \hline
l                  & 小写            & {[}42{]} john doe  \\ \hline
u                  & 大写            & {[}42{]} JOHN DOE  \\ \hline
\end{longtable}

\begin{center}
表 2.18: 支持自定义 employee 类的规范符
\end{center}

使用 L 规范符时,employee 格式化为方括号内的标识符,后面跟着姓氏、逗号,然后是名字,比如 \verb|[42] Doe, John|。这些规范符的组合也是可能的,\verb|{:Ll}| 会产生 \verb|[42] doe, john|n,而 \verb|{:uL}| 则会产生 \verb|[42] DOE, JOHN|。

std::formatter 类针对 employee 类的特化实现:

\begin{cpp}
template<>
struct std::formatter<employee>
{
    constexpr auto parse(std::format_parse_context& ctx)
    {
        auto iter = begin(ctx);
        while(iter != ctx.end() && *iter != '}')
    {
        switch (*iter)
        {
            case 'L': lexicographic_order = true; break;
            case 'u': uppercase = true; break;
            case 'l': lowercase = true; break;
        }
        ++iter;
    }
    return iter;
}
auto format(employee const& e, std::format_context& ctx) const
{
    if (lexicographic_order)
    return std::format_to(ctx.out(),
                          "[{}] {}, {}",
                          e.id,
                          text_format(e.lastName),
                          text_format(e.firstName));
    return std::format_to(ctx.out(),
                          "[{}] {} {}",
                          e.id,
                          text_format(e.firstName),
                          text_format(e.lastName));
}
private:
bool lexicographic_order = false;
bool uppercase = false;
bool lowercase = false;

constexpr std::string text_format(std::string text) const
{
    if(lowercase)
        std::transform(text.begin(), text.end(), text.begin(),                         ::tolower);
    else if(uppercase)
        std::transform(text.begin(), text.end(), text.begin(),                         ::toupper);
    return text;
}
};
\end{cpp}

\verb|parse()| 函数接收包括格式字符串的解析上下文。\verb|begin()| 迭代器指向格式分隔符(:)后的格式字符串的第一个元素。下表提供了一些示例:

% Please add the following required packages to your document preamble:
% \usepackage{longtable}
% Note: It may be necessary to compile the document several times to get a multi-page table to line up properly
\begin{longtable}{|l|l|l|}
\hline
\textbf{格式     } & \textbf{begin() 迭代器} & \textbf{范围  } \\ \hline
\endfirsthead
%
\endhead
%
"\{\}"          & 等于end()            & 空          \\ \hline
"\{0\}"         & 等于end()            & 空          \\ \hline
"\{0:L\}"       & 指向'L'             & L\}            \\ \hline
"\{:L\}"        & 指向'L'             & L\}            \\ \hline
"\{:Lul\}"      & 指向'L'             & Lul\}          \\ \hline
\end{longtable}

\begin{center}
表 2.19: 解析上下文内容示例
\end{center}

有了这些,前面的示例代码(使用 \verb|{:L}| 格式参数)就可以正常工作了,还可以使用 L、u 和 l 规范符的各种组合:

\begin{cpp}
auto s1 = std::format("{}", e);     // [42] John Doe
auto s2 = std::format("{:L}", e);   // [42] Doe, John
auto s3 = std::format("{:u}", e);   // [42] JOHN DOE
auto s4 = std::format("{:lL}", e);  // [42] doe, john
// uppercase ignored when lowercase also specified
auto s5 = std::format("{:ulL}", e); // [42] doe, john
\end{cpp}

也可以多次使用同一个参数:

\begin{cpp}
auto s6 = std::format("{0} = {0:L}", e);
// [42] John Doe = [42] Doe, John
\end{cpp}

使用其他格式规范符(如 A)将不起作用;会忽略规范符,使用默认格式化:

\begin{cpp}
auto s7 = std::format("{:A}", e);   // [42] John Doe
\end{cpp}

如果不需要解析格式规范符以支持各种选项,可以完全省略 \verb|parse()| 方法,但为此 \verb|std::formatter| 特化必须从另一个 \verb|std::formatter| 类型进行派生:

\begin{cpp}
template<>
struct std::formatter<employee> : std::formatter<char const*>
{
    auto format(employee const& e, std::format_context& ctx) const
    {
        return std::format_to(ctx.out(), "[{}] {} {}",
        e.id, e.firstName, e.lastName);
    }
};
\end{cpp}

这个针对 employee 类型的特化与“How to do it...”展示的第一个实现等价。

\mySubsubsection{}{There's more...}

C++23 标准引入了一个新的概念,称为 std::formattable(位于 <format> 头文件中),该概念规定了一个类型是可格式化的。所以存在一个类型 T 的 std::formatter 特化,并且定义了 parse() 和 format() 成员函数。

