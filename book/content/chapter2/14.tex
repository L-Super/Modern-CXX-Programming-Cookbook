

正则表达式库还支持基于正则表达式的文本替换。本示例中,将介绍如何使用 \verb|std::regex_replace()| 进行文本替换。

\mySubsubsection{}{How to do it...}

为了使用正则表达式进行文本替换,应该按照以下步骤操作:

\begin{itemize}
\item
包含头文件 <regex> 和 <string> 以及 C++14 标准的字符串用户定义字面量命名空间 \verb|std::string_literals|:

\begin{cpp}
#include <regex>
#include <string>
using namespace std::string_literals;
\end{cpp}

\item
使用带有替换字符串作为第三个参数的 \verb|std::regex_replace()| 算法。以下示例中,将所有由 a, b 或 c 组成的三个字符组成的单词替换为三个减号:

\begin{cpp}
auto text{"abc aa bca ca bbbb"s};
auto rx = std::regex{ R"(\b[a|b|c]{3}\b)"s };
auto newtext = std::regex_replace(text, rx, "---"s);
\end{cpp}

\item
使用以 \verb|$| 前缀的匹配标识符作为第三个参数的 \verb|std::regex_replace()| 算法。例如,将格式为“lastname, firstname”的名字替换为“firstname lastname”格式的名字:

\begin{cpp}
auto text{ "bancila, marius"s };
auto rx = std::regex{ R"((\w+),\s*(\w+))"s };
auto newtext = std::regex_replace(text, rx, "$2 $1"s);
\end{cpp}
\end{itemize}

\mySubsubsection{}{How it work...}

\verb|std::regex_replace()| 算法有几个具有不同类型参数的重载版本,其参数的意义为:

\begin{itemize}
\item
需要替换的输入字符串

\item
用于识别待替换字符串部分的正则表达式 \verb|std::basic_regex|类型实例

\item
用于替换的字符串格式

\item
可选的匹配标志
\end{itemize}

根据使用的版本,返回值可以是字符串或作为参数提供的输出迭代器的副本。用于替换的字符串格式可以是简单的字符串,也可以由 \verb|$| 前缀指示的匹配标识符:

\begin{itemize}
\item
\verb|$&| 表示整个匹配。

\item
\verb|$1, $2, $3|表示第一个、第二个、第三个子匹配。

\item
\verb|$`| 表示第一个匹配之前的字符串部分。

\item
\verb|$'| 表示最后一个匹配之后的字符串部分。
\end{itemize}

“How to do it...”显示的第一个示例中,初始文本包含两个恰好由 \verb|a|, \verb|b|, 和 \verb|c| 字符组成的三个字符的单词:\verb|abc| 和 \verb|bca|。正则表达式指定了一个在单词边界之间恰好有三个字符的表达式,所以像 \verb|bbbb| 这样的子串不会匹配该表达式。替换的结果是字符串 text 变成了 \verb|--- aa --- ca bbbb|。

可为 \verb|std::regex_replace()| 算法指定匹配标志。默认情况下,匹配标志是 \verb|std::regex_constants::match_default|,这指定了构建正则表达式时使用的语法是 ECMAScript。如果只想替换第一次出现的情况,可以指定 \verb|std::regex_constants::format_first_only|。下面的示例中,因为替换在找到第一个匹配后就停止了,所以结果为 \verb|--- aa bca ca bbbb|:

\begin{cpp}
auto text{ "abc aa bca ca bbbb"s };
auto rx = std::regex{ R"(\b[a|b|c]{3}\b)"s };
auto newtext = std::regex_replace(text, rx, "---"s,
                 std::regex_constants::format_first_only);
\end{cpp}

替换字符串可以包含整个匹配、特定子匹配或未匹配部分的特殊指示符,如前所述。“How to do it...”显示的第二个示例中,正则表达式识别一个至少有一个字符的单词,后跟一个逗号和可能的空格,然后再跟另一个至少有一个字符的单词。第一个单词可认为是姓氏,而第二个单词可认为是名字。替换字符串采用 \verb|$2 $1| 格式。这是一个指令,用于将匹配的表达式(此示例中为整个原字符串)替换为另一个字符串,该字符串由第二个子匹配项后,跟一个空格和第一个子匹配项组成。

整个字符串都是一个匹配。下面的示例中,字符串内将有多个匹配,都将用指示字符串替换。这个示例中,当不定冠词 a 出现在以元音开头的单词之前时(并不涵盖以元音发音开头的单词),将其替换为不定冠词 an:

\begin{cpp}
auto text{"this is a example with a error"s};
auto rx = std::regex{R"(\ba ((a|e|i|u|o)\w+))"s};
auto newtext = std::regex_replace(text, rx, "an $1");
\end{cpp}

正则表达式识别单个字母 a 作为一个单词(\verb|\b| 表示单词边界,\verb|\ba| 代表一个只有一个字母 a 的单词),后跟一个空格和一个以元音开头且至少有两个字符的单词。当识别到这样的匹配时,它会替换为一个字符串,该字符串由固定的字符串 an 后跟一个空格和匹配的第一个子表达式(即单词本身)组成。这个示例中,\verb|newtext| 字符串将是 \verb|this is an example with an error|。

除了子表达式的标识符(\verb|$1|, \verb|$2| 等)之外,还有其他标识符用于整个匹配(\verb|&|)、第一个匹配之前的字符串部分(\verb|`|)和最后一个匹配之后的字符串部分(\verb|$'|)。最后一个示例中,将日期的格式从 \verb|dd.mm.yyyy| 更改为 \verb|yyyy.mm.dd|:

\begin{cpp}
auto text{"today is 1.06.2023!!"s};
auto rx = std::regex{R"((\d{1,2})(\.|-|/)(\d{1,2})(\.|-|/)(\d{4}))"s};
// today is 2023.06.1!!
auto newtext1 = std::regex_replace(text, rx, R"($5$4$3$2$1)");
// today is [today is ][1.06.2023][!!]!!
auto newtext2 = std::regex_replace(text, rx, R"([$`][$&][$'])");
\end{cpp}

正则表达式匹配一个一或两位数,后跟一个点、连字符或斜杠;再后跟另一个一或两位数;然后是一个点、连字符或斜杠;最后是一个四位数。请记住这只是个例子,解析日期有更好的表达式。

对于 \verb|newtext1|,替换字符串是 \verb|$5$4$3$2$1|;这表示年份,后跟第二个分隔符,然后月份,第一个分隔符,最后是日期。对于输入字符串 \verb|today is 1.06.2023!!|,结果是 \verb|today is 2023.06.1!!|。

对于 \verb|newtext2|,替换字符串是 \verb|[$`][$&][$']|;所以第一个匹配之前的部分,后跟整个匹配,最后是最后一个匹配之后的部分,都用方括号括起来,但结果不是 \verb|[!!][1.06.2023][today is ]|,而是 \verb|today is [today is ][1.06.2023][!!]!!|。这是因为替换的是匹配的表达式,只有日期(\verb|1.06.2023|)替换成了另一个由初始字符串所有部分组成的字符串。













