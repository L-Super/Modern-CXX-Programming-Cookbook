

处理字符串时,临时对象会不断地创建。很多时候,这些临时对象无关紧要,仅用于从一处复制数据到另一处(例如,从函数复制到调用者)。这会成为一个性能问题,因为需要内存分配和数据复制,这些都应该避免。为此,C++17 标准提供了一个新的字符串类模板 \verb|std::basic_string_view|,表示一个非拥有常量引用到字符串(即,一系列字符)。本示例中,将介绍何时以及如何使用此类。

\mySubsubsection{}{Getting ready}

\verb|string_view| 类位于 \verb|std| 命名空间中,可以通过包含 \verb|string_view| 头文件来使用。

\mySubsubsection{}{How to do it...}

应该使用 \verb|std::string_view| 来传递函数参数(或从函数返回值),而不是 \verb|std::string const &|,除非代码需要调用接受 \verb|std::string| 参数的其他函数(这种情况下,需要进行转换):

\begin{cpp}
std::string_view get_filename(std::string_view str)
{
    auto const pos1 {str.find_last_of('')};
    auto const pos2 {str.find_last_of('.')};
    return str.substr(pos1 + 1, pos2 - pos1 - 1);
}
char const file1[] {R"(c:\test\example1.doc)"};
auto name1 = get_filename(file1);
std::string file2 {R"(c:\test\example2)"};
auto name2 = get_filename(file2);
auto name3 = get_filename(std::string_view{file1, 16});
\end{cpp}

\mySubsubsection{}{How it work...}

了解新字符串类型是如何工作之前,来看一个函数的例子,该函数旨在提取文件名而不带扩展名。C++17 之前会这样写这个函数:

\begin{cpp}
std::string get_filename(std::string const & str)
{
    auto const pos1 {str.find_last_of('\\')};
    auto const pos2 {str.find_last_of('.')};
    return str.substr(pos1 + 1, pos2 - pos1 - 1);
}
auto name1 = get_filename(R"(c:\test\example1.doc)"); // example1
auto name2 = get_filename(R"(c:\test\example2)");     // example2
if(get_filename(R"(c:\test\_sample_.tmp)").front() == '_') {}
\end{cpp}

\begin{myNotic}
文件分隔符是 \verb|\|(反斜杠),就像在 Windows 中一样。对于基于 Linux 的系统,必须改为 /(斜杠)。
\end{myNotic}

\verb|get_filename()| 函数相对简单,接受一个 \verb|std::string| 的常量引用,并识别由最后一个文件分隔符和最后一个点界定的子串,这代表一个不带扩展名(也不带文件夹名称)的文件名。

这段代码的问题在于,可能会创建一个、两个或更多临时对象,具体取决于编译器优化。函数参数是一个常量 \verb|std::string| 引用,但函数是用字符串字面量调用的,这需要从字面量构造一个 \verb|std::string|。这些临时对象需要分配和复制数据,这既耗时又耗费资源。最后一个例子中,只是想检查文件名的第一个字符是否是下划线,但为了达到这个目的,至少创建了两个临时字符串对象。

\verb|std::basic_string_view| 类型模板旨在解决这个问题。这个类模板与 \verb|std::basic_string| 非常相似,两者几乎有相同的接口。原因在于 \verb|std::basic_string_view| 设计为可以直接替代对 \verb|std::basic_string| 的常量引用,而无需进一步更改代码。就像 \verb|std::basic_string| 一样,也有针对所有标准字符类型的特化:

\begin{cpp}
typedef basic_string_view<char>     string_view;
typedef basic_string_view<wchar_t>  wstring_view;
typedef basic_string_view<char16_t> u16string_view;
typedef basic_string_view<char32_t> u32string_view;
\end{cpp}

\verb|std::basic_string_view| 类模板定义了指向常量连续字符序列的引用,用于表示一个视图,不能用来修改引用的字符序列。一个 \verb|std::basic_string_view| 实例相对较小,只需要一个指向序列中第一个字符的指针和长度。可以从一个 \verb|std::basic_string| 类型实例构造,还可以从一个指针和长度,或者从一个以 null 结尾的字符序列(需要初始化遍历字符串以找到长度)构造,所以\verb|std::basic_string_view| 类型模板也可以作为多种类型字符串的通用接口(只读取数据即可)。另一方面,不可能从 \verb|std::basic_string_view| 转换到 \verb|std::basic_string| 。

必须显式地从 \verb|std::basic_string_view| 构造一个 \verb|std::basic_string| 实例:

\begin{cpp}
std::string_view sv{ "demo" };
std::string s{ sv };
\end{cpp}

\begin{myNotic}
主流编译器都提供了 \verb|std::basic_string|,包括小字符串优化。尽管实现细节不同,但通常依赖于静态分配一定数量的字符缓冲区(VC++ 和 GCC 5 或更新版本为 16 个字符),这些缓冲区不需要堆操作,只有当字符串的大小超过这个数量时才需要堆操作。
\end{myNotic}

除了与 \verb|std::basic_string| 提供的方法相同的方法外,\verb|std::basic_string_view| 还有另外两个方法:

\begin{itemize}
\item
\verb|remove_prefix()|: 通过增加起始位置 n 个字符并减少长度 n 个字符来缩小视图。

\item
\verb|remove_prefix()|: 通过减少长度 n 个字符来缩小视图。
\end{itemize}

以下示例中使用这两个成员函数来修剪 \verb|std::string_view| 开头和结尾的空格。该函数的实现首先查找第一个不是空格的元素,然后查找最后一个不是空格的元素。然后,从末尾移除最后一个非空格字符之后的所有内容,从开头移除直到第一个非空格字符之前的所有内容。函数返回两端都修剪过的新视图:

\begin{cpp}
std::string_view trim_view(std::string_view str)
{
    auto const pos1{ str.find_first_not_of(" ") };
    auto const pos2{ str.find_last_not_of(" ") };
    str.remove_suffix(str.length() - pos2 - 1);
    str.remove_prefix(pos1);
    return str;
}
auto sv1{ trim_view("sample") };
auto sv2{ trim_view("  sample") };
auto sv3{ trim_view("sample  ") };
auto sv4{ trim_view("  sample  ") };
std::string s1{ sv1 };
std::string s2{ sv2 };
std::string s3{ sv3 };
std::string s4{ sv4 };
\end{cpp}

当使用 \verb|std::basic_string_view| 时,需要注意两点:不能更改视图引用的数据,并且必须管理数据的生命周期,因为视图是非拥有的引用。

