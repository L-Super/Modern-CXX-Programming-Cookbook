

用户自定义字面量有两种形式:一种是经过编译器处理后传递给字面量操作符的形式,称为熟(cooked)形式;另一种是在传递给字面量操作符之前未经编译器处理的形式,称为生(raw)形式。后者仅适用于整型和浮点型。原始字面量对于改变编译器的常规行为非常有用。例如,像3.1415926这样的序列通常会编译器解释为浮点值,可通过使用原始用户自定义字面量,解释为用户定义的十进制值。本示例中,将介绍如何创建原始用户自定义字面量。

\mySubsubsection{}{Getting ready}

为了介绍如何创建原始用户自定义字面量,将定义二进制字面量。这些二进制字面量可以是8位、16位和32位(无符号)类型。这些类型将称为byte8、byte16和byte32,而要创建的字面量称为\verb|_b8|、\verb|_b16|和\verb|_b32|。

\mySubsubsection{}{How to do it...}

创建原始自定义字面量,应该遵循以下步骤:

\begin{enumerate}
\item
在单独的命名空间中定义你的字面量以避免符号冲突。

\item
始终以前缀下划线(\verb|_|)开始用户自定义后缀。

\item
定义如下形式的字面量操作符或字面量操作符模板:

\begin{cpp}
T operator "" _suffix(const char*);
template<char...> T operator "" _suffix();
\end{cpp}
\end{enumerate}

下面的例子展示了8位、16位和32位二进制字面量的一种实现:

\begin{cpp}
namespace binary
{
    using byte8  = unsigned char;
    using byte16 = unsigned short;
    using byte32 = unsigned int;
    namespace binary_literals
    {
        namespace binary_literals_internals
        {
            template <typename CharT, char... bits>
            struct binary_struct;
            template <typename CharT, char... bits>
            struct binary_struct<CharT, '0', bits...>
            {
                static constexpr CharT value{
                    binary_struct<CharT, bits...>::value };
            };
            template <typename CharT, char... bits>
            struct binary_struct<CharT, '1', bits...>
            {
                static constexpr CharT value{
                    static_cast<CharT>(1 << sizeof...(bits)) |
                    binary_struct<CharT, bits...>::value };
            };
            template <typename CharT>
            struct binary_struct<CharT>
            {
                static constexpr CharT value{ 0 };
            };
        }
        template<char... bits>
        constexpr byte8 operator""_b8()
        {
            static_assert(
            sizeof...(bits) <= 8,
            "binary literal b8 must be up to 8 digits long");
            return binary_literals_internals::
            binary_struct<byte8, bits...>::value;
        }
        template<char... bits>
        constexpr byte16 operator""_b16()
        {
            static_assert(
            sizeof...(bits) <= 16,
            "binary literal b16 must be up to 16 digits long");
            return binary_literals_internals::
            binary_struct<byte16, bits...>::value;
        }
        template<char... bits>
        constexpr byte32 operator""_b32()
        {
            static_assert(
            sizeof...(bits) <= 32,
            "binary literal b32 must be up to 32 digits long");
            return binary_literals_internals::
            binary_struct<byte32, bits...>::value;
        }
    }
}
\end{cpp}

\mySubsubsection{}{How it work...}

首先,在名为\verb|binary|的命名空间内部定义所有内容,并引入几个类型别名:\verb|byte8|、\verb|byte16|和\verb|byte32|,分别代表8位、16位和32位的整型。

上述实现使我们能够定义形如\verb|1010_b8|(值为十进制10的\verb|byte8|)或\verb|000010101100_b16|(值为十进制2130496的\verb|byte16|)的二进制字面量。要确保不超出每种类型的数字数量,像\verb|111100001_b8|这样的值应该是非法的,编译器应该产生错误。

字面量操作符模板定义在嵌套命名空间\verb|binary_literal_internals|中。这是一个好的实践,避免与其他命名空间中的其他字面量操作符产生符号冲突。如果发生这种情况,可以选择在正确的范围内使用适当的命名空间(比如:在一个函数或块中使用一个命名空间,在另一个函数或块中使用另一个命名空间)。

三个字面量操作符模板非常相似。不同之处在于其名称(\verb|_b8|、\verb|_b16|和\verb|_b32|)、返回类型(\verb|byte8|、\verb|byte16|和\verb|byte32|),以及静态断言中检查数字数量的条件。

后续的章节中会继续探讨变长模板参数包和模板递归;不过,为了更好地理解,这是该特定实现的工作方式:bits是一个模板参数包,不是一个值,而是模板实例化时可能拥有的所有值。例如,考虑字面量\verb|1010_b8|,则字面量操作符模板将实例化为\verb|operator"" _b8<'1', '0', '1', '0'>()|。进行计算二进制值的操作之前,先检查字面量中的数字数量。对于\verb|_b8|,这不应超过八位(包括任何尾随零)。同样地,对于\verb|_b16|,不应超过十六位,对于\verb|_b32|,不应超过三十二位。为此,使用\verb|sizeof...|操作符它返回参数包中元素的数量(即\verb|bits|)。

如果数字数量正确,可以展开参数包并递归地计算由二进制字面量表示的十进制值,可通过一个类型模板及其特化来完成。这些模板定义在另一个嵌套的命名空间\verb|binary_literals_internals|中。这也是一个好的实践,隐藏了(除非有\verb|using namespace|指令使其在当前命名空间中可用)实现细节。

\begin{myNotic}
虽然看起来像是递归,但这并不是真正的运行时递归。当编译器展开了模板并生成了代码之后,最终得到的是对具有不同参数数量重载函数的调用。这一点在第3章中有详细解释。
\end{myNotic}

\verb|binary_struct| 类模板具有一个模板类型 \verb|CharT| 作为函数的返回类型(需要这个是因为字面量操作符模板应该返回 \verb|byte8|、\verb|byte16| 或 \verb|byte32|),以及一个参数包:

\begin{cpp}
template <typename CharT, char... bits>
struct binary_struct;
\end{cpp}

该类模板有几个带有参数包分解的特化(参阅第3章)。当参数包的第一个数字是 \verb|'0'| 时,计算出的值保持不变,继续展开剩余部分。如果参数包的第一个数字是 \verb|'1'|,则新值是1,向左移动剩余包中数字的数量位,或者剩余包的值:

\begin{cpp}
template <typename CharT, char... bits>
struct binary_struct<CharT, '0', bits...>
{
    static constexpr CharT value{ binary_struct<CharT, bits...>::value };
};
template <typename CharT, char... bits>
struct binary_struct<CharT, '1', bits...>
{
    static constexpr CharT value{
        static_cast<CharT>(1 << sizeof...(bits)) |
        binary_struct<CharT, bits...>::value };
};
\end{cpp}

最后一个特化处理的是参数包为空的情况,返回0:

\begin{cpp}
template <typename CharT>
struct binary_struct<CharT>
{
    static constexpr CharT value{ 0 };
};
\end{cpp}

定义了这些辅助类后,可以按预期实现 \verb|byte8|、\verb|byte16| 和 \verb|byte32| 二进制字面量。需要将 \verb|binary_literals| 命名空间的内容引入当前命名空间,才能使用字面量操作符模板:

\begin{cpp}
using namespace binary;
using namespace binary_literals;
auto b1 = 1010_b8;
auto b2 = 101010101010_b16;
auto b3 = 101010101010101010101010_b32;
\end{cpp}

以下定义会触发编译器错误:

\begin{cpp}
// binary literal b8 must be up to 8 digits long
auto b4 = 0011111111_b8;
// binary literal b16 must be up to 16 digits long
auto b5 = 001111111111111111_b16;
// binary literal b32 must be up to 32 digits long
auto b6 = 0011111111111111111111111111111111_b32;
\end{cpp}

因为 \verb|static_assert| 中的条件没有满足,所以位于字面量操作符之前的字符序列长度都超过了预期。











