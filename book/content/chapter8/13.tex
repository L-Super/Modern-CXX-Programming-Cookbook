C++11 的线程支持库包含了互斥锁和条件变量,使线程能够同步访问共享资源。互斥锁允许多个进程中的一个线程执行,而其他想要访问共享资源的线程则会阻塞。某些场景下使用互斥锁可能会代价高昂,C++20 标准引入了几种新的、更简单的同步机制:门闩(latche)、栅栏(barrier)和信号量(semaphore)。虽然这些机制没有提供新的使用用例,但可能依赖于无锁机制,所以使用起来更简单且性能更高。

\mySubsubsection{}{Getting ready}

新的 C++20 同步机制定义在新的头文件中。对于 std::latch,需要包含 <latch>;对于 std::barrier,需要包含 <barrier>;而对于 \verb|std::counting_semaphore| 和 \verb|std::binary_semaphore|,则需要包含 <semaphore>。

本示例中的代码将使用以下两个函数:

\begin{cpp}
void process(std::vector<int> const& data) noexcept
{
    for (auto const e : data)
        std::cout << e << ' ';
    std::cout << '\n';
}
int create(int const i, int const factor) noexcept
{
    return i * factor;
}
\end{cpp}

\mySubsubsection{}{How to do it...}

按照以下方式使用 C++20 同步机制:

\begin{itemize}
\item
当需要线程等待一个由其他线程递减的计数器到达零时,使用 std::latch。门闩必须用非零值初始化,多个线程可以递减,而其他线程则等待计数器达到零。当这种情况发生时,所有等待的线程都会唤醒,而且门闩不能再可用。如果门闩计数器没有递减到零(即没有足够的线程递减它),等待的线程将永远阻塞。下面的例子中,四个线程正在创建数据(存储在整数vector中),主线程使用一个 std::latch 等待它们全部完成,每个线程在完成工作后递减门闩::

\begin{cpp}
int const jobs = 4;
std::latch work_done(jobs);
std::vector<int> data(jobs);
std::vector<std::jthread> threads;
for(int i = 1; i <= jobs; ++i)
{
    threads.push_back(std::jthread([&data, i, &work_done]{
        using namespace std::chrono_literals;
        std::this_thread::sleep_for(1s); // simulate work
        data[i-1] = create(i, 1);        // create data

        work_done.count_down();          // decrement counter
    }));
}
work_done.wait();             // wait for all jobs to finish
process(data);                // process data from all jobs
\end{cpp}

\item
当需要在并行任务之间执行循环同步时,使用 std::barrier。可以用一个计数器构建栅栏,并且可以选择性地提供一个完成函数。线程到达栅栏,递减内部计数器并阻塞。当计数器达到零时,调用完成函数,所有阻塞的线程都唤醒,一个新的周期开始。下面的例子中,四个线程正在创建它们存储在整数vector中的数据。当所有线程完成一个周期后,主线程通过一个完成函数处理数据。每个线程在一个周期完成后都会被 std::barrier 对象阻塞,该实例也存储了完成函数。这个过程重复十次:

\begin{cpp}
int const jobs = 4;
std::vector<int> data(jobs);
int cycle = 1;
std::stop_source st;
// completion function
auto on_completion = [&data, &cycle, &st]() noexcept {
    process(data);          // process data from all jobs
    cycle++;
    if (cycle == 10)        // stop after ten cycles
        st.request_stop();
};
std::barrier work_done(jobs, on_completion);
std::vector<std::jthread> threads;
for (int i = 1; i <= jobs; ++i)
{
    threads.push_back(std::jthread(
        [&data, &cycle, &work_done](std::stop_token st,
                                    int const i)
    {
        while (!st.stop_requested())
        {
            using namespace std::chrono_literals;
            // simulate work
            std::this_thread::sleep_for(200ms);
            data[i-1] = create(i, cycle); // create data
            work_done.arrive_and_wait();  // decrement counter
        }
    },
    st.get_token(),
    i));
}
for (auto& t : threads) t.join();
\end{cpp}

\item
当想限制 N 个线程(在 \verb|binary_semaphore| 的情况下是一个线程)访问共享资源,或者想在不同线程间传递通知时,使用 \verb|std::counting_semaphore<N>| 或 \verb|std::binary_semaphore|。下面的例子中,四个线程正在创建添加到整数向量末尾的数据。为了避免竞态条件,使用了一个 \verb|binary_semaphore| 实例来限制只有一个线程可访问vector:

\begin{cpp}
int const jobs = 4;
std::vector<int> data;
std::binary_semaphore sem(1);
std::vector<std::jthread> threads;
for (int i = 1; i <= jobs; ++i)
{
    threads.push_back(std::jthread([&data, i, &sem] {
        for (int k = 1; k < 5; ++k)
        {
            // simulate work
            using namespace std::chrono_literals;
            std::this_thread::sleep_for(200ms);
            // create data
            int value = create(i, k);
            // acquire the semaphore
            sem.acquire();
            // write to the shared resource
            data.push_back(value);
            // release the semaphore
            sem.release();
        }
    }));
}
for (auto& t : threads) t.join();
process(data); // process data from all jobs
\end{cpp}
\end{itemize}

\mySubsubsection{}{How it works...}

std::latch 实现了可以用于同步线程的计数器,是一个无竞争的类型,工作原理如下:

\begin{itemize}
\item
计数器在创建门闩时初始化,并且只能减少。

\item
一个线程可以减少门闩的值,并且可以多次这样做。

\item
一个线程可以通过等待直到门闩计数器达到零来进行阻塞。

\item
当计数器达到零时,门闩变为永久信号状态,会唤醒所有在门闩上阻塞的线程。
\end{itemize}

std::latch 类型有以下方法:

% Please add the following required packages to your document preamble:
% \usepackage{longtable}
% Note: It may be necessary to compile the document several times to get a multi-page table to line up properly
\begin{longtable}{|l|l|}
\hline
\textbf{方法} &
\textbf{描述} \\ \hline
\endfirsthead
%
\endhead
%
count\_down() &
\begin{tabular}[c]{@{}l@{}}
以 N(默认为 1)递减内部计数器而不阻塞调用者。此操作以原子方式\\执行。N 必须是一个不大于内部计数器值的正数;否则,行为未定义
\end{tabular}
\\ \hline
try\_wait() &
\begin{tabular}[c]{@{}l@{}}
指示内部计数器是否达到零,如果是则返回 true。尽管计数器已达到\\零,该函数仍有可能返回 false,概率非常低。
\end{tabular}
\\ \hline
wait() &
\begin{tabular}[c]{@{}l@{}}
阻塞调用线程,直到内部计数器达到零。如果内部计数器已经是零,函\\数将立即返回而不会阻塞。
\end{tabular}
\\ \hline
arrive\_and\_wait() &
\begin{tabular}[c]{@{}l@{}}
此函数等同于调用 count\_down(),然后调用 wait()。它以 N(默\\认为 1)递减内部计数器,然后阻塞调用线程,直到内部计数器达到零。
\end{tabular}
\\ \hline
\end{longtable}

\begin{center}
表 8.3: 描述原子操作内存访问排序的 std::memory\_order 成员
\end{center}

前一节的第一个例子中,有一个名为 \verb|work_done| 的 std::latch,用线程数(或任务数)初始化。每个线程产生数据,然后写入共享资源,即一个整数vector。尽管这是共享的,但由于每个线程写入不同的位置,不存在竞态条件,不需要同步机制。每个线程完成工作后都会递减门闩的计数器。主线程等待直到门闩的计数器达到零,之后处理线程的数据。

由于 std::latch 的内部计数器不能增加或重置,这种同步机制只能使用一次。一个类似但可重用的同步机制是 std::barrier。栅栏允许线程阻塞,直到某个操作完成,对于管理多个线程执行的重复任务很有用。

栅栏的工作原理如下:

\begin{itemize}
\item
栅栏包含一个在创建时初始化的计数器,可以到达栅栏的线程递减。当计数器达到零时,会重置为其初始值,栅栏可以重新使用。

\item
栅栏还包含一个完成函数,当计数器达到零时调用。如果使用默认完成函数,在调用 \verb|arrive_and_wait()| 或 \verb|arrive_and_drop()| 时调用。否则,完成函数将在参与完成阶段的某个线程上调用。

\item
栅栏从开始到重置的过程称为完成阶段。这从所谓的同步点开始,并以完成步骤结束。

\item
栅栏构建后的第一个到达同步点的 N 个线程称为参与线程集。只有这些线程会在随后的每个周期到达栅栏。

\item
到达同步点的线程可以通过调用 \verb|arrive_and_wait()| 参与完成阶段,但线程也可以通过调用 \verb|arrive_and_drop()| 将自己从参与集合中移除。这时,另一个线程必须在参与集合中取代它的位置。

\item
当参与集合中的所有线程都到达同步点时,将进入完成阶段。完成阶段开始于最后一个到达的线程调用 \verb|arrive_and_wait()| 或 \verb|arrive_and_drop()|,并以调用完成函数结束。
\end{itemize}

std::barrier 类型具有以下方法:

% Please add the following required packages to your document preamble:
% \usepackage{longtable}
% Note: It may be necessary to compile the document several times to get a multi-page table to line up properly
\begin{longtable}{|l|l|}
\hline
\textbf{方法} &
\textbf{描述} \\ \hline
\endfirsthead
%
\endhead
%
arrive() &
\begin{tabular}[c]{@{}l@{}}
到达栅栏的同步点,并将预期计数减去一个值 n。如果 n 的值大于\\预期计数,或者等于或小于零,则行为未定义。此函数以原子方式执\\行。
\end{tabular}
 \\ \hline
wait() &
在同步点处阻塞,直到完成步骤被执行。 \\ \hline
arrive\_and\_wait() &
\begin{tabular}[c]{@{}l@{}}
到达栅栏的同步点并阻塞。调用线程必须在参与集合中;否则,行为\\未定义。此函数仅在完成阶段结束后返回。
\end{tabular}
\\ \hline
arrive\_and\_drop() &
\begin{tabular}[c]{@{}l@{}}
到达栅栏的同步点并从参与集合中移除线程。函数是否阻塞直至完成\\阶段结束是一个实现细节。调用线程必须在参与集合中;否则,行为\\未定义。
\end{tabular}
\\ \hline
\end{longtable}

\begin{center}
表 8.4: std::barrier 类型的成员函数
\end{center}

“How to do it...”第二段代码中看到了一个 std::barrier 的例子。这个例子中,创建并初始化了一个 std::barrier,其中计数器表示线程的数量,还有一个完成函数。这个函数处理所有线程产生的数据,然后增加一个循环计数器,并在循环达到10次后请求线程停止。所以在线程完成工作之前,栅栏将执行10个周期。每个线程循环运行,直到接收到停止请求,在每次迭代中,它们生成一些数据,写入共享的整数vector中。在循环结束时,每个线程到达栅栏的同步点,递减计数器,并等待计数器归零以及完成函数执行。这是通过调用 std::barrier 类型的 \verb|arrive_and_wait()| 方法完成的。

C++20 线程支持库中提供的最后一种同步机制是信号量。信号量包含一个内部计数器,该计数器可以由多个线程递增和递减。当计数器达到零时,进一步尝试递减它会导致线程阻塞,直到另一个线程增加了计数器。

有两种信号量类:\verb|std::counting_semaphore<N>| 和 \verb|std::binary_semaphore|。后者实际上是 \verb|std::counting_semaphore<1>| 的别名。

\verb|counting_semaphore| 允许 N 个线程访问共享资源,而互斥锁只允许一个线程访问。所以,\verb|binary_semaphore| 类似于互斥锁,因为只有一个线程可以访问共享资源。另一方面,互斥锁绑定到一个线程:锁定互斥锁的线程必须解锁。然而,对于信号量来说并非如此。信号量可以由未获取的线程释放,获取了信号量的线程也不必释放。

\verb|std::counting_semaphore| 类具有以下方法:

% Please add the following required packages to your document preamble:
% \usepackage{longtable}
% Note: It may be necessary to compile the document several times to get a multi-page table to line up properly
\begin{longtable}{|l|l|}
\hline
\textbf{方法} &
\textbf{描述} \\ \hline
\endfirsthead
%
\endhead
%
acquire() &
\begin{tabular}[c]{@{}l@{}}
如果计数器大于 0,则将其递减 1。否则,会阻塞,直到计数\\器大于 0。
\end{tabular}
\\ \hline
try\_acquire() &
\begin{tabular}[c]{@{}l@{}}
尝试将计数器递减 1,前提是计数器大于 0。如果成功,返回\\ true;否则返回 false。此方法不会阻塞。
\end{tabular}
\\ \hline
try\_acquire\_for() &
\begin{tabular}[c]{@{}l@{}}
尝试将计数器递减 1,前提是计数器大于 0。否则,会阻塞直\\到计数器大于 0 或指定超时时间到达。如果成功减少计数器,\\函数返回 true。
\end{tabular}
\\ \hline
try\_acquire\_until() &
\begin{tabular}[c]{@{}l@{}}
尝试将计数器递减 1,前提是计数器大于 0。否则,会阻塞直\\到计数器大于 0 或指定的时间点过去。如果成功减少计数器,\\函数返回 true。
\end{tabular}
\\ \hline
release() &
\begin{tabular}[c]{@{}l@{}}
将内部计数器递增指定值(默认为 1)。任何因等待计数器大于\\ 0 而阻塞的线程都会被唤醒。
\end{tabular}
\\ \hline
\end{longtable}

\begin{center}
表 8.5: \verb|std::counting_semaphore| 类型的成员函数
\end{center}

这里列出的方法对计数器进行的所有递增和递减操作,都以原子方式执行。

“How to do it...”的最后一个例子展示了如何使用 \verb|binary_semaphore|。这个例子中,一定数量的线程(这个例子中是四个)在一个循环中生成工作并写入共享资源。与之前的例子不同,只是将数据添加到整数vector的末尾。因此,线程之间的vector访问必须同步,这就是二进制信号量发挥作用的地方。每次循环中,线程函数创建一个新值(这可能需要一些时间)。然后,这个值添加到vector的末尾。但线程必须调用信号量的 \verb|acquire()| 方法,以确保其是唯一可以继续执行并访问共享资源的线程。写操作完成后,线程调用信号量的 \verb|release()| 方法以增加内部计数器,允许另一个线程访问共享资源。

信号量可以用于多种目的:阻止对共享资源的访问(类似于互斥锁)、在线程之间发送或传递通知(类似于条件变量),或者实现栅栏,通常比类似的机制性能更好。


