一个线程是一组指令序列,可以由调度器(例如操作系统)独立管理。线程可能是软件的也可能是硬件的,软件线程是由操作系统管理的执行线程。线程可以在单个处理单元上运行,通常通过时间片轮转的方式进行调度。这是一种机制,每个线程在一个处理单元上获得一段执行时间(以毫秒为单位),然后操作系统会安排另一个软件线程在同一处理单元上运行。硬件线程是在物理层面的执行线程,就是一个CPU或CPU核心。可以在多处理器或多核系统上同时运行,即并行运行。许多软件线程可以通过使用时间片轮转的方式,并发地在一个硬件线程上运行,C++库提供了对软件线程工作的支持。本示例中,将介绍如何创建和操作线程。

\mySubsubsection{}{Getting ready}

执行线程由<thread>头文件中的std命名空间下的thread类表示。其他线程工具同样位于该头文件下,但在std::this\_thread命名空间中提供。

在下面的例子中,使用了print\_time()函数。这个函数将本地时间输出到控制台:

\begin{cpp}
inline void print_time()
{
    auto now = std::chrono::system_clock::now();
    auto stime = std::chrono::system_clock::to_time_t(now);
    auto ltime = std::localtime(&stime);
    std::cout << std::put_time(ltime, "%c") << '\n';
}
\end{cpp}

下一节中,将看到如何进行常见的线程操作。

\mySubsubsection{}{How to do it...}

使用以下解决方案来管理线程:

\begin{enumerate}
\item
要创建std::thread实例,而不启动新线程的执行,可以使用其默认构造函数:

\begin{cpp}
std::thread t;
\end{cpp}

\item
通过构建std::thread实例,并传递函数作为参数来在另一个线程上执行函数:

\begin{cpp}
void func1()
{
    std::cout << "thread func without params" << '\n';
}
std::thread t1(func1);
std::thread t2([]() {
    std::cout << "thread func without params" << '\n'; });
\end{cpp}

\item
通过构建一个std::thread实例,并向其构造函数传递函数及其参数,可在另一个线程上运行带有参数的函数:

\begin{cpp}
void func2(int const i, double const d, std::string const s)
{
    std::cout << i << ", " << d << ", " << s << '\n';
}
std::thread t(func2, 42, 42.0, "42");
\end{cpp}

\item
使用std::thread实例上的join()方法等待线程完成其执行:

\begin{cpp}
t.join();
\end{cpp}

\item
使用detach()方法允许线程独立于当前线程对象继续执行。线程将继续执行直到结束,而不会由std::thread实例管理,该实例将不再拥有线程:

\begin{cpp}
t.detach();
\end{cpp}

\item
若要按引用传递参数给函数线程,请将其包装在std::ref或std::cref(如果引用是常量)中:

\begin{cpp}
void func3(int & i)
{
    i *= 2;
}
int n = 42;
std::thread t(func3, std::ref(n));
t.join();
std::cout << n << '\n'; // 84
\end{cpp}

\item
使用std::this\_thread::sleep\_for()停止线程的执行一段时间:

\begin{cpp}
void func4()
{
    using namespace std::chrono;
    print_time();
    std::this_thread::sleep_for(2s);
    print_time();
}
std::thread t(func4);
t.join();
\end{cpp}

\item
使用std::this\_thread::sleep\_until()停止线程的执行直到指定的时间点:

\begin{cpp}
void func5()
{
    using namespace std::chrono;
    print_time();
    std::this_thread::sleep_until(
        std::chrono::system_clock::now() + 2s);
    print_time();
}
std::thread t(func5);
t.join();
\end{cpp}

\item
使用std::this\_thread::yield()暂停当前线程的执行,为其他线程提供执行机会:

\begin{cpp}
void func6(std::chrono::seconds timeout)
{
    auto now = std::chrono::system_clock::now();
    auto then = now + timeout;
    do
    {
        std::this_thread::yield();
    } while (std::chrono::system_clock::now() < then);
}
std::thread t(func6, std::chrono::seconds(2));
t.join();
print_time();
\end{cpp}

\end{enumerate}

\mySubsubsection{}{How it works...}

代表执行线程的std::thread类型有几个构造函数:

\begin{itemize}
\item
默认构造函数仅创建线程实例,但不启动新线程的执行。

\item
移动构造函数创建一个新的线程实例,来表示先前构造的实例。新实例构造后,其他实例将不再与执行线程关联。

\item
带有可变数量参数的构造函数:第一个参数代表顶级线程函数的函数,其余的是要传递给线程函数的参数。参数需要按值传递给线程函数。如果线程函数接受按引用或常量引用传递的参数,则必须用std::ref或std::cref。这些是生成std::reference\_wrapper类型实例的辅助函数模板,该实例将引用包裹在一个可复制和可赋值的对象中。
\end{itemize}

而线程函数不能返回值,但函数可能具有非void的返回类型,返回值会被忽略。如果需要返回值,可以使用共享变量或函数参数来实现。

如果函数以异常终止,则该异常无法在线程启动上下文中通过try...catch语句捕获,程序将调用std::terminate()的方式异常终止。所有异常都必须在执行线程内部捕获,但可以通过std::exception\_ptr实例跨线程传输。

线程开始执行后,即可汇入也可以分离。汇入线程则阻塞当前线程的执行,直到已加入的线程结束其执行。分离线程则将线程对象与其表示的执行线程解耦,使得当前线程和分离线程可以同时执行。分离的线程有时也称为后台线程或守护线程。当程序终止(从主函数返回)时,不会等待仍在运行的分离线程。所以这些线程的栈不会展开,所以这些线程上实例的析构函数不会调用,这可能导致资源泄漏或资源损坏(文件、共享内存等)。

通过调用join()汇入线程,通过调用detach()分离线程。调用了这两个方法中的任何一个,线程就认为是不可汇入的,线程实例可以安全销毁。当线程分离时,可能需要访问的共享数据,需要在整个执行过程中都可用。

分离了一个线程后,就无法再汇入,尝试这样做会导致运行时错误。可以通过使用成员函数joinable()检查线程是否可以汇入,来防止这种情况。

如果线程对象超出作用域并销毁,但未调用join()或detach(),则会调用std::terminate()。

每个线程都有一个可以检索的标识符。对于当前线程,调用std::this\_thread::get\_id()函数。对于由线程实例表示的另一个执行线程,调用其get\_id()方法。

std::this\_thread命名空间中还提供了几个附加的实用函数:

\begin{itemize}
\item
yield()方法提示调度器激活另一个线程。这在实现忙等待例程时很有用,而实际行为特定于实现。调用此函数实际上不会影响线程的执行。

\item
sleep\_for()方法阻止当前线程至少执行指定的时间段(由于调度原因,线程实际睡眠的时间可能比请求的时间长)。

\item
sleep\_until()方法阻止当前线程至少执行到指定的时间点(由于调度原因,睡眠的实际持续时间可能比请求的时间长)。
\end{itemize}

std::thread类要求显式调用join()方法来等待线程完成,这可能会导致编程错误(如上所述)。C++20标准提供了一个新的线程类,名为std::jthread,解决了这一问题。











