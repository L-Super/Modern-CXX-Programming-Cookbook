
std::cout 是 std::ostream 类型的一个全局对象,用于向标准输出控制台写入文本。尽管对其写入操作保证是线程安全的,但这一保障仅限于 operator<{}< 的单次调用。多次此类连续调用 operator<{}< 可能会在调用之间中断并稍后恢复,这使得有必要采用同步机制来避免破坏结果。这适用于所有多线程操作同一输出流的情况。为了简化,C++20 引入了 std::basic\_osyncstream 来提供一种机制,同步写入同一输出流的线程。本示例中,将介绍如何使用这一新工具。

\mySubsubsection{}{How to do it...}

为了同步来自多个线程的输出流写入访问,按以下步骤操作:

\begin{itemize}
\item
包含 <syncstream> 头文件。

\item
定义一个 std::osyncstream 类型的变量来包装共享的输出流,例如 std::cout。

\item
仅使用该包装变量写入输出流。
\end{itemize}

下面的代码展示了这种模式的一个例子:

\begin{cpp}
std::vector<std::jthread> threads;
for (int i = 1; i <= 10; ++i)
{
    threads.push_back(
        std::jthread([](const int id)
            {
                std::osyncstream scout{ std::cout };
                scout << "thread " << id << " running\n";
            }, i));
}
\end{cpp}

\mySubsubsection{}{How it works...}

默认情况下,标准 C++ 流对象 std::cin/std::wcin、std::cout/std::wcout、std::cerr/std::wcerr 和 std::clog/std::wclog 同步了相应的 C 标准输入输出流 stdin、stdout 和 stderr(除非调用了 \verb|std::ios_base::sync_with_stdio()| 函数禁用了这种同步),所以应用到 C++ 流对象上的任何操作都会立即应用于对应的 C 流。此外,访问这些流是保证线程安全的,对 \verb|operator <<| 或 \verb|>>| 的调用是原子的;另一个线程不能访问流直到当前调用完成,但多次调用可以中断:

\begin{cpp}
std::vector<std::jthread> threads;
for (int i = 1; i <= 10; ++i)
{
    threads.push_back(
        std::jthread([](const int id)
            {
                std::cout << "thread " << id << " running\n";
            }, i));
}
\end{cpp}

输出在不同的执行中有所不同:

\begin{shell}
thread thread thread 6 running
thread 2 running
1 running
thread 3 running
5 running
thread 4thread 7 running
thread 10 running
thread 9 running
running
thread 8 running
\end{shell}

线程函数中有三次不同的调用 \verb|operator <<|。虽然每次调用本身是原子的,但线程可以在调用之间挂起,以便其他线程有机会执行。这就是为什么会看到的输出呈现上述形状的原因。

这个问题可以通过几种方式解决。可以使用同步机制,比如互斥锁。但在特定情况下,更简单的解决方案是使用本地的 std::stringstream 实例构建要在控制台上显示的文本,并且只对 \verb|operator<<| 进行一次调用:

\begin{cpp}
std::vector<std::jthread> threads;
for (int i = 1; i <= 10; ++i)
{
    threads.push_back(
        std::jthread([](const int id)
            {
                std::stringstream ss;
                ss << "thread " << id << " running\n";
                std::cout << ss.str();
            }, i));
}
\end{cpp}

经过这些修改后,输出呈现出预期的形式:

\begin{shell}
thread 1 running
thread 2 running
thread 3 running
thread 4 running
thread 5 running
thread 6 running
thread 7 running
thread 8 running
thread 9 running
thread 10 running
\end{shell}

C++20 中,可以使用 std::osyncstream/std::wosyncstream 实例来包装输出流以同步访问。osyncstream 类保证,如果所有来自不同线程的写操作都通过此类的实例发生,则不会有数据竞争。std::basic\_osyncstream 类型包装了一个 std::basic\_syncbuf 实例,后者又包装了一个输出缓冲区,同时包含一个单独的内部缓冲区。这个类在内部缓冲区中累积输出,并在实例销毁或显式调用 emit() 成员函数时,将输出传输给包装的缓冲区。

同步流包装器可以用来同步输出流的访问,不仅仅是 std::ostream/std::wostream(即 std::cout/std::wcout 的类型)。例如,它可以用来同步对字符串流的访问:

\begin{cpp}
int main()
{
    std::ostringstream str{ };
    {
        std::osyncstream syncstr{ str };
        syncstr << "sync stream demo";
        std::cout << "A:" << str.str() << '\n'; // [1]
    }
    std::cout << "B:" << str.str() << '\n';    // [2]
}
\end{cpp}

这里,定义了一个名为 str 的 std::ostringstream 实例。内层代码块中,该实例包装入一个 std::osyncstream 实例中,然后通过这个包装器向字符串流中写入文本 "sync stream demo"。在标记为 [1] 的行中,将字符串流的内容打印到控制台。但是,由于同步流尚未销毁,也没有调用 emit(),因此流缓冲区的内容为空。当同步流超出作用域时,其内部缓冲区的内容转移到包装的流中。因此,在标记为 [2] 的行中,str 字符串流包含了文本 "sync stream demo"。程序的输出如下所示:

\begin{shell}
A:
B:sync stream demo
\end{shell}

可以扩展这个例子来展示 emit() 成员函数如何影响流的行为:

\begin{cpp}
int main()
{
    std::ostringstream str{ };
    {
        std::osyncstream syncstr{ str };
        syncstr << "sync stream demo";
        std::cout << "A:" << str.str() << '\n'; // [1]
        syncstr.emit();
        std::cout << "B:" << str.str() << '\n'; // [2]
        syncstr << "demo part 2";
        std::cout << "C:" << str.str() << '\n'; // [3]
    }
    std::cout << "D:" << str.str() << '\n';    // [4]
}
\end{cpp}

这个第二个例子的第一部分是相同的。在标记为 [1] 的行中,字符串缓冲区的内容为空。但是在调用 emit() 之后,同步流将其内部缓冲区的内容转移给包装的输出流。因此,在标记为 [2] 的行中,字符串缓冲区包含了文本 "sync stream demo"。新的文本 "demo part 2" 通过同步流向字符串流写入,但在执行标记为 [3] 的行之前,这些内容不会转移到字符串流中,所以字符串流的内容没有变化。当内层代码块结束时,同步流的作用域结束,同步流内部缓冲区的新内容再次转移到已包装的字符串流中,现在该字符串流将包含文本 "sync stream demodemo part 2"。因此,这个第二个例子的输出如下:

\begin{shell}
A:
B:sync stream demo
C:sync stream demo
D:sync stream demodemo part
\end{shell}

std::basic\_syncstream 类型有一个名为 \verb|get_wrapped()| 的成员函数,返回一个指向包装流缓冲区的指针。这可以用来构造新的 std::basic\_syncstream 类型实例,可以通过不同的 std::basic\_osyncstream 实例序列化内容到同一个输出流:

\begin{cpp}
int main()
{
    std::ostringstream str{ };
    {
        std::osyncstream syncstr{ str };
        syncstr << "sync stream demo";
        std::cout << "A:" << str.str() << '\n';    // [1]
        {
            std::osyncstream syncstr2{ syncstr.get_wrapped() };
            syncstr2 << "demo part 3";
            std::cout << "B:" << str.str() << '\n'; // [2]
        }
        std::cout << "C:" << str.str() << '\n';    // [3]
    }
    std::cout << "D:" << str.str() << '\n';       // [4]
}
\end{cpp}

同样地,这个例子的第一部分没有变化。但有一个第二个内层代码块,其中构建了 std::osyncstream 的第二个实例,使用的是调用 syncstr 的 \verb|get_wrapped()| 成员函数返回的流缓冲区指针。在标记为 [2] 的行中,两个 std::osyncstream 实例都没有销毁,因此str 字符串流的内容仍然为空。第一个销毁的同步流是 syncstr2,在第二个内层代码块结束时。在标记为 [3] 的行中,字符串流的内容将是 "demo part 3"。然后,第一个同步流对象 syncstr 在第一个内层代码块结束时超出作用域,将文本 "sync stream demo" 添加到字符串流中。运行这个程序的输出如下:

\begin{shell}
A:
B:
C:demo part 3
D:demo part 3sync stream demo
\end{shell}

虽然在所有这些例子中都定义了命名变量,但也可以使用临时同步流来写入输出流:

\begin{cpp}
threads.push_back(
    std::jthread([](const int id)
        {
            std::osyncstream{ std::cout } << "thread " << id
                                          << " running\n";
        }, i));
\end{cpp}

