
标准库提供了几种互斥锁类型来保护对共享资源的访问。\verb|std::recursive_mutex| 和 \verb|std::recursive_timed_mutex| 是两种允许在同一线程中进行多重锁定的实现。递归互斥锁的典型用途是从递归函数中保护对共享资源的访问。一个 \verb|std::recursive_mutex| 可以从一个线程中通过调用 \verb|lock()| 或 \verb|try_lock()| 多次锁定。当一个线程锁定一个可用的递归互斥锁时,获得了该互斥锁的所有权;因此,从同一个线程连续尝试锁定互斥锁不会阻塞线程的执行,从而避免了死锁,并且递归互斥锁只有在进行了相等数量的 \verb|unlock()| 调用后才会释放。递归互斥锁也可能比非递归互斥锁有更大的开销。由于这些原因,应避免使用递归互斥锁。本示例,将介绍如何将递归互斥锁转换为使用非递归互斥锁。


\mySubsubsection{}{Getting ready}


对于本示例,将使用以下类型:

\begin{cpp}
class foo_rec
{
    std::recursive_mutex m;
    int data;
public:
    foo_rec(int const d = 0) : data(d) {}
    void update(int const d)
    {
        std::lock_guard<std::recursive_mutex> lock(m);
        data = d;
    }
    int update_with_return(int const d)
    {
        std::lock_guard<std::recursive_mutex> lock(m);
        auto temp = data;
        update(d);
        return temp;
    }
};
\end{cpp}

本示例的目的是将 \verb|foo_rec| 转换为不使用 \verb|std::recursive_mutex| 的形式。

\mySubsubsection{}{How to do it...}

要将上述实现转换为使用非递归互斥锁的线程安全类型,请按照以下步骤操作:

\begin{enumerate}
\item
将 \verb|std::recursive_mutex| 替换为 \verb|std::mutex|:

\begin{cpp}
class foo
{
    std::mutex m;
    int        data;
    // continued at 2.
};
\end{cpp}

\item
定义私有非线程安全的公共方法或辅助函数,以便在线程安全的公共方法中使用:

\begin{cpp}
void internal_update(int const d) { data = d; }
// continued at 3.
\end{cpp}

\item
重写公共方法以使用新定义的非线程安全的私有方法:

\begin{cpp}
public:
    foo(int const d = 0) : data(d) {}
    void update(int const d)
    {
        std::lock_guard<std::mutex> lock(m);
        internal_update(d);
    }
    int update_with_return(int const d)
    {
        std::lock_guard<std::mutex> lock(m);
        auto temp = data;
        internal_update(d);
        return temp;
    }
\end{cpp}
\end{enumerate}

\mySubsubsection{}{How it works...}

刚才讨论的 \verb|foo_rec| 使用递归互斥锁来保护对共享数据的访问,它会从两个线程安全共函数访问的整型成员变量:

\begin{itemize}
\item
\verb|update()|在私有变量中设置一个新值。

\item
\verb|update_and_return()| 在私有变量中设置一个新值,并将之前的值返回给调用函数。此函数调用 \verb|update()| 来设置新值。
\end{itemize}

\verb|foo_rec| 的实现可能是为了避免代码重复,但这种方法实际上是一个设计错误,可以通过“How to do it...”部分的方式来改进。与其重新使用公共线程安全函数,可以提供私有的非线程安全函数,然后从公共接口调用这些函数。

同样的解决方案也可以应用于其他类似的问题:定义一个非线程安全版本的代码,然后提供可能是轻量级、线程安全的包装器。












