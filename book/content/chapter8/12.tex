std::thread 表示单一的执行线程,允许多个函数并发执行,但有一个主要的不便之处:必须显式地调用 join() 方法来等待线程完成执行。这可能会导致问题,如果 std::thread 对象在仍然可连接(joinable)的情况下销毁,会调用 std::terminate()。C++20 提供了一个改进的线程类叫做 std::jthread(从可连接线程而来),当对象销毁时,如果线程仍然是可汇入,会自动调用 join()。此外,这种类型支持通过 \verb|std::stop_source/std::stop_token| 进行取消,并且其析构函数也会在连接之前请求线程停止。这个示例中,将介绍如何使用这些 C++20 类型。

\mySubsubsection{}{Getting ready}

要使用 std::jthread,需要包含相同的 <thread> 头文件。对于 \verb|std::stop_source| 和 \verb|std::stop_token|,你需要包含头文件 \verb|<stop_token>|。

\mySubsubsection{}{How to do it...}

使用可汇入线程和合作取消机制的典型场景如下:

\begin{itemize}
\item
如果想在线程对象超出作用域时自动汇入线程,使用 std::jthread。仍然可以使用 std::thread 所有的方法,比如显式地使用 join():

\begin{cpp}
void thread_func(int i)
{
    while(i-- > 0)
    {
        std::cout << i << '\n';
    }
}
int main()
{
    std::jthread t(thread_func, 10);
}
\end{cpp}

\item
如果需要能够取消线程的执行,应该按照以下步骤操作:

\begin{itemize}
\item
确保线程函数的第一个参数是一个 \verb|std::stop_token| 实例。

\item
线程函数中,定期使用 \verb|std::stop_token| 实例的 \verb|stop_requested()| 方法检查是否请求停止,并在收到信号时停止。

\item
使用 std::jthread 在单独的线程上执行函数。

\item
从调用线程使用 std::jthread 实例的 \verb|request_stop()| 方法请求线程函数停止并返回:

\begin{cpp}
void thread_func(std::stop_token st, int& i)
{
    while(!st.stop_requested() && i < 100)
    {
        using namespace std::chrono_literals;
        std::this_thread::sleep_for(200ms);
        i++;
    }
}
int main()
{
    int a = 0;

    std::jthread t(thread_func, std::ref(a));

    using namespace std::chrono_literals;
    std::this_thread::sleep_for(1s);

    t.request_stop();

    std::cout << a << '\n';       // prints 4
}
\end{cpp}
\end{itemize}

\item
如果需要取消多个线程的工作,可以按照以下步骤操作:

\begin{itemize}
\item
所有线程函数都必须将 \verb|std::stop_token| 实例作为第一个参数。

\item
所有线程函数应该定期调用 \verb|std::stop_token| 的 \verb|stop_requested()| 方法检查是否请求停止,如果请求停止,则终止执行。

\item
使用 std::jthread 在不同的线程上执行函数。

\item
调用线程中创建一个 \verb|std::stop_source| 实例。

\item
通过调用 \verb|std::stop_source| 实例的 \verb|get_token()| 方法获取一个 \verb|std::stop_token| 实例,当创建 std::jthread 实例时,将其作为第一个参数传递给线程函数。

\item
当想要停止线程函数的执行时,调用 \verb|std::stop_source| 实例的 \verb|request_stop()| 方法。

\begin{cpp}
void thread_func(std::stop_token st, int& i)
{
    while(!st.stop_requested() && i < 100)
    {
        using namespace std::chrono_literals;
        std::this_thread::sleep_for(200ms);
        i++;
    }
}
int main()
{
    int a = 0;
    int b = 10;

    std::stop_source st;

    std::jthread t1(thread_func, st.get_token(),
    std::ref(a));
    std::jthread t2(thread_func, st.get_token(),
    std::ref(b));

    using namespace std::chrono_literals;
    std::this_thread::sleep_for(1s);

    st.request_stop();

    std::cout << a << ' ' << b << '\n';       // prints 4
    // and 14
}
\end{cpp}
\end{itemize}

\item
如果需要在一个停止源请求取消时执行一段代码,可以使用 \verb|std::stop_callback| 创建与 \verb|std::stop_token| 实例关联的回调函数,该回调函数会在停止请求通过 \verb|std::stop_source| 实例发出时调用:

\begin{cpp}
void thread_func(std::stop_token st, int& i)
{
    while(!st.stop_requested() && i < 100)
    {
        using namespace std::chrono_literals;
        std::this_thread::sleep_for(200ms);
        i++;
    }
}
int main()
{
    int a = 0;

    std::stop_source src;
    std::stop_token token = src.get_token();
    std::stop_callback cb(token, []{std::cout << "the end\n";});

    std::jthread t(thread_func, token, std::ref(a));

    using namespace std::chrono_literals;
    std::this_thread::sleep_for(1s);

    src.request_stop();

    std::cout << a << '\n';       // prints "the end" and 4
}
\end{cpp}

\end{itemize}

\mySubsubsection{}{How it works...}

std::jthread 类似于 std::thread,它是为了弥补 C++11 中线程所缺失的功能而做出的努力。其公共接口与 std::thread 相似。std::thread 所有的方法也存在于 std::jthread 中,但在以下几个关键方面有所不同:

\begin{itemize}
\item
内部至少逻辑上维护了一个共享停止状态,这允许请求线程函数停止执行。

\item
几个用于处理合作取消的方法:\verb|get_stop_source()|,返回与线程共享停止状态相关联的 \verb|std::stop_source| 实例;\verb|get_stop_token()|,返回与线程共享停止状态相关联的 \verb|std::stop_token| 实例;以及 \verb|request_stop()|,通过共享停止状态请求取消线程函数的执行。

\item
其析构函数的行为,当线程可连接时,会先调用 \verb|request_stop()| 请求停止执行,然后调用 join() 等待直到线程完成其执行。
\end{itemize}

可以像创建 std::thread 实例一样创建 std::jthread 实例,但传递给 std::jthread 的可调用函数可以有一个类型为 \verb|std::stop_token| 的第一个参数。当要能够取消线程的执行时,需要这样做。

典型的场景包括图形用户界面中用户交互可能取消正在进行的工作,但也可以设想许多其他情况。调用这样的线程函数发生如下:

\begin{itemize}
\item
如果在线程函数构造 std::jthread 时提供的第一个参数是 \verb|std::stop_token|,则会转发给可调用函数。

\item
如果有参数,且可调用函数的第一个参数不是一个 \verb|std::stop_token| 实例,则与 std::jthread 实例内部共享停止状态相关联的 \verb|std::stop_token| 实例会传递给函数。此令牌通过调用 \verb|get_stop_token()| 获得。
\end{itemize}

线程函数必须定期检查 \verb|std::stop_token| 实例的状态。\verb|stop_requested()| 方法用来检查是否请求了停止。停止请求来自 \verb|std::stop_source| 实例。

如果多个停止令牌与同一个停止源关联,停止请求对所有停止令牌都可见。请求停止,就不能撤回,连续的停止请求也没有意义。要请求停止,应该调用 \verb|request_stop()| 方法。可以通过调用 \verb|stop_possible()| 方法检查 \verb|std::stop_source| 是否与停止状态关联并可以请求停止。

如果需要在停止源请求停止时调用回调函数,可以使用 \verb|std::stop_callback| 类型。这将一个 \verb|std::stop_token| 对象与一个回调函数关联起来。当停止令牌的停止源请求停止时,就会调用回调函数。回调函数的调用如下:

\begin{itemize}
\item
在同一调用 \verb|request_stop()| 的线程中。

\item
如果在构建停止回调对象之前已请求停止,则在构建 \verb|std::stop_callback| 示例的线程中。
\end{itemize}

可以为同一个停止令牌创建任意数量的 \verb|std::stop_callback| 实例,但回调函数调用的顺序未指定。唯一可以保证是,如果停止是在构建 \verb|std::stop_callback| 实例之后请求,将会同步执行。

还有一点很重要,如果回调函数通过异常返回,将会调用 std::terminate()。

