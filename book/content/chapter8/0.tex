主流计算机包含多个处理器或至少是多核,并且利用这种计算能力对于许多类别的应用程序至关重要。不幸的是,许多开发人员仍然持有顺序代码执行的思维模式,即使那些相互之间没有依赖关系的操作可以并行执行。本章介绍了标准库对线程、异步任务及相关组件的支持,以及一些实践示例。

大多数现代处理器(除了那些专用于不需要大量计算能力的应用程序的处理器,如物联网应用)具有两个、四个或更多的核心,能够同时执行多个执行线程。应用程序必须明确地编写以利用存在的多个处理单元;可以通过同时在多个线程上执行函数,来编写这样的应用程序。C++11起,标准库提供了线程工作、共享数据同步、线程通信和异步任务的支持。本章将探讨与线程和任务相关的最重要主题。

本章包括以下内容:

\begin{itemize}
\item
使用线程

\item
使用互斥锁和锁同步访问共享数据

\item
寻找递归互斥锁的替代方案

\item
处理来自线程函数的异常

\item
线程间发送通知

\item
使用promise和future从线程返回值

\item
异步执行函数

\item
使用原子类型

\item
使用线程实现并行映射和折叠

\item
使用任务实现并行映射和折叠

\item
使用标准并行算法实现并行映射和折叠

\item
使用可汇入的线程和取消机制

\item
使用门闩、栅栏和信号量同步线程

\item
同步从多个线程写入输出流
\end{itemize}
