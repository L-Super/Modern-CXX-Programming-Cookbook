
C++ 是非常受欢迎的编程语言,在设计上注重性能、效率和灵活性,融合了面向对象、命令式、泛型和函数式编程等多种范式。国际标准化组织(ISO)对 C++ 进行了标准化,并在过去的时间里对该语言进行了大规模的改革。随着 C++11 标准化的完成,C++ 进入了一个新时代,称为“现代 C++”。类型推断、移动语义、lambda 表达式、智能指针、统一初始化、变长模板等众多新特性,改变了编写 C++ 代码的方式,使其看起来像是一种全新的编程语言。随着 C++20 的发布,这种变革还在深化,该版本包括了许多新的语言特性,如模块、概念、协程,以及标准库的变化,如范围、文本格式化和日历功能。而今,随着 C++23 的引入和即将到来的 C++26,这门语言正在经历更多的变化。

本书涵盖了 C++11、C++14、C++17、C++20 和 C++23 中的许多新特性。全书按照示例的形式编排,每个示例聚焦于一个特定的语言或库特性,或是开发者常见的问题,以及使用现代 C++ 的解决方案。通过跟随示例的学习,将掌握核心语言特性和标准库的使用,包括字符串、容器、算法、迭代器、流、正则表达式、线程、文件系统、原子操作、工具类和范围等内容。

本书第三版的撰写耗时数月,在此期间 C++23 标准的工作已经完成。然而,撰写序言时,该标准尚未获得批准,预计将于今年(2024 年)公布。

第二版和第三版中超过 30 个新的或更新的示例,涵盖了 C++20 的特性,包括模块、概念、协程、范围、线程和同步机制、文本格式化、日历和时区、立即函数、三向比较操作符和新的 std::span 类型。此外,第三版中近 20 个新的或更新的示例涉及 C++23 的特性,如 std::expected 类、std::mdspan 类、堆栈跟踪库、span 缓冲区、多维下标操作符,以及对文本格式库的扩展。

书中所有示例均包含代码,展示了如何使用某一特性或解决某个问题。这些代码示例是在 Visual Studio 2022 上编写的,也使用 Clang 和 GCC 进行了编译。鉴于各种语言和库特性在这些编译器中的支持逐步增加,建议使用每个编译器的最新版本,以确保所有新特性都能得到支持。

撰写此序言时,最新的版本为 GCC 14.0、Clang 18.0 和 VC++ 2022 版本 14.37(来自 Visual Studio 2019 版本 17.7)。尽管所有这些编译器完全支持 C++17,但它们对 C++23 的支持程度各不相同。请参考  \url{https://en.cppreference.com/w/cpp/compiler_support} 查看相应编译器对 C++23 特性的支持情况。

\mySubsectionNoFile{}{适读人群}

本书面向所有 C++ 开发者,无论其经验多少。很多读者是初学者或中级 C++ 开发者,也希望深入掌握这门语言,成为熟练的现代 C++ 开发者。对于有经验的 C++ 开发者而言,本书也是一本很好的参考资料,详细介绍了 C++11、C++14、C++17、C++20 和 C++23 的语言及库特性,这些内容在日常开发中可能会非常有用。本书包含了超过 150 个示例,涵盖了从基础到中级乃至高级的主题。不过,所有这些内容都需要读者需要具备一定的 C++ 知识,包括但不限于函数、类、模板、命名空间、宏等方面的内容。如果对 C++ 尚不熟悉,建议首先阅读一本入门书籍,对语言有所了解之后,再来学习本书中的内容。

\mySubsectionNoFile{}{关于本书}

第1章 \textit{现代语言的核心特性},包括类型推断、统一初始化、作用域枚举、基于范围的 for 循环、结构化绑定、类模板参数推导等内容。

第2章 \textit{数字与字符串},介绍了数字与字符串之间的转换、伪随机数的生成、正则表达式的使用、多种类型的字符串处理,以及如何使用 C++20 文本格式化库进行文本格式化。

第3章 \textit{探索函数},深入介绍了默认(=default)和删除(=delete)函数、变长模板、lambda 表达式以及高阶函数。

第4章 \textit{预处理与编译},介绍了编译过程中的各个方面,从条件编译、编译时断言、代码生成,到使用属性提示编译器的方法。

第5章 \textit{标准库容器、算法与迭代器},介绍了几种标准容器、多个算法,并演示如何编写自己的随机访问迭代器。

第6章 \textit{通用工具},深入介绍了 chrono 库,包括 C++20 对日历和时区的支持;any、optional、variant 类型,以及 span 和 mdspan 类型;当然,还有类型特征。

第7章 \textit{文件与流}, 介绍了如何读写流中的数据、使用 I/O 控制符管理流,以及探索文件系统库。

第8章 \textit{线程与并发},介绍如何使用线程、互斥锁、锁、条件变量、promise、future、原子类型,以及 C++20 中的锁存器(门闩)、栅栏和信号量。

第9章 \textit{健壮性与性能},关注异常处理、常量正确性、类型转换、智能指针和移动语义。

第10章 \textit{实现模式与惯用法},介绍了各种模式和惯用法,如 pimpl 惯用法、非虚接口惯用法、奇特递归模板模式和混入(mixins)模式。

第11章 \textit{探索测试框架},提供了三个最常用的测试框架 Boost.Test、Google Test 和 Catch2 的入门指导。

第12章 \textit{C++20的核心特性},引入了 C++20 标准最重要的新增特性——模块、概念、协程和范围,也包括 C++23 中的更新。

\mySubsectionNoFile{}{新版本修改的内容}

本节提供了一份新增或更新的示例列表,以及简短的更新说明

\textbf{第1章,现代语言的核心特性}

\begin{itemize}
\item
使用作用域枚举:示例更新至 C++23,增加了std::to\_underlying和std::is\_scoped\_enum

\item
使用基于范围的 for 循环遍历范围:示例更新至 C++23,加入了初始化语句。

\item
使用下标操作符访问集合中的元素:(新增)介绍了 C++23 的多维下标操作符。
\end{itemize}

\textbf{第2章,数字与字符串}

\begin{itemize}
\item
不同的数值类型:(新增)介绍了 C++ 的数值类型。

\item
不同的字符和字符串类型:(新增)介绍了 C++ 的字符类型。

\item
向输出控制台打印 Unicode 字符:(新增)讨论了如何处理 UNICODE 及控制台输出。

\item
创建字符串助手库:示例更新至 C++20,增加了 starts\_with(), ends\_with()和contains()。

\item
使用 std::format 和 std::print 格式化和打印文本:示例更新至 C++23,增加了 std::print() 和 std::println() 函数。

\item
使用 std::format 与用户定义类型:示例更新至 C++23,增加了 std::formattable 支持和更好的示例。
\end{itemize}

\textbf{第3章,探索函数}

\begin{itemize}
\item
使用 lambda 表达式与标准算法:示例更新至 C++23,增加了函数调用操作符上的属性。

\item
编写递归 lambda 表达式:示例更新至 C++14,增加了递归泛型 lambda 表达式。

\item
编写函数模板:(新增)提供了编写函数模板的指南。
\end{itemize}

\textbf{第4章,预处理与编译}

\begin{itemize}
\item
条件编译源代码:示例更新至 C++23,新增了 \#warning、\#elifdef 和 \#elifndef。

\item
使用 static\_assert 执行编译时断言检查:示例更新至 C++26,增加了用户生成的消息。

\item
使用属性向编译器提供元数据:示例更新至 C++23,新增了 lambda 表达式的属性、[[assume]] 属性和重复属性。
\end{itemize}

\textbf{第5章,标准库容器、算法与迭代器}

\begin{itemize}
\item
使用 vector 作为默认容器:示例更新至 C++23,新增了对范围感知成员函数的支持。

\item
选择合适的标准容器:(新增)对比了不同的标准容器。
\end{itemize}

\textbf{第6章,通用工具}

\begin{itemize}
\item
使用日历:示例更新至 C++20。

\item
时区之间转换时间:示例更新至 C++20。

\item
可能产生或不产生值的计算:(新增)介绍了 C++23 中 std::optional 的一元操作。

\item
使用 std::expected 返回值或错误:(新增)介绍了 C++23 中的 std::expected 类型。

\item
使用 std::mdspan 获取对象序列的多维视图:(新增)介绍了 C++23 中的 std::mdspan 类型。

\item
使用 source\_location 提供日志:(新增)介绍了 C++20 中的 std::source\_location 类型。

\item
使用堆栈跟踪库打印调用堆栈:(新增)介绍了 C++23 中的堆栈跟踪库。
\end{itemize}

\textbf{第7章,文件与流}

\begin{itemize}
\item
在确定大小的外部缓冲区上使用流:(新增)介绍了 C++23 中的 span 缓冲区。
\end{itemize}

\textbf{第8章,线程与并发}

\begin{itemize}
\item
使用线程同步机制:示例更新至 C++20。

\item
同步输出流:(新增)介绍了 C++20 中的同步流。
\end{itemize}

\textbf{第9章,健壮性与性能}

\begin{itemize}
\item
创建编译时常量表达式:示例更新至 C++23,增加了静态 constexpr 变量。

\item
常量求值上下文中优化代码:(新增)介绍了 C++23 中的 if consteval。

\item
常量表达式中使用虚函数调用:(新增)介绍了 C++20 中的 constexpr 虚函数。
\end{itemize}

\textbf{第10章,实现模式与惯用法}

\begin{itemize}
\item
使用混入模式向类添加功能:(新增)介绍了混入模式。

\item
使用类型擦除泛型处理无关类型:(新增)介绍了类型擦除惯用法。

\item
向输出控制台打印 Unicode 字符:(新增)介绍了处理 UNICODE 和控制台输出。

\item
创建字符串助手库:介绍更新至 C++20,增加了 starts\_with()、ends\_with() 和 contains() 函数。

\item
使用 std::format 和 std::print 格式化和打印文本:介绍更新至 C++23,增加了 std::print() 和 std::println() 函数。

\item
使用 std::format 与用户定义类型:介绍更新至 C++23,增加了 std::formattable 支持和更好的示例。
\end{itemize}

\textbf{第11章,探索测试框架}

\begin{itemize}
\item
入门 Catch2:更新了 Catch2 版本 3.4.0 的安装说明。

\item
使用 Catch2 断言:更新了 Catch2 版本 3.4.0 的示例。
\end{itemize}

\textbf{第12章,C++20的核心特性}

\begin{itemize}
\item
探索简化函数模板:(新增)介绍了 C++20 中的简化函数模板。

\item
探索标准范围适配器:(新增)介绍了 C++20 和 C++23 中的范围适配器。

\item
将范围转换为容器:(新增)介绍了如何使用 C++23 中的 std::ranges::to() 将范围转换为标准容器。

\item
使用受约束的算法:(新增)探索了直接用于范围的通用算法。

\item
创建用于异步计算的协程任务:更新至标准示例(并提供了 libcoro 库的平替方案)。

\item
创建用于值序列的协程生成器类型:更新至标准示例(并提供了 libcoro 库的平替方案)。

\item
使用 std::generator 类型递归生成值:(新增)介绍了 C++23 中的 std::generator。
\end{itemize}

\mySubsectionNoFile{}{积极实践}

本书的代码可以从 \url{https://github.com/PacktPublishing/Modern-Cpp-Programming-Cookbook-Third-Edition}获取,希望各位读者自己动手编写所有的示例代码。为了编译这些代码,需要在 Windows 上使用 VC++ 2022 17.7 版本,在 Linux 和 Mac 上使用 GCC 14.0 或 Clang 18.0。如果没有最新版本的编译器,或者想尝试其他编译器,可以使用在线提供的编译器。

虽然有许多在线平台可以选择,但推荐使用 \href{https://wandbox.org/}{Wandbox}和\href{https://godbolt.org/}{Compiler Explorer}。这两个平台非常适合尝试和测试 C++ 代码。

\mySubsectionNoFile{}{源码下载}

本书的代码托管在 GitHub 上,地址为 \url{https://github.com/PacktPublishing/Modern-Cpp-Programming-Cookbook-Third-Edition}。此外,还可以在 \url{https://github.com/PacktPublishing/} 浏览图书和视频目录中的其他代码包。欢迎查看!

\mySubsectionNoFile{}{彩图下载}

还提供了一个PDF文件,其中包含了本书中使用的屏幕截图/图表的彩色图片。可以在这里下载: \url{https://packt.link/gbp/9781835080542}






