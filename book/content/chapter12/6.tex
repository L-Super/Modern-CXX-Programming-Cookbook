
C++ 标准库提供了三个重要的支柱——容器、迭代器和算法——能够处理集合。因为这些算法是为了通用目的设计,并且旨在与定义了范围的迭代器一起工作,所以通常需要编写显式且有时复杂的代码来完成简单的任务。C++20 范围库就为了解决这个问题而设计,提供了处理元素范围的组件。这些组件包括范围适配器(或视图)和适用于范围而非迭代器的受约束算法。本示例中,将介绍其中的一些视图和算法,并了解其如何简化编码。

\mySubsubsection{}{Getting ready}

接下来的代码中,将引用一个名为 \verb|is_prime()| 的函数,该函数接受一个整数并返回一个布尔值,指示该数字是否为素数。这里展示了一个简单的实现:

\begin{cpp}
bool is_prime(int const number)
{
    if (number != 2)
    {
        if (number < 2 || number % 2 == 0) return false;
        auto root = std::sqrt(number);
        for (int i = 3; i <= root; i += 2)
        if (number % i == 0) return false;
    }
    return true;
}
\end{cpp}

\begin{myNotic}
对于高效的算法,虽然超出了本示例的范畴,但我推荐使用 Miller–Rabin 素性测试。
\end{myNotic}

范围库可以在新的 <ranges> 头文件中找到,位于 std::ranges 命名空间中。为了简便起见,本实例中将使用以下命名空间别名:

\begin{cpp}
namespace rv = std::ranges::views;
namespace rg = std::ranges;
\end{cpp}

接下来的部分中,将探讨范围库的各种用途。

\mySubsubsection{}{How to do it...}

范围库可以用于通过以下操作遍历范围:

\begin{itemize}
\item
使用 \verb|iota_view| / \verb|views::iota| 视图生成连续整数序列。以下代码输出从 1 到 9 的所有整数:

\begin{cpp}
for (auto i : rv::iota(1, 10))
    std::cout << i << ' ';
\end{cpp}

\item
使用 \verb|filter_view| / \verb|views::filter| 过滤范围中的元素,只保留满足谓词的元素。第一个代码输出从 1 到 99 的所有素数。然而,第二个代码段保留并打印来自整数vector的所有素数:

\begin{cpp}
// prints 2 3 5 7 11 13 ... 79 83 89 97
for (auto i : rv::iota(1, 100) | rv::filter(is_prime))
    std::cout << i << ' ';
// prints 2 3 5 13
std::vector<int> nums{ 1, 1, 2, 3, 5, 8, 13, 21 };
for (auto i : nums | rv::filter(is_prime))
    std::cout << i << ' ';
\end{cpp}

\item
使用 \verb|transform_view| / \verb|views::transform| 应用一元函数转换范围中的每个元素。以下代码输出从 1 到 99 的所有素数的后继数:

\begin{cpp}
// prints 3 4 6 8 12 14 ... 80 84 90 98
for (auto i : rv::iota(1, 100) |
              rv::filter(is_prime) |
              rv::transform([](int const n) {return n + 1; }))
    std::cout << i << ' ';
\end{cpp}

\item
使用 \verb|take_view| / \verb|views::take| 保留视图的前 N 个元素。以下代码仅输出从 1 到 99 的前 10 个素数:

\begin{cpp}
// prints 2 3 5 7 11 13 17 19 23 29
for (auto i : rv::iota(1, 100) |
              rv::filter(is_prime) |
              rv::take(10))
    std::cout << i << ' ';
\end{cpp}

\item
使用 \verb|reverse_view| / \verb|views::reverse| 反向遍历范围。第一个代码输出从 99 到 1 的前 10 个素数(降序),而第二个代码输出从 1 到 99 的最后 10 个素数(升序):

\begin{cpp}
// prints 97 89 83 79 73 71 67 61 59 53
for (auto i : rv::iota(1, 100) |
              rv::reverse |
              rv::filter(is_prime) |
              rv::take(10))
    std::cout << i << ' ';
// prints 53 59 61 67 71 73 79 83 89 97
for (auto i : rv::iota(1, 100) |
              rv::reverse |
              rv::filter(is_prime) |
              rv::take(10) |
              rv::reverse)
    std::cout << i << ' ';
\end{cpp}

\item
使用 \verb|drop_view| / \verb|views::drop| 跳过范围的前 N 个元素。以下代码按升序输出从 1 到 99 的素数,但跳过了序列中的前 10 个和最后 10 个素数:

\begin{cpp}
// prints 31 37 41 43 47
for (auto i : rv::iota(1, 100) |
              rv::filter(is_prime) |
              rv::drop(10) |
              rv::reverse |
              rv::drop(10) |
              rv::reverse)
    std::cout << i << ' ';
\end{cpp}
\end{itemize}

范围库还可以用于使用范围而非迭代器调用算法,大多数算法都有为此目的的重载版本。以下是一些示例:

\begin{itemize}
\item
确定范围的最大元素:

\begin{cpp}
std::vector<int> v{ 5, 2, 7, 1, 4, 2, 9, 5 };
auto m = rg::max(v); // 5
\end{cpp}

\item
排序范围:

\begin{cpp}
rg::sort(v); // 1 2 2 4 5 5 7 9
\end{cpp}

\item
复制范围。以下代码将范围的元素复制到标准输出流:

\begin{cpp}
rg::copy(v, std::ostream_iterator<int>(std::cout, " "));
\end{cpp}

\item
反转范围中的元素:

\begin{cpp}
rg::reverse(v);
\end{cpp}

\item
计算范围中的元素(满足谓词的元素):

\begin{cpp}
auto primes = rg::count_if(v, is_prime);
\end{cpp}
\end{itemize}

\mySubsubsection{}{How it works...}

C++20 范围库提供了处理元素范围的各种组件。包括:

\begin{itemize}
\item
范围概念,如 range 和 view。

\item
范围访问函数,如 begin()、end()、size()、empty() 和 data()。

\item
创建元素序列的范围工厂,如 \verb|empty_view|、\verb|single_view| 和 \verb|iota_view|。

\item
创建范围的惰性求值视图的范围适配器,如 \verb|filter_view|、\verb|transform_view|、\verb|take_view| 和 \verb|drop_view|。
\end{itemize}

范围定义为可以通过迭代器和结束哨兵遍历的一系列元素。根据定义范围的迭代器的能力,范围有不同的类型。以下概念定义了范围的类型:

% Please add the following required packages to your document preamble:
% \usepackage{longtable}
% Note: It may be necessary to compile the document several times to get a multi-page table to line up properly
\begin{longtable}{|l|l|l|}
\hline
\textbf{概念}     & \textbf{概念}  & \textbf{能力}                       \\ \hline
\endfirsthead
%
\endhead
%
input\_range         & input\_iterator         & 至少可以一次读取地遍历。  \\ \hline
output\_range        & output\_iterator        & 可以写入地遍历。                \\ \hline
forward\_range       & forward\_iterator       & 可以多次遍历。             \\ \hline
bidirectional\_range & bidirectional\_iterator & 也可以反向遍历。      \\ \hline
random\_access\_range & random\_access\_iterator & 元素可以常数时间内随机访问。 \\ \hline
contiguous\_range    & contiguous\_iterator    & 元素在内存中连续存储。 \\ \hline
\end{longtable}

\begin{center}
表 12.1: 定义范围类型的概念列表
\end{center}

\verb|forward_iterator| 满足 \verb|input_iterator| 要求,\verb|bidirectional_iterator| 满足 \verb|forward_iterator| 要求,依此类推(从前向后看上表),所以范围也是如此。\verb|forward_range| 满足 \verb|input_range| 要求,\verb|bidirectional_range| 满足 \verb|forward_range| 要求,依此类推。除了上表列出的范围概念之外,还有其他范围概念。值得一提的是 \verb|sized_range|,要求范围必须能在常数时间内知道其大小。

标准容器满足不同范围概念的要求。最重要的几个列在下表中:

% Please add the following required packages to your document preamble:
% \usepackage{longtable}
% Note: It may be necessary to compile the document several times to get a multi-page table to line up properly
\begin{longtable}{|l|l|l|l|l|l|}
\hline
\textbf{} & \textbf{输入范围} & \textbf{前向范围} & \textbf{双向范围} & \textbf{随机访问范围} & \textbf{连续范围} \\ \hline
\endfirsthead
%
\endhead
%
forward\_list       & √ & √ &   &   &   \\ \hline
list                & √ & √ & √ &   &   \\ \hline
dequeue             & √ & √ & √ & √ &   \\ \hline
array               & √ & √ & √ & √ & √ \\ \hline
vector              & √ & √ & √ & √ & √ \\ \hline
set                 & √ & √ & √ &   &   \\ \hline
map                 & √ & √ & √ &   &   \\ \hline
multiset            & √ & √ & √ &   &   \\ \hline
multimap            & √ & √ & √ &   &   \\ \hline
unordered\_set      & √ & √ &   &   &   \\ \hline
unordered\_map      & √ & √ &   &   &   \\ \hline
unordered\_multiset & √ & √ &   &   &   \\ \hline
unordered\_multimap & √ & √ &   &   &   \\ \hline
\end{longtable}

\begin{center}
表 12.2: 标准容器及其满足的要求列表
\end{center}

范围库的一个核心概念是范围适配器,也称为视图。视图是对元素范围的非拥有包装,要求常数时间内复制、移动或赋值元素。视图是范围的可组合适应,而这些适应只有在视图迭代时才会惰性发生。

上一节中,介绍了使用各种视图的例子:过滤、转换、获取、跳过和反转。库中共有16种视图可用。所有视图都在 std::ranges 命名空间中,名称如 \verb|filter_view|、\verb|transform_view|、\verb|take_view|、\verb|drop_view| 和 \verb|reverse_view|。但为了使用的简便性,这些视图可以用 \verb|views::filter|、\verb|views::take|、\verb|views::reverse| 等形式的表达式来使用。注意,这些表达式的类型和值未指定,属于编译器实现细节。

要理解视图是如何工作的,来看下面的例子:

\begin{cpp}
std::vector<int> nums{ 1, 1, 2, 3, 5, 8, 13, 21 };
auto v = nums | rv::filter(is_prime) | rv::take(3) | rv::reverse;
for (auto i : v) std::cout << i << ' '; // prints 5 3 2
\end{cpp}

这段代码中,v 表示一个视图。它不会计算所适应的范围,直到开始迭代元素为止。这个例子中,这是通过 for 语句完成的。视图是惰性的,管道操作符(|)重载以简化视图的组合。

视图的组合等同于以下代码:

\begin{cpp}
auto v = rv::reverse(rv::take(rv::filter(nums, is_prime), 3));
\end{cpp}

一般来说,适用以下规则:

\begin{itemize}
\item
如果适配器 A 只接受一个参数,即一个范围 R,则 A(R) 和 R|A 等价。

\item
如果适配器 A 接受多个参数,即一个范围 R 和 args...,则以下三种情况等价:A(R, args...)、A(args...)(R) 和 R|A(args...)。
\end{itemize}

除了范围和范围适配器(或视图)之外,C++20 在同一个 std::ranges 命名空间中还提供了通用算法的重载版本。这些重载版本称为受约束的算法。可以提供一个单独的范围作为参数(如本配方中的例子所示),或者提供一个迭代器-哨兵对。此外,对于这些重载版本,返回类型已更改,以便提供在算法执行期间计算出额外的信息。

\mySubsubsection{}{There's more...}

标准范围库的设计基于 Eric Niebler 创建的 range-v3 库,该库可在 GitHub 上获得,网址为 \url{https://github.com/ericniebler/range-v3}。这个库提供了一组更大的范围适配器(视图),以及提供修改操作的动作(如排序、删除、洗牌等)。从 range-v3 库过渡到 C++20 范围库可以非常平滑。实际上,本示例提供的所有示例都可以在这两个库中工作。所需要做的就是包含适当的头文件,并使用特定于 range-v3 的命名空间:

\begin{cpp}
#include "range/v3/view.hpp"
#include "range/v3/algorithm/sort.hpp"
#include "range/v3/algorithm/copy.hpp"
#include "range/v3/algorithm/reverse.hpp"
#include "range/v3/algorithm/count_if.hpp"
#include "range/v3/algorithm/max.hpp"
namespace rv = ranges::views;
namespace rg = ranges;
\end{cpp}

有了这些替换,"How to do it..."部分中的所有代码可以使用支持 C++17 标准的编译器。


