有时,能够存储值或空指针(当特定值不可用时)的类型会很有用。一个典型的例子就是一个可能无法生成返回值的函数,但这种失败并不是错误。例如,一个通过指定键从字典中查找并返回值的函数。找不到值是一种可能的情况,该函数可能会返回一个布尔值(需要更多的错误码,则可能是整数值)和一个引用参数来保存返回值,或者返回一个指针(原始指针或智能指针)。在 C++17 中,\verb|std::optional| 是这些解决方案的更好的选择。类型模板 \verb|std::optional| 是一个用于存储可能存在值的模板容器。本示例中,将介绍如何使用这个容器。

\mySubsubsection{}{Getting ready}

类型模板 \verb|std::optional<T>| 基于 \verb|boost::optional| 设计,并且可以在 \verb|<optional>| 头文件中找到。如果熟悉 \verb|boost::optional| 并且在代码中使用过,则可以无缝迁移到 \verb|std::optional|。

在接下来的代码中,将使用以下的 \verb|foo| 类型:

\begin{cpp}
struct foo
{
    int    a;
    double b;
};
\end{cpp}

\mySubsubsection{}{How to do it...}

使用以下操作来操作 \verb|std::optional|:

\begin{itemize}
\item
存储值时,可以使用构造函数或将值赋给 \verb|std::optional| 实例:

\begin{cpp}
std::optional<int> v1;      // v1 is empty
std::optional<int> v2(42);  // v2 contains 42
v1 = 42;                    // v1 contains 42
std::optional<int> v3 = v2; // v3 contains 42
\end{cpp}

\item
读取存储的值时,使用操作符 * 或 ->:

\begin{cpp}
std::optional<int> v1{ 42 };
std::cout << *v1 << '\n';   // 42
std::optional<foo> v2{ foo{ 42, 10.5 } };
std::cout << v2->a << ", "
          << v2->b << '\n'; // 42, 10.5
\end{cpp}

\item
或者,使用成员函数 \verb|value()| 和 \verb|value_or()| 来读取存储的值:

\begin{cpp}
std::optional<std::string> v1{ "text"s };
std::cout << v1.value() << '\n'; // text
std::optional<std::string> v2;
std::cout << v2.value_or("default"s) << '\n'; // default
\end{cpp}

\item
要检查容器是否存储了值,可以使用转换为布尔值的操作符或成员函数 \verb|has_value()|:

\begin{cpp}
std::optional<int> v1{ 42 };
if (v1) std::cout << *v1 << '\n';
std::optional<foo> v2{ foo{ 42, 10.5 } };
if (v2.has_value())
    std::cout << v2->a << ", " << v2->b << '\n';
\end{cpp}

\item
要修改存储的值,可以使用成员函数 \verb|emplace()|、\verb|reset()| 或 \verb|swap()|:

\begin{cpp}
std::optional<int> v{ 42 }; // v contains 42
v.reset();                  // v is empty
\end{cpp}
\end{itemize}

使用 \verb|std::optional| 来建模以下的情况:

\begin{itemize}
\item
从可能无法生成值的函数返回值:

\begin{cpp}
template <typename K, typename V>
std::optional<V> find(K const key,
std::map<K, V> const & m)
{
    auto pos = m.find(key);
    if (pos != m.end())
        return pos->second;
    return {};
}
std::map<int, std::string> m{
    { 1, "one"s },{ 2, "two"s },{ 3, "three"s } };
auto value = find(2, m);
if (value) std::cout << *value << '\n'; // two
value = find(4, m);
if (value) std::cout << *value << '\n';
\end{cpp}

\item
函数的可选参数:

\begin{cpp}
std::string extract(std::string const & text,
                    std::optional<int> start,
                    std::optional<int> end)
{
    auto s = start.value_or(0);
    auto e = end.value_or(text.length());
    return text.substr(s, e - s);
}
auto v1 = extract("sample"s, {}, {});
std::cout << v1 << '\n'; // sample
auto v2 = extract("sample"s, 1, {});
std::cout << v2 << '\n'; // ample
auto v3 = extract("sample"s, 1, 4);
std::cout << v3 << '\n'; // amp
\end{cpp}

\item
类的可选数据成员:

\begin{cpp}
struct book
{
    std::string                title;
    std::optional<std::string> subtitle;
    std::vector<std::string>   authors;
    std::string                publisher;
    std::string                isbn;
    std::optional<int>         pages;
    std::optional<int>         year;
};
\end{cpp}
\end{itemize}

\mySubsubsection{}{How it works...}

类模板 \verb|std::optional| 是一个表示可选值容器的类模板。如果容器确实包含一个值,该值作为 \verb|std::optional| 对象的一部分存储;不涉及堆分配和指针。类型模板 \verb|std::optional| 概念实现如下所示:

\begin{cpp}
template <typename T>
class optional
{
    bool _initialized;
    std::aligned_storage_t<sizeof(t), alignof(T)> _storage;
};
\end{cpp}

\verb|std::aligned_storage_t| 别名模板允许创建未初始化的内存块,这些内存块可以容纳给定类型的对象。如果 \verb|std::optional| 类型模板默认构造,或者是从另一个空的 \verb|std::optional| 实例或 \verb|std::nullopt_t| 值复制构造或复制赋值,则不包含值。这样的值是 \verb|std::nullopt|,这是一个 constexpr 值,用于指示处于未初始化状态的可选对象。是一个辅助类型,实现为一个空类,表示处于未初始化状态的可选对象。

可选类型(其他编程语言中称为“可空类型”)的用途是从可能失败的函数返回值。这种情况下的可能解决方案包括:

\begin{itemize}
\item
返回一个 \verb|std::pair<T, bool>|,其中 T 是返回值的类型;对的第二个元素是一个布尔标志,指示第一个元素的值是否有效。

\item
返回一个布尔值,多带一个类型为 \verb|T&| 的参数,并仅在函数成功时为此参数赋值。

\item
返回一个原始指针或智能指针类型,并使用 \verb|nullptr| 表示失败。
\end{itemize}

\verb|std::optional| 类型模板是一个更好的方法。一方面,不需要涉及函数的输出参数(在 C 和 C++ 之外,这不是返回值的标准形式),也不需要处理指针,另一方面,更好地封装了 \verb|std::pair<T, bool>| 的细节。

然而,可选对象也可以用作类的数据成员,并且编译器能够优化内存布局以实现高效存储。

\begin{myNotic}
类型模板 \verb|std::optional| 不能用于返回多态类型。例如,编写了一个工厂方法,需要从类型层次结构中返回不同的类型,则不能依赖 \verb|std::optional|,而需要返回一个指针,最好是 \verb|std::unique_ptr| 或 \verb|std::shared_ptr|(取决于是否需要共享对象的所有权)。
\end{myNotic}

使用 \verb|std::optional| 向函数传递可选参数时,可能会创建副本,这对于大型对象来说可能是一个性能问题。看看以下函数,该函数有一个对 \verb|std::optional| 参数的常量引用:

\begin{cpp}
struct bar { /* details */ };
void process(std::optional<bar> const & arg)
{
    /* do something with arg */
}
std::optional<bar> b1{ bar{} };
bar b2{};
process(b1); // no copy
process(b2); // copy construction
\end{cpp}

第一次调用 \verb|process()| 不涉及任何实例的构造,因为传递了一个 \verb|std::optional<bar>| 类型实例。然而,第二次调用将涉及 \verb|bar| 类型实例的复制构造,b2 是一个 \verb|bar| 并且需要复制到一个 \verb|std::optional<bar>|中;即使 \verb|bar| 实现了移动语义,也会创建副本。如果 \verb|bar| 是一个小实例,这不应该成为大问题,但对于大型实例,这可能成为一个性能问题。避免这个问题的方法取决于上下文,包括创建一个接受 \verb|bar| 常量引用的第二个重载,或者完全避免使用 \verb|std::optional|。

\mySubsubsection{}{there's more}

虽然 \verb|std::optional| 让从可能失败的函数返回值变得更容易,但将多个这样的函数链接起来会产生冗余的代码。为了简化这种情况,C++23 中\verb|std::optional| 添加了几个新的成员(\verb|transform()|、\verb|and_then()| 和 \verb|or_else()|),将在下一个示例中介绍这些操作。

