程序退出时,通常需要执行清理代码来释放资源、写入日志或进行其他收尾操作。标准库提供了两个工具函数,允许在程序正常终止时(无论是从 main() 返回还是通过调用 std::exit() 或 \verb|std::quick_exit()|)注册要调用的函数。这对于那些需要在程序终止前执行某个动作的库来说特别有用,而无需依赖用户显式地调用结束函数。本示例中,将介绍如何使用退出时处理器。

\mySubsubsection{}{Getting ready}

本示例中讨论的所有函数,包括 \verb|exit()|、\verb|quick_exit()|、\verb|atexit()| 和 \verb|at_quick_exit()|,都在头文件 <cstdlib> 中的命名空间 std 内可用。

\mySubsubsection{}{How to do it...}

为了注册在程序终止时要调用的函数,应该使用以下方法:

\begin{itemize}
\item
使用 std::atexit() 注册当 main() 返回或调用 std::exit() 时要调用的函数:

\begin{cpp}
void exit_handler_1()
{
    std::cout << "exit handler 1" << '\n';
}
void exit_handler_2()
{
    std::cout << "exit handler 2" << '\n';
}
std::atexit(exit_handler_1);
std::atexit(exit_handler_2);
std::atexit([]() {std::cout << "exit handler 3" << '\n'; });
\end{cpp}

\item
使用 \verb|std::at_quick_exit()| 注册当调用 \verb|std::quick_exit()| 时要调用的函数:

\begin{cpp}
void quick_exit_handler_1()
{
    std::cout << "quick exit handler 1" << '\n';
}
void quick_exit_handler_2()
{
    std::cout << "quick exit handler 2" << '\n';
}
std::at_quick_exit(quick_exit_handler_1);
std::at_quick_exit(quick_exit_handler_2);
std::at_quick_exit([]() {
    std::cout << "quick exit handler 3" << '\n'; });
\end{cpp}
\end{itemize}

\mySubsubsection{}{How it works...}

无论退出处理程序是通过哪种方法注册的,只在程序正常或快速终止时调用。如果程序是以异常方式终止,比如调用 std::terminate() 或 std::abort(),则不会调用处理程序。如果这些处理程序中的任何一个由于异常而退出,则 std::terminate() 将调用。退出处理程序不能有参数,并且必须返回 void 类型。注册后,不能取消注册。

一个程序可以安装多个处理程序。标准保证每个方法至少可以注册 32 个处理程序,尽管实际的实现可能支持更多。\verb|std::atexit()| 和 \verb|std::at_quick_exit()| 都是线程安全的,可以从不同的线程同时调用。

如果注册了多个处理程序,其将以与注册顺序相反的顺序调用。下表展示了当程序通过 \verb|std::exit()| 调用或 \verb|std::quick_exit()| 调用终止时,如前一节所示注册的退出处理程序的输出:

\begin{itemize}
\item
std::exit(0);

\begin{shell}
exit handler 3
exit handler 2
exit handler 1
\end{shell}

\item
\verb|std::quick_exit(0);|

\begin{shell}
quick exit handler 3
quick exit handler 2
quick exit handler 1
\end{shell}
\end{itemize}

\begin{center}
表6.7:调用\verb|exit()|和\verb|quick_exit()|而退出前的输出
\end{center}

另一方面,程序正常终止时,具有局部存储期的对象销毁、具有静态存储期的对象销毁,以及对已注册退出处理程序并发调用。

但是,保证在静态对象构造之前注册的退出处理程序,将在该静态对象销毁之后调用。而在静态对象构造之后注册的退出处理程序,将在该静态对象销毁之前调用。

为了更好地说明这一点,来看下面这个例子:

\begin{cpp}
struct static_foo
{
    ~static_foo() { std::cout << "static foo destroyed!" << '\n'; }
    static static_foo* instance()
    {
        static static_foo obj;
        return &obj;
    }
};
\end{cpp}

上下文中,引用以下代码:

\begin{cpp}
std::atexit(exit_handler_1);
static_foo::instance();
std::atexit(exit_handler_2);
std::atexit([]() {std::cout << "exit handler 3" << '\n'; });
std::exit(42);
\end{cpp}

当上述代码执行时,\verb|exit_handler_1| 在创建静态对象 \verb|static_foo| 之前注册。另一方面,\verb|exit_handler_2| 和 Lambda 表达式都按此顺序在静态对象构建后注册,所以正常终止时的调用顺序如下:

\begin{enumerate}
\item
Lambda 表达式

\item
\verb|exit_handler_2|

\item
\verb|static_foo| 的析构函数

\item
\verb|exit_handler_1|
\end{enumerate}

上述程序的输出如下:

\begin{shell}
exit handler 3
exit handler 2
static foo destroyed!
exit handler 1
\end{shell}

使用 \verb|std::at_quick_exit()| 时,如果程序正常终止,注册的函数不会被调用。如果在这种情况下需要调用某个函数,则必须使用 \verb|std::atexit()| 进行注册。

