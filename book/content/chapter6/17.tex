
前几个示例中,了解了类型支持库中的一些特性,特别是类型特征。另一个值得深入讨论的类型转换特性是 std::conditional,能够根据编译时的布尔表达式在两种类型之间进行选择。本示例中,将通过几个例子了解其工作原理以及如何使用。

\mySubsubsection{}{How to do it...}

以下是几个示例,展示了如何使用 \verb|<type_traits>| 头文件中提供的 std::conditional(和 \verb|std::conditional_t|)在编译时选择两种类型:

\begin{itemize}
\item
类型别名或 typedef 中,根据平台(32位平台上指针大小为4字节,64位平台上为8字节)选择32位或64位整数类型:

\begin{cpp}
using long_type = std::conditional_t<
    sizeof(void*) <= 4, long, long long>;
auto n = long_type{ 42 };
\end{cpp}

\item
别名模板中,根据用户指定(作为一个非类型模板参数)选择8位、16位、32位或64位整数类型:

\begin{cpp}
template <int size>
using number_type =
    typename std::conditional_t<
        size<=1,
        std::int8_t,
        typename std::conditional_t<
            size<=2,
            std::int16_t,
            typename std::conditional_t<
                size<=4,
                std::int32_t,
                std::int64_t
            >
        >
    >;
auto n = number_type<2>{ 42 };
static_assert(sizeof(number_type<1>) == 1);
static_assert(sizeof(number_type<2>) == 2);
static_assert(sizeof(number_type<3>) == 4);
static_assert(sizeof(number_type<4>) == 4);
static_assert(sizeof(number_type<5>) == 8);
static_assert(sizeof(number_type<6>) == 8);
static_assert(sizeof(number_type<7>) == 8);
static_assert(sizeof(number_type<8>) == 8);
static_assert(sizeof(number_type<9>) == 8);
\end{cpp}

\item
类型模板参数中,根据类型模板参数是否为整型或浮点型来选择整数或实数均匀分布:

\begin{cpp}
template <typename T,
          typename D = std::conditional_t<
                       std::is_integral_v<T>,
                       std::uniform_int_distribution<T>,
                       std::uniform_real_distribution<T>>,
          typename = typename std::enable_if_t<
                        std::is_arithmetic_v<T>>>
std::vector<T> GenerateRandom(T const min, T const max,
                              size_t const size)
{
    std::vector<T> v(size);
    std::random_device rd{};
    std::mt19937 mt{ rd() };
    D dist{ min, max };
    std::generate(std::begin(v), std::end(v),
        [&dist, &mt] {return dist(mt); });
    return v;
}
auto v1 = GenerateRandom(1, 10, 10);     // integers
auto v2 = GenerateRandom(1.0, 10.0, 10); // doubles
\end{cpp}
\end{itemize}

\mySubsubsection{}{How it works...}

std::conditional 是一个类模板,定义了一个名为 type 的成员,该成员可以是其两个类型模板参数。这种选择是基于,作为非类型模板参数提供的编译时常量布尔表达式:

\begin{cpp}
template<bool Test, class T1, class T2>
struct conditional
{
    typedef T2 type;
};
template<class T1, class T2>
struct conditional<true, T1, T2>
{
    typedef T1 type;
};
\end{cpp}

总结一下上一节的例子:

\begin{itemize}
\item
第一个例子中,如果平台是32位的,那么指针类型的大小是4字节,编译时表达式 \verb|sizeof(void*) <= 4| 为真;结果,std::conditional 将其成员类型定义为 long。如果平台是64位的,因为指针类型的大小是8字节,所以条件评估为假,成员类型定义为 long long。

\item
第二个例子中遇到了类似的情况,多次使用 std::conditional 来模拟一系列 if...else 语句以选择适当类型。

\item
第三个例子中,使用别名模板 \verb|std::conditional_t| 来简化函数模板 GenerateRandom 的声明。这里,std::conditional 用于定义表示统计分布的类型模板参数的默认值。根据第一个类型模板参数 T 是否为整型或浮点型,默认分布类型在 \verb|std::uniform_int_distribution<T>| 和 \verb|std::uniform_real_distribution<T>| 之间选择。使用第三个模板参数 \verb|std::enable_if| 禁用了其他类型。
\end{itemize}

为了简化 std::conditional 的使用,C++14 提供了一个别名模板 \verb|std::conditional_t|,其定义为:

\begin{cpp}
template<bool Test, class T1, class T2>
using conditional_t = typename conditional_t<Test,T1,T2>;
\end{cpp}

可以选择使用这个辅助类(以及许多类似的来自标准库的类),有助于编写更加简洁的代码。

