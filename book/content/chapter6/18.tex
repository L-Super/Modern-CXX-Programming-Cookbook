
调试是软件开发中不可或缺的一部分。无论是多么简单或复杂的程序,都很难从第一次尝试起就按预期运行。开发者会花费大量时间调试代码,使用各种工具和技术,从调试器到打印到控制台或文本文件的消息。有时候,希望在日志消息中提供详细的信息来源,包括文件名、行号,甚至可能是函数名。虽然这在过去可以通过一些标准宏实现, C++20 中一个新的实用类型 \verb|std::source_location| 能够以现代方式做到这一点。

\mySubsubsection{}{How to do it...}

要记录包括文件名、行号和函数名在内的信息,请执行以下步骤:

\begin{itemize}
\item
定义一个带有所有需要提供的信息参数(如消息、严重性等)的日志记录函数。

\item
添加一个 \verb|std::source_location| 类型参数(需要包含 \verb|<source_location>| 头文件),并将其默认值设为 \verb|std::source_location::current()|。

\item
使用成员函数 \verb|file_name()|、\verb|line()|、\verb|column()| 和 \verb|function_name()| 获取调用源的信息。
\end{itemize}

下面展示了一个这样的日志记录函数的例子:

\begin{cpp}
void log(std::string_view message,
         std::source_location const location = std::source_location::current())
{
    std::cout   << location.file_name() << '('
                << location.line() << ':'
                << location.column() << ") '"
                << location.function_name() << "': "
                << message << '\n';
}
\end{cpp}

\mySubsubsection{}{How it works...}

C++20 之前,记录诸如源文件、行号和函数名之类的信息只能借助多个宏来实现:

\begin{itemize}
\item
\verb|__FILE__|, 展开为当前文件的名称

\item
\verb|__LINE__|, 展开为源文件的行号
\end{itemize}

此外,所有编译器都支持非标准宏,包括 \verb|__func__| / \verb|__FUNCTION__|,这些宏提供当前函数的名称。

使用这些宏,可以编写如下日志记录函数:

\begin{cpp}
void log(std::string_view message,
         std::string_view file,
         int line,
         std::string_view function)
{
    std::cout << file << '('
              << line << ") '"
              << function << "': "
              << message << '\n';
}
\end{cpp}

然而,这些宏必须从函数执行的上下文中使用:

\begin{cpp}
int main()
{
    log("This is a log entry!", __FILE__, __LINE__, __FUNCTION__);
}
\end{cpp}

运行此函数的结果将在控制台上显示如下:

\begin{shell}
[...]\source.cpp(23) 'main': This is a log entry!
\end{shell}

C++20 的 \verb|std::source_location| 是一个更好的替代方案:

\begin{itemize}
\item
不再依赖于宏。

\item
包括列信息,而不仅仅是行信息。

\item
可以在日志记录函数的签名中使用,简化调用。
\end{itemize}

“How to do it...”定义的 log() 函数有如下调用:

\begin{cpp}
int main()
{
    log("This is a log entry!");
}
\end{cpp}

将生成以下输出:

\begin{shell}
[...]\source.cpp(23:4) 'int __cdecl main(void)': This is a log entry!
\end{shell}

尽管存在默认构造函数,但用默认值初始化数据。要获取正确的值,必须调用静态成员函数 current()。此函数的工作方式如下:

\begin{itemize}
\item
直接在一个函数调用中调用时,用关于调用位置的信息初始化数据。

\item
在默认成员初始化器中使用时,用关于构造函数聚合初始化数据成员的位置的信息初始化数据。

\item
用作默认参数(如本例所示)时,用调用位置(函数调用)的信息初始化数据。

\item
其他上下文中使用时,行为未定义。
\end{itemize}

需要注意的是,预处理器指令 \verb|#line| 会更改源代码行号和文件名。这会影响宏 \verb|__FILE__| 和 \verb|__LINE__| 返回的值。同样地,\verb|std::source_location| 也会受到 \verb|#line| 指令的影响,与标准宏相同。

