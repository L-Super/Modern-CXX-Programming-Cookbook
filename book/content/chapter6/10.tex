
std::variant 是基于 boost.variant 库添加到 C++17 的新标准容器。variant 是种类型安全的联合,可以持有其替代类型之一的值。前一个示例中,已经看到了与 variant 相关的各种操作,但所使用的 variant 主要是简单的 POD 类型,这并不是 std::variant 创建的实际目的。variant设计用来持有相似的非多态性和非 POD 类型的替代值。本示例中,将介绍一个更贴近现实的例子,了解如何使用 variant 和访问 variant。

\mySubsubsection{}{Getting ready}

为了完成这个示例,应该熟悉 std::variant 类型。

为了说明如何进行 variant 访问,将考虑一个用于表示媒体 DVD 的 variant。假设要建模一个商店或图书馆,其中的 DVD 可能包含音乐、电影或软件。但这些选项并没有形成具有共同数据和虚拟函数的层次结构,而是作为可能具有类似属性(标题)的非关联类型。为了简化起见,考虑以下属性:

\begin{itemize}
\item
对于电影:标题和长度(以分钟为单位)

\item
对于专辑:标题、艺术家姓名和曲目列表(每个曲目都有标题和长度,以秒为单位)

\item
对于软件:标题和制造商
\end{itemize}

下面的代码展示了这些类型的简单实现,没有包含任何函数,这与访问持有这些类型替代值的 variant 无关:

\begin{cpp}
enum class Genre { Drama, Action, SF, Comedy };
struct Movie
{
    std::string title;
    std::chrono::minutes length;
    std::vector<Genre> genre;
};
struct Track
{
    std::string title;
    std::chrono::seconds length;
};
struct Music
{
    std::string title;
    std::string artist;
    std::vector<Track> tracks;
};
struct Software
{
    std::string title;
    std::string vendor;
};
using dvd = std::variant<Movie, Music, Software>;
std::vector<dvd> dvds
{
    Movie{ "The Matrix"s, 2h + 16min,{ Genre::Action, Genre::SF } },
    Music{ "The Wall"s, "Pink Floyd"s,
        { { "Mother"s, 5min + 32s },
            { "Another Brick in the Wall"s, 9min + 8s } } },
    Software{ "Windows"s, "Microsoft"s },
};
\end{cpp}

另一方面,利用以下函数将文本转换为大写:

\begin{cpp}
template <typename CharT>
using tstring = std::basic_string<CharT, std::char_traits<CharT>,
                                         std::allocator<CharT>>;
template<typename CharT>
inline tstring<CharT> to_upper(tstring<CharT> text)
{
    std::transform(std::begin(text), std::end(text),
                   std::begin(text), toupper);
    return text;
}
\end{cpp}

有了这些定义后,来看看如何对 variant 进行访问。

\mySubsubsection{}{How to do it...}

要访问variant,必须为 variant 的替代值提供一个或多个动作。有几种类型的访问器,其用于不同的目的:

\begin{itemize}
\item
不返回任何值但有副作用的 void 访问器。下面的示例将每个 DVD 的标题打印到控制台:

\begin{cpp}
for (auto const & d : dvds)
{
    std::visit([](auto&& arg) {
                std::cout << arg.title << '\n'; },
            d);
}
\end{cpp}

\item
返回一个值的访问器;无论 variant 的当前替代值是什么,返回的值都应该具有相同的类型,或者本身也可以是一个 variant。下面的例子中,访问一个 variant,并返回一个新的相同类型的 variant,该 variant 将从其替代值中提取标题属性,并转换为大写字母:

\begin{cpp}
for (auto const & d : dvds)
{
    dvd result = std::visit(
        [](auto&& arg) -> dvd
        {
            auto cpy { arg };
            cpy.title = to_upper(cpy.title);
            return cpy;
        },
    d);
    std::visit(
        [](auto&& arg) {
            std::cout << arg.title << '\n'; },
        result);
}
\end{cpp}

\item
实现了类型匹配的访问器(可以是不返回值的 void 访问器,也可以是返回值的访问器),通过为 variant 的每个替代类型提供重载调用操作符的函数实例来实现:

\begin{cpp}
struct visitor_functor
{
    void operator()(Movie const & arg) const
    {
        std::cout << "Movie" << '\n';
        std::cout << " Title: " << arg.title << '\n';
        std::cout << " Length: " << arg.length.count()
                  << "min" << '\n';
    }
    void operator()(Music const & arg) const
    {
        std::cout << "Music" << '\n';
        std::cout << " Title: " << arg.title << '\n';
        std::cout << " Artist: " << arg.artist << '\n';
        for (auto const & t : arg.tracks)
        std::cout << " Track: " << t.title
                  << ", " << t.length.count()
                  << "sec" << '\n';
    }
    void operator()(Software const & arg) const
    {
        std::cout << "Software" << '\n';
        std::cout << " Title: " << arg.title << '\n';
        std::cout << " Vendor: " << arg.vendor << '\n';
    }
};
for (auto const & d : dvds)
{
    std::visit(visitor_functor(), d);
}
\end{cpp}

\item
实现了类型匹配的访问器,提供一个基于替代值类型执行操作的 lambda 表达式来实现:

\begin{cpp}
for (auto const & d : dvds)
{
    std::visit([](auto&& arg) {
        using T = std::decay_t<decltype(arg)>;
        if constexpr (std::is_same_v<T, Movie>)
        {
            std::cout << "Movie" << '\n';
            std::cout << " Title: " << arg.title << '\n';
            std::cout << " Length: " << arg.length.count()
                      << "min" << '\n';
        }
        else if constexpr (std::is_same_v<T, Music>)
        {
            std::cout << "Music" << '\n';
            std::cout << " Title: " << arg.title << '\n';
            std::cout << " Artist: " << arg.artist << '\n';
            for (auto const & t : arg.tracks)
            std::cout << " Track: " << t.title
                      << ", " << t.length.count()
                      << "sec" << '\n';
        }
        else if constexpr (std::is_same_v<T, Software>)
        {
            std::cout << "Software" << '\n';
            std::cout << " Title: " << arg.title << '\n';
            std::cout << " Vendor: " << arg.vendor << '\n';
        }
    },
    d);
}
\end{cpp}
\end{itemize}

\mySubsubsection{}{How it works...}

访问器是一个可调用的实例(一个函数、一个 lambda 表达式或一个函数对象),接受来自 variant 的每一个可能的替代值。访问是通过调用带有访问器和一个或多个 variant 实例的 std::visit() 完成。这些 variant 不必是同一种类型,但是访问器必须能够接受所调用 variant 的每一个替代值。前面的例子中,访问了一个单一的 variant 对象,但是访问多个 variant 也可以将它们作为参数传递给 std::visit()。

访问variant 时,可调用示例会作为当前存储在 variant 中的值调用。如果访问器不接受 variant 中存储的类型的参数,程序将无法运行。如果访问器是一个函数实例,则必须为 variant 的所有替代类型重载其调用操作符。如果访问器是一个 lambda 表达式,应该是一个泛型 lambda,这基本上是一个带有调用操作符模板的函数对象,该模板由编译器根据实际调用的类型实例化。

上一节中,两种方法的例子都展示了一个类型匹配访问器。第一个例子中的函数对象非常直观,不需要解释。另一方面,泛型 lambda 表达式使用 constexpr if 根据参数类型在编译时选择特定的 if 分支。结果是编译器会创建一个带有调用操作符模板的函数实例,其主体包含 constexpr if 语句;实例化该函数模板时,将为 variant 的每一个可能的替代类型生成一个重载,并且在这些重载中的每一个中,只会选择与调用操作符参数类型匹配的 constexpr if 分支。结果在概念上等同于 \verb|visitor_functor| 类型的实现。

