
模板元编程是C++语言的一个强大特性,能够编写并重用适用于所有类型的泛型代码。而实践中,往往需要泛型代码根据不同的类型以不同的方式工作,或者根本无法工作,这可能是出于设计意图、语义正确性、性能或其他原因。例如,一个泛型算法对于POD(Plain Old Data)类型和非POD类型有不同的实现,或者一个函数模板只能用整数类型实例化。C++11提供了一组类型特征(type traits)来帮助实现这一目标。

类型特征是提供关于其他类型信息的元类型。类型特征库包含了一个长长的列表,用于查询类型属性(如检查一个类型是否为整数类型或两个类型是否相同),也用于执行类型转换(如移除const和volatile限定符或将指针添加到类型)。这本书的前面章节中,已经在多个示例中使用了类型特征;然而,本示例中,将介绍类型特征是什么,以及其工作原理。

\mySubsubsection{}{Getting ready}

C++11中引入的所有类型特征都可以在 \verb|<type_traits>| 头文件的 std 命名空间中找到。

类型特征可以在许多元编程场景中使用。在本示例中,将总结一些这样的使用案例,并理解类型特征的工作原理。

本示例中,将讨论全特化和偏特化。熟悉这些概念将有助于更好地理解类型特征的工作方式。

\mySubsubsection{}{How to do it...}

以下列表显示了使用类型特征实现不同设计目标:

\begin{itemize}
\item
\verb|enable_if| 定义函数模板可以实例化的类型的预条件:

\begin{cpp}
template <typename T,
          typename = typename std::enable_if_t<
                std::is_arithmetic_v<T> > >
T multiply(T const t1, T const t2)
{
    return t1 * t2;
}
auto v1 = multiply(42.0, 1.5);     // OK
auto v2 = multiply("42"s, "1.5"s); // error
\end{cpp}

\item
\verb|static_assert| 确保满足不变量的需求:

\begin{cpp}
template <typename T>
struct pod_wrapper
{
    static_assert(std::is_standard_layout_v<T> &&
                  std::is_trivial_v<T>,
                  "Type is not a POD!");
    T value;
};
pod_wrapper<int> i{ 42 };            // OK
pod_wrapper<std::string> s{ "42"s }; // error
\end{cpp}

\item
\verb|std::conditional| 在类型之间进行选择:

\begin{cpp}
template <typename T>
struct const_wrapper
{
    typedef typename std::conditional_t<
            std::is_const_v<T>,
            T,
            typename std::add_const_t<T>> const_type;
};
static_assert(
    std::is_const_v<const_wrapper<int>::const_type>);
static_assert(
    std::is_const_v<const_wrapper<int const>::const_type>);
\end{cpp}

\item
\verb|constexpr if| 让编译器基于模板实例化的类型生成不同的代码:

\begin{cpp}
template <typename T>
auto process(T arg)
{
    if constexpr (std::is_same_v<T, bool>)
        return !arg;
    else if constexpr (std::is_integral_v<T>)
        return -arg;
    else if constexpr (std::is_floating_point_v<T>)
        return std::abs(arg);
    else
        return arg;
}
auto v1 = process(false); // v1 = true
auto v2 = process(42);    // v2 = -42
auto v3 = process(-42.0); // v3 = 42.0
auto v4 = process("42"s); // v4 = "42"
\end{cpp}
\end{itemize}

\mySubsubsection{}{How it works...}

类型特征是提供类型或其属性的信息类型,也可以用来修改类型。实际上存在两类类型特征:

\begin{itemize}
\item
\item 提供关于类型、其属性或关系信息的特征(如 \verb|is_integer|, \verb|is_arithmetic|, \verb|is_array|, \verb|is_enum|, \verb|is_class|, \verb|is_const|, \verb|is_trivial|, \verb|is_standard_layout|, \verb|is_constructible|, \verb|is_same| 等)。这些特征提供了名为 value 的常量 bool 成员。

\item
修改类型属性的特征(如 \verb|add_const|, \verb|remove_const|, \verb|add_pointer|, \verb|remove_pointer|, \verb|make_signed|, \verb|make_unsigned| 等)。这些特征提供了一个表示转换后类型的成员 typedef,称为 type。
\end{itemize}

这两种类型的特征都在“How to do it...”有所展示,这里提供了一个简短的概述:

\begin{itemize}
\item
第一个例子中,函数模板 multiply() 只允许用算术类型(即整数或浮点数)实例化;当用不同类型的类型实例化时,\verb|enable_if| 不定义一个名为 type 的 typedef 成员,从而产生编译错误。

\item
第二个例子中,\verb|pod_wrapper| 是一个类模板,只能用 POD 类型实例化。如果使用非 POD 类型,\verb|static_assert| 声明会产生编译错误(要么不是简单类型,要么不符合标准布局)。

\item
第三个例子中,\verb|const_wrapper| 是一个类模板,提供了一个表示常量限定类型的 typedef 成员,称为 \verb|const_type|。

\item
这个例子中,使用 std::conditional 在编译时在两种类型之间进行选择:如果类型参数 T 已经是一个常量类型,就直接选择 T。否则,使用 \verb|add_const| 类型特征来用 const 限定符修饰类型。

\item
第四个例子中,process() 是一个函数模板,其中包含一系列 \verb|if constexpr| 分支。根据使用多种类型特征(\verb|is_same|, \verb|is_integer|, \verb|is_floating_point|)在编译时查询的类型类别,编译器会选择一个分支放入生成的代码中,并丢弃其余分支。像 process(42) 这样的调用,将产生以下函数模板实例:

\begin{cpp}
int process(int arg)
{
    return -arg;
}
\end{cpp}
\end{itemize}

类型特征是通过提供一个类模板,及其部分或完全特化来实现的。以下是某些类型特征的概念实现:

\begin{itemize}
\item
\verb|is_void()| 方法指示一个类型是否为 void;这使用了全特化:

\begin{cpp}
template <typename T>
struct is_void
{ static const bool value = false; };
template <>
struct is_void<void>
{ static const bool value = true; };
\end{cpp}

\item
\verb|is_pointer()| 方法指示一个类型是否是指向对象或函数的指针;这使用了偏特化:

\begin{cpp}
template <typename T>
struct is_pointer
{ static const bool value = false; };
template <typename T>
struct is_pointer<T*>
{ static const bool value = true; };
\end{cpp}

\item
\verb|enable_if()| 类型特征仅当其非类型模板参数是一个评估结果为真的表达式时,才为其类型模板参数定义一个类型别名:

\begin{cpp}
template<bool B, typename T = void>
struct enable_if {};
template<typename T>
struct enable_if<true, T> { using type = T; };
\end{cpp}
\end{itemize}

由于使用 bool 成员 value 查询属性特征(如 \verb|std::is_integer<int>::value|)或使用成员类型别名 type 修改属性特征(如 \verb|std::enable_if<true, T>::type|)过于冗长,C++14 和 C++17 标准引入了一些简化功能:

\begin{itemize}
\item
形式为 \verb|std::trait_v<T>| 的变量模板,用于 \verb|std::trait<T>::value|。例如,\verb|std::is_integer_v<T>| 的定义如下:

\begin{cpp}
template <typename T>
inline constexpr bool is_integral_v = is_integral<T>::value;
\end{cpp}

\item
形式为 \verb|std::trait_t<T>| 的别名模板,用于 \verb|std::trait<T>::type|。例如,\verb|std::enable_if_t<B, T>| 的定义如下:

\begin{cpp}
template <bool B, typename T = void>
using enable_if_t = typename enable_if<B,T>::type;
\end{cpp}
\end{itemize}

C++20 起,POD 类型的概念已废弃。这也包括废弃 \verb|std::is_pod| 类型特征。POD 类型既是简单类型(具有编译器提供的特殊成员或显式默认的特殊成员,并占据连续的内存区域),又符合标准布局(不包含与 C 语言不兼容的语言特性,如虚函数,且所有成员具有相同的访问控制)。从 C++20 开始,更细粒度的简单类型和标准布局类型的概念更为优先,所以不应再使用 \verb|std::is_pod|,而要使用 \verb|std::is_trivial| 和 \verb|std::is_standard_layout|。

\mySubsubsection{}{there's more}

类型特征并不局限于标准库所提供的内容。使用类似的技术,可以定义自己的类型特征以实现各种目标。

