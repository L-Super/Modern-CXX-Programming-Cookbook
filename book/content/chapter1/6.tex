
枚举(enumeration)是C++中的一种基本类型,用于定义一组值,这些值始终属于某个整型基础类型。枚举的命名值(称为枚举器,enumerators)是常量,使用关键字enum声明的枚举称为无作用域枚举(unscoped enumerations),而使用enum class或enum struct声明的枚举则称为有作用域枚举(scoped enumerations)。后者是在C++11中引入的,旨在解决无作用域枚举中存在的若干问题。

\mySubsubsection{}{How to do it...}

使用枚举时应该:

\begin{itemize}
\item
尽可能使用有作用域枚举,而非无作用域枚举。

\item
使用enum class或enum struct声明有作用域枚举:

\begin{cpp}
enum class Status { Unknown, Created, Connected };
Status s = Status::Created;
\end{cpp}
\end{itemize}

\begin{myNotic}
enum class和enum struct声明等价,在本书其余部分中,将使用enum class。
\end{myNotic}

有作用域枚举是有作用域限定的命名空间,C++20标准可以通过using指令与它们关联。可以这样做:

\begin{itemize}
\item
使用using指令在局部作用域中引入有作用域枚举标识符:

\begin{cpp}
int main()
{
    using Status::Unknown;
    Status s = Unknown;
}
\end{cpp}

\item
使用using指令在局部作用域中引入有作用域枚举的所有标识符:

\begin{cpp}
struct foo
{
    enum class Status { Unknown, Created, Connected };
    using enum Status;
};
foo::Status s = foo::Created; // instead of
// foo::Status::Created
\end{cpp}

\item
switch语句中使用using enum指令引入枚举标识符以简化代码:

\begin{cpp}
void process(Status const s)
{
    switch (s)
    {
        using enum Status;
        case Unknown:   /*…*/ break;
        case Created:   /*...*/ break;
        case Connected: /*...*/ break;
    }
}
\end{cpp}
\end{itemize}

将有作用域枚举转换为其基础类型有时是必要的,特别是在使用旧风格API时,这些API通常接受整型参数。C++23,可以使用std::to\_underlying()工具函数将有作用域枚举转换为其基础类型:

\begin{cpp}
void old_api(unsigned flag);
enum class user_rights : unsigned
{
    None, Read = 1, Write = 2, Delete = 4
};
old_api(std::to_underlying(user_rights::Read));
\end{cpp}

\mySubsubsection{}{How it works...}

无作用域枚举存在一些问题,这些问题对开发者造成了困扰:

\begin{itemize}
\item
将枚举器导出到周围的上下文中,这有两个缺点:

\begin{itemize}
\item
如果两个枚举在同一个命名空间中有相同名称的枚举器,可能导致名称冲突。

\item
无法使用完全限定名来使用枚举器:

\begin{cpp}
enum Status {Unknown, Created, Connected};
enum Codes {OK, Failure, Unknown};   // error
auto status = Status::Created;       // error
\end{cpp}
\end{itemize}

\item
C++11之前,不能指定基础类型,而这个类型必须是整型,且不能大于int,除非枚举器值不能适应有符号或无符号整型,所以枚举不可能前向声明。原因在于枚举的大小未知,直到定义枚举器值后,编译器才能选择合适的基础整型。这个问题已在C++11中得到解决。

\item
枚举器值可以隐式转换为int。可以有意或无意地混合具有特定意义的枚举值和整数(这些整数可能与枚举的意义无关),而编译器不会发出警告:

\begin{cpp}
enum Codes { OK, Failure };
void include_offset(int pixels) {/*...*/}
include_offset(Failure);
\end{cpp}
\end{itemize}

相比之下,有作用域枚举基本上是强类型的枚举,行为与无作用域枚举不同:

\begin{itemize}
\item
不会将其枚举器导出到周围的作用域。上面提到的两个枚举将变为如下形式,不再产生名称冲突,并且可以使用完全限定名:

\begin{cpp}
enum class Status { Unknown, Created, Connected };
enum class Codes { OK, Failure, Unknown }; // OK
Codes code = Codes::Unknown;               // OK
\end{cpp}

\item
可以指定基础类型。对于基础类型的规定规则同样适用于有作用域枚举,只是用户可以显式指定基础类型。可以在定义可用之前知道基础类型,所以也解决了前向声明的问题:

\begin{cpp}
enum class Codes : unsigned int;
void print_code(Codes const code) {}
enum class Codes : unsigned int
{
    OK = 0,
    Failure = 1,
    Unknown = 0xFFFF0000U
};
\end{cpp}

\item
有作用域枚举的值不再隐式转换为int。将enum class的值赋给整型变量会触发编译器错误,除非指定了显式转换:

\begin{cpp}
Codes c1 = Codes::OK;                       // OK
int c2 = Codes::Failure;                    // error
int c3 = static_cast<int>(Codes::Failure);  // OK
\end{cpp}

\end{itemize}

然而,有作用域枚举有一个缺点:受限制的作用域。不会在外部作用域中导出标识符,这在某些情况下可能不方便,例如:正在编写一个switch语句,并且需要为每个case标签重复枚举名称:

\begin{cpp}
std::string_view to_string(Status const s)
{
    switch (s)
    {
        case Status::Unknown:   return "Unknown";
        case Status::Created:   return "Created";
        case Status::Connected: return "Connected";
    }
}
\end{cpp}

C++20,可以使用带有有作用域枚举名称的using指令来简化。上述代码可简化为:

\begin{cpp}
std::string_view to_string(Status const s)
{
    switch (s)
    {
        using enum Status;
        case Unknown:   return "Unknown";
        case Created:   return "Created";
        case Connected: return "Connected";
    }
}
\end{cpp}

using指令的效果是,所有的枚举器标识符都会引入到了局部作用域中,可以使用非限定形式来引用。也可以通过使用带有限定标识符名称的using指令,只将特定的枚举标识符引入局部作用域,比如using Status::Connected。

C++23版本的标准添加了几个处理有作用域枚举的工具函数。其中第一个是std::to\_underlying()(需要头文件<utility>),其功能是将枚举转换为其基础类型。

其目的是与不使用有作用域枚举的API(无论是遗留的还是新的)一起工作。来看一个例子:考虑以下函数old\_api(),接受一个整型参数,将其解释为控制用户权限的标志:

\begin{cpp}
void old_api(unsigned flag)
{
    if ((flag & 0x01) == 0x01) { /* can read */ }
    if ((flag & 0x02) == 0x02) { /* can write */ }
    if ((flag & 0x04) == 0x04) { /* can delete */ }
}
\end{cpp}

这个函数有如下调用:

\begin{cpp}
old_api(1); // read only
old_api(3); // read & write
\end{cpp}

系统的新定义了如下用于用户权限的有作用域枚举:

\begin{cpp}
enum class user_rights : unsigned
{
    None,
    Read = 1,
    Write = 2,
    Delete = 4
};
\end{cpp}

不能使用来自user\_rights的枚举来调用old\_api()函数,必须使用static\_cast:

\begin{cpp}
old_api(static_cast<int>(user_rights::Read)); // read only
old_api(static_cast<int>(user_rights::Read) |
        static_cast<int>(user_rights::Write)); // read & write
\end{cpp}

为了避免这些static\_cast,C++23提供了std::to\_underlying()函数:

\begin{cpp}
old_api(std::to_underlying(user_rights::Read));
old_api(std::to_underlying(user_rights::Read) |
        std::to_underlying(user_rights::Write));
\end{cpp}

C++23中引入的另一个工具是类型特征is\_scoped\_enum<T>,在<type\_traits>头文件中。包含一个名为value的成员常量,如果模板类型参数T是有作用域枚举类型,则该常量等于true,否则等于false;还有一个辅助变量模板is\_scoped\_enum\_v<T>。

这个类型特征的目的是识别枚举是否是有作用域的,以便根据枚举的类型应用不同的行为。这里有一个简单的例子:

\begin{cpp}
enum A {};
enum class B {};
int main()
{
    std::cout << std::is_scoped_enum_v<A> << '\n';
    std::cout << std::is_scoped_enum_v<B> << '\n';
}
\end{cpp}

因为A是一个无作用域的枚举,第一行将打印0;因为B是一个有作用域的枚举,第二行将打印1。










