
C++下标操作符 [],用来访问数组或类似容器的数据结构中的元素的基本方法。这个操作符可以重载,以便为自定义类提供对内部存储数据的访问能力。特别是在模拟容器的类中,重载下标操作符是一个常见的需求。本节将介绍如何重载下标操作符以及C++23带来的变化。

\mySubsubsection{}{How to do it...}

要为容器中的元素提供随机访问,可以按照以下方式重载下标操作符:

\begin{itemize}
\item
对于一维容器,无论标准版本如何,都可以重载接受一个参数的下标操作符:

\begin{cpp}
template <typename T>
struct some_buffer
{
    some_buffer(size_t const size):data(size)
    {}
    size_t size() const { return data.size(); }
    T const& operator[](size_t const index) const
    {
        if(index >= data.size())
        std::runtime_error("invalid index");
        return data[index];
    }
    T & operator[](size_t const index)
    {
        if (index >= data.size())
        std::runtime_error("invalid index");
        return data[index];
    }
private:
    std::vector<T> data;
};
\end{cpp}

\item
对于多维容器,从C++23开始,可以重载接受多个参数的下标操作符:

\begin{cpp}
template <typename T, size_t ROWS, size_t COLS>
struct matrix
{
    T& operator[](size_t const row, size_t const col)
    {
        if(row >= ROWS || col >= COLS)
        throw std::runtime_error("invalid index");
        return data[row * COLS + col];
    }
    T const & operator[](size_t const row, size_t const col) const
    {
        if (row >= ROWS || col >= COLS)
        throw std::runtime_error("invalid index");
        return data[row * COLS + col];
    }
private:
    std::array<T, ROWS* COLS> data;
};
\end{cpp}
\end{itemize}

\mySubsubsection{}{How it works...}

下标操作符通常用于访问数组中的元素,也可以作为成员函数在类中重载,特别是模拟容器(或一般意义上的集合)的类型。例如,标准容器如 std::vector, std::set 和 std::map 都提供了下标操作符的重载,可以编写如下代码:

\begin{cpp}
std::vector<int> v {1, 2, 3};
v[2] = v[1] + v[0];
\end{cpp}

上面的部分中,了解了如何重载下标操作符。通常会有两个重载版本,常量版本和非常量版本。常量版本返回对常量对象的引用,而非常量版本返回对非常量对象的引用。

C++23之前的版本中,下标操作符的主要问题是只能有一个参数,不能直接用于访问多维容器的元素,开发人员通常会使用调用操作符 operator() 来替代。例如:

\begin{cpp}
template <typename T, size_t ROWS, size_t COLS>
struct matrix
{
    T& operator()(size_t const row, size_t const col)
    {
        if(row >= ROWS || col >= COLS)
        throw std::runtime_error("invalid index");
        return data[row * COLS + col];
    }
    T const & operator()(size_t const row, size_t const col) const
    {
        if (row >= ROWS || col >= COLS)
        throw std::runtime_error("invalid index");
        return data[row * COLS + col];
    }
private:
    std::array<T, ROWS* COLS> data;
};
matrix<int, 2, 3> m;
m(0, 0) = 1;
\end{cpp}

为了解决这个问题,C++11中可以使用 [\{expr1, expr2, ...\}] 这样的语法来重载下标操作符。下面是一个修改后的矩阵类的实现,就利用了这种语法:

\begin{cpp}
template <typename T, size_t ROWS, size_t COLS>
struct matrix
{
    T& operator[](std::initializer_list<size_t> index)
    {
        size_t row = *index.begin();
        size_t col = *(index.begin() + 1);
        if (row >= ROWS || col >= COLS)
        throw std::runtime_error("invalid index");
        return data[row * COLS + col];
    }
    T const & operator[](std::initializer_list<size_t> index) const
    {
        size_t row = *index.begin();
        size_t col = *(index.begin() + 1);
        if (row >= ROWS || col >= COLS)
        throw std::runtime_error("invalid index");
        return data[row * COLS + col];
    }
    private:
    std::array<T, ROWS* COLS> data;
};
matrix<int, 2, 3> m;
m[{0, 0}] = 1;
\end{cpp}

这种语法较为笨拙,在实践中可能很少使用。因此,C++23标准允许使用多个参数来重载下标操作符。下面是一个修改后的矩阵类的示例:

\begin{cpp}
template <typename T, size_t ROWS, size_t COLS>
struct matrix
{
    T& operator[](size_t const row, size_t const col)
    {
        if(row >= ROWS || col >= COLS)
        throw std::runtime_error("invalid index");
        return data[row * COLS + col];
    }
    T const & operator[](size_t const row, size_t const col) const
    {
        if (row >= ROWS || col >= COLS)
        throw std::runtime_error("invalid index");
        return data[row * COLS + col];
    }
    private:
    std::array<T, ROWS* COLS> data;
};
matrix<int, 2, 3> m;
m[0, 0] = 1;
\end{cpp}

这样做的结果是使得调用语法与访问一维容器保持一致,std::mdspan 类就是利用这一点来提供对对象序列的多维视图访问。std::mdspan 是C++23中引入的一个新类型,表示一个非拥有性的连续序列(如数组)的视图,但将该序列重新解释为多维数组。前面提到的矩阵类实际上可以用 std::mdspan 视图代替:

\begin{cpp}
int data[2*3] = {};
auto m = std::mdspan<int, std::extents<2, 3>> (data);
m[0, 0] = 1;
\end{cpp}






