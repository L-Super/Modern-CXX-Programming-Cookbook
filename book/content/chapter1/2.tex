
C++可以通过typedef创建用作类型名替代的同义词,比如:为类型创建更短或更有意义的名字,或者是为函数指针创建名字。然而,typedef声明不能与模板一起来创建模板类型别名。例如,std::vector<T>不是一个类型(std::vector<int>才是一个类型),而是一种类型族。当类型占位符T替换为实际类型时,就可以创建这些类型。

从C++11开始,类型别名是指向另一种已经声明的类型的名称,而别名模板是指向另一种已经声明的模板的名称。这两种类型的别名可以使用using。

\mySubsubsection{}{How to do it...}

\begin{itemize}
\item
using identifier = type-id来创建类型别名:

\begin{cpp}
using byte     = unsigned char;
using byte_ptr = unsigned char *;
using array_t  = int[10];
using fn       = void(byte, double);
void func(byte b, double d) { /*...*/ }
byte b{42};
byte_ptr pb = new byte[10] {0};
array_t a{0,1,2,3,4,5,6,7,8,9};
fn* f = func;
\end{cpp}

\item
template<template-params-list> identifier = type-id创建别名模板:

\begin{cpp}
template <class T>
class custom_allocator { /* ... */ };
template <typename T>
using vec_t = std::vector<T, custom_allocator<T>>;
vec_t<int>           vi;
vec_t<std::string>   vs;
\end{cpp}

\end{itemize}

为了保持一致性和可读性:

\begin{itemize}
\item
创建别名时不混合使用typedef和using。

\item
创建函数指针类型名称时首选using。
\end{itemize}

\mySubsubsection{}{How it works...}

typedef声明引入了类型名称的同义词(就是别名),不会引入新类型(像类、结构体、联合或枚举)。通过typedef声明引入的类型名称,遵循与标识符名称相同的隐藏规则。为了引用相同类型的同义词可以重新声明(只要是对同一类型的同义词,就可以在一个翻译单元中进行多次typedef声明):

\begin{cpp}
typedef unsigned char   byte;
typedef unsigned char * byte_ptr;
typedef int array_t[10];
typedef void(*fn)(byte, double);
template<typename T>
class foo {
    typedef T value_type;
};
typedef std::vector<int> vint_t;
typedef int INTEGER;
INTEGER x = 10;
typedef int INTEGER; // redeclaration of same type
INTEGER y = 20;
\end{cpp}

类型别名声明等价于typedef声明,可以出现在块作用域、类型作用域或命名空间作用域中。根据C++11标准(第9.2.4段,文档版本N4917):

\begin{myTip}
类型定义名称也可以通过别名声明来引入。关键字using之后的标识符成为一个类型定义名,紧跟在标识符后面的可选属性说明符序列属于该类型定义名。其语义与通过类型定义说明符引入的一样。特别是,它不定义新的类型,也不应该出现在类型ID中。
\end{myTip}

当涉及到为数组类型和函数指针类型创建别名时,别名声明更加易读,也更清晰地表明了别名的实际类型。在“How to do it...”部分的示例中,很容易理解array\_t是包含10个整数数组类型的名称,而fn是接受两个byte和double类型参数并返回void的函数类型的名称,与声明std::function对象的语法相一致(例如,std::function<void(byte, double)> f)。

还需要注意以下几点:

\begin{itemize}
\item
别名模板不能部分或显式特化。

\item
模板参数推导过程中,别名模板永远不会推导为模板参数。

\item
特化别名模板时产生的类型,不允许直接或间接地使用自定义类型。
\end{itemize}

新语法的主要目的是定义别名模板。这些模板在特化时,等同于将别名模板的模板参数替换到类型ID后得到的结果。






