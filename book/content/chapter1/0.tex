随着C++11、C++14、C++17、C++20和C++23的陆续推出,C++经历了多次重大的变革。这些新标准引入了很多新概念,简化并扩展了现有的语法和语义,改变了C++的编程方式。现在,C++11看起来和感觉上就像是全新的语言,使用这些新标准编写的代码可称为“现代C++”代码。本章将介绍C++11引入的一些语言特性,这些特性能够帮助完成许多编程任务。然而,语言的核心详解超出了本章讨论的范畴,其他章节会探讨不同的特性。

本章包括以下内容:

\begin{itemize}
\item
尽可能使用auto

\item
创建类型和模板的别名

\item
统一初始化

\item
非静态成员的初始化

\item
控制和查询对象的对齐状态

\item
作用域枚举

\item
对虚函数使用override和final

\item
基于范围的for循环

\item
为自定义类型启用基于范围的for循环

\item
避免隐式转换——显式构造函数和转换操作符

\item
使用匿名命名空间

\item
使用内联命名空间进行API版本控制

\item
使用结构化绑定处理多个返回值

\item
使用类模板参数推导简化代码

\item
使用下标操作符访问集合中的元素
\end{itemize}


