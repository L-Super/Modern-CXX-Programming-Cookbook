
许多编程语言支持一种称为 for-each 的循环——即对集合中的元素重复执行一组语句。直到 C++11,才在核心语言层面支持这种特性,尽管之前标准库中有一个通用算法 std::for\_each,可以将一个函数应用于某个范围内的所有元素。C++11 引入了语言级别的 for-each 支持,称为基于范围的for循环 (Range-Based For Loops)。新的 C++17 标准对这一原始语言特性进行了改进。

\mySubsubsection{}{Getting ready}

C++11 中,基于范围的循环语法如下:

\begin{cpp}
for ( range_declaration : range_expression ) loop_statement
\end{cpp}

从 C++20 开始,在范围声明之前可以有一个初始化语句(必须以分号结束):

\begin{cpp}
for(init-statement range-declaration : range-expression)
loop-statement
\end{cpp}

为了举例说明使用基于范围循环的各种方式,将使用以下返回元素序列的函数:

\begin{cpp}
std::vector<int> getRates()
{
    return std::vector<int> {1, 1, 2, 3, 5, 8, 13};
}
std::multimap<int, bool> getRates2()
{
    return std::multimap<int, bool> {
        { 1, true },
        { 1, true },
        { 2, false },
        { 3, true },
        { 5, true },
        { 8, false },
        { 13, true }
    };
}
\end{cpp}

\mySubsubsection{}{How to do it...}

基于范围循环可用的多种方式:

\begin{itemize}
\item
通过为序列中的元素指定特定类型:

\begin{cpp}
auto rates = getRates();
for (int rate : rates)
    std::cout << rate << '\n';
for (int& rate : rates)
    rate *= 2;
\end{cpp}

\item
不指定类型,让编译器推断:

\begin{cpp}
for (auto&& rate : getRates())
    std::cout << rate << '\n';
for (auto & rate : rates)
    rate *= 2;
for (auto const & rate : rates)
    std::cout << rate << '\n';
\end{cpp}

\item
使用 C++17 中的结构化绑定和分解声明:

\begin{cpp}
for (auto&& [rate, flag] : getRates2())
    std::cout << rate << '\n';
\end{cpp}
\end{itemize}


\mySubsubsection{}{How it works...}

"How to do it..."中展示的范围基循环表达式实际上是语法糖,编译器会将其转换成其他东西。C++17 之前,编译器生成的代码为:

\begin{cpp}
{
    auto && __range = range_expression;
    for (auto __begin = begin_expr, __end = end_expr;
    __begin != __end; ++__begin) {
        range_declaration = *__begin;
        loop_statement
    }
}
\end{cpp}

begin\_expr 和 end\_expr 是什么取决于范围的类型:

\begin{itemize}
\item
对于 C 风格数组:\_\_range 和 \_\_range + \_\_bound(其中 \_\_bound 是数组中的元素数量)。

\item
对于带有 begin 和 end 成员的类型(不论其类型和访问权限):\_\_range.begin() 和 \_\_range.end()。

\item
对于其他类型,是 begin(\_\_range) 和 end(\_\_range),这些通过参数依赖查找确定。
\end{itemize}

需要注意的是,如果一个类型包含名为 begin 或 end 的成员(无论是函数、数据成员还是枚举器),无论其类型和访问权限如何,这些成员都会选作 begin\_expr 和 end\_expr,所以这样的类型不能用于基于范围的循环。

C++17 开始,编译器生成的代码略有不同:

\begin{cpp}
{
    auto && __range = range_expression;
    auto __begin = begin_expr;
    auto __end = end_expr;
    for (; __begin != __end; ++__begin) {
        range_declaration = *__begin;
        loop_statement
    }
}
\end{cpp}

新标准移除了 begin 表达式和 end 表达式必须是同种类型的限制。end 表达式不必是一个真正的迭代器,但必须能够与迭代器进行不等比较。原因是,范围可以由谓词界定。相反,end 表达式只会在首次使用时计算一次,而不是每次循环迭代时都重新计算,这会提高程序的性能。

C++20 开始,范围声明前可以有一个初始化语句:

\begin{cpp}
{
    init-statement
    auto && __range = range_expression;
    auto __begin = begin_expr;
    auto __end = end_expr;
    for (; __begin != __end; ++__begin) {
        range_declaration = *__begin;
        loop_statement
    }
}
\end{cpp}

初始化语句可以是空语句、表达式语句、简单声明,或者从 C++23 开始,还可以是别名声明:

\begin{cpp}
for (auto rates = getRates(); int rate : rates)
{
    std::cout << rate << '\n';
}
\end{cpp}

C++23 之前,这对于避免范围表达式中的临时对象导致未定义行为很有帮助。范围表达式返回的临时对象的生命期会延长到循环结束。但如果这些临时对象会在范围表达式结束时销毁,则它们的生命周期不会延长。

可以通过以下代码段来解释这一点:

\begin{cpp}
struct item
{
    std::vector<int> getRates()
    {
        return std::vector<int> {1, 1, 2, 3, 5, 8, 13};
    }
};
item make_item()
{
    return item{};
}
// undefined behavior, until C++23
for (int rate : make_item().getRates())
{
    std::cout << rate << '\n';
}
\end{cpp}

由于 make\_item() 按值返回,在范围表达式中有一个临时对象。这会导致未定义行为,可以通过使用初始化语句来避免:

\begin{cpp}
for (auto item = make_item(); int rate : item.getRates())
{
    std::cout << rate << '\n';
}
\end{cpp}

这个问题在 C++23 中不再出现,因为标准延长了范围表达式内所有临时对象的生命周期,会到循环结束。







