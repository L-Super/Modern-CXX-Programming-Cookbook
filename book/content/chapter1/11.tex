
当程序需要链接到多个翻译单元时,遇到命名冲突的概率就越大。在源文件中声明的、本意只为该翻译单元局部使用的函数或变量,可能与其他翻译单元中声明的类似函数或变量发生冲突。

因为所有未声明为static的符号都具有外部链接性,其名称在整个程序中必须唯一。这个问题的传统C语言解决方法是将这些符号声明为static,从而将其链接性从外部改为内部,使得它们成为翻译单元的局部变量。另一种方法是在名字前加上所属模块或库的名字作为前缀。本示例中,将介绍C++对此问题的解决方案。

\mySubsubsection{}{Getting ready}

本示例中,将介绍全局函数和静态函数的概念,以及变量、命名空间和翻译单元。希望读者们对这些概念有一个基本的理解。除此之外,还需要了解内部链接和外部链接的区别;这是本示例的关键。

\mySubsubsection{}{How to do it...}

需要将全局符号声明为static以避免链接问题的情况时,应该优先使用匿名命名空间:

\begin{enumerate}
\item
在源文件中声明一个没有名字的命名空间。

\item
将全局函数或变量的定义放在匿名命名空间中,而不需要将其声明为static。
\end{enumerate}

以下示例展示了两个不同的翻译单元中定义的名为print()的函数;每个函数都在匿名命名空间中定义:

\begin{cpp}
// file1.cpp
namespace
{
    void print(std::string const & message)
    {
        std::cout << "[file1] " << message << '\n';
    }
}
void file1_run()
{
    print("run");
}
// file2.cpp
namespace
{
    void print(std::string const & message)
    {
        std::cout << "[file2] " << message << '\n';
    }
}
void file2_run()
{
    print("run");
}
\end{cpp}

\mySubsubsection{}{How it works...}

当一个函数在一个翻译单元中声明时,具有外部链接性,所以来自两个不同翻译单元的同名函数会导致链接错误(不能有两个同名符号)。这个问题在C和C++中,通过将函数或变量声明为static来解决,从而改变其链接性从外部变为内部。这种情况下,其名字不再导出到翻译单元之外,链接问题也就得以避免。

C++中的正确解决方案是使用匿名命名空间,像前面那样定义命名空间时,编译器会将其转换为如下形式:

\begin{cpp}
// file1.cpp
namespace _unique_name_ {}
using namespace _unique_name_;
namespace _unique_name_
{
    void print(std::string message)
    {
        std::cout << "[file1] " << message << '\n';
    }
}
void file1_run()
{
    print("run");
}
\end{cpp}

首先,声明一个具有唯一名称的命名空间(这个名字是什么以及如何生成,取决于编译器的具体实现)。此时,命名空间为空,这一行的基本目的是建立命名空间。其次,using指令将\_unique\_name\_命名空间中的所有内容引入当前命名空间。最后,使用编译器生成的名称定义命名空间。

通过在匿名命名空间中定义翻译单元局部的print()函数,仅具有局部可见性,但由于具有外部唯一的名称,不会因外部链接性而导致链接错误。

匿名命名空间还在涉及模板的一个或许更为隐蔽的情况下发挥作用。C++11之前,模板非类型参数不能是具有内部链接性的名称,因此不能使用静态变量。匿名命名空间中的符号具有外部链接性,可以作为模板参数使用。尽管C++11放宽了对模板非类型参数的链接限制,但在最新版本的VC++编译器中,这一限制仍然存在:

\begin{cpp}
template <int const& Size>
class test {};
static int Size1 = 10;
namespace
{
    int Size2 = 10;
}
test<Size1> t1;
test<Size2> t2;
\end{cpp}

t1变量的声明会导致编译错误,非类型参数表达式Size1具有内部链接性。t2变量的声明是正确的,Size2具有外部链接性。(注意,使用Clang和GCC编译此代码段不会出错)











