标准库提供了多种类型的容器,用于存储对象集合;库中包含序列容器(如 vector、array 和 list)、有序和无序关联容器(如 set 和 map),以及不存储数据但提供面向容器适配接口的容器适配器(如 stack 和 queue)。它们都以类型模板进行实现,所以可以与任何类型一起使用(前提是满足容器的要求)。一般来说,应该使用最适合解决问题的容器,这不仅能在插入、删除、访问元素的速度和内存使用方面提供良好的性能,还能使代码易于阅读和维护。然而,默认的选择应该是 vector。这个示例中,将了解到为什么在许多情况下 vector 应该作为首选容器,并且会介绍 vector 最常见的操作。

\mySubsubsection{}{Getting ready}

对于本示例,需要熟悉数组,无论是静态分配还是动态分配。这里提供了一些例子:

\begin{cpp}
double d[3];           // a statically allocated array of 3 doubles
int* arr = new int[5]; // a dynamically allocated array of 5 ints
\end{cpp}

vector 类模板在 <vector> 头文件中可用,位于 std 命名空间下。

\mySubsubsection{}{How to do it...}

要初始化一个 std::vector 类模板,可以使用以下任意方法:

\begin{itemize}
\item
从初始化列表初始化:

\begin{cpp}
std::vector<int> v1 { 1, 2, 3, 4, 5 };
\end{cpp}

\item
从数组初始化:

\begin{cpp}
int arr[] = { 1, 2, 3, 4, 5 };
std::vector<int> v21(arr, arr + 5); // v21 = { 1, 2, 3, 4, 5 }
std::vector<int> v22(arr+1, arr+4); // v22 = { 2, 3, 4 }
\end{cpp}

\item
从另一个容器初始化:

\begin{cpp}
std::list<int> l{ 1, 2, 3, 4, 5 };
std::vector<int> v3(l.begin(), l.end()); //{ 1, 2, 3, 4, 5 }
\end{cpp}

\item
从计数和值初始化:

\begin{cpp}
std::vector<int> v4(5, 1); // {1, 1, 1, 1, 1}
\end{cpp}

\end{itemize}

要修改 std::vector 的内容,可以使用以下方法(同样不限于这些方法):

\begin{itemize}
\item
使用 \verb|push_back()| 在 vector 的末尾添加一个元素:

\begin{cpp}
std::vector<int> v1{ 1, 2, 3, 4, 5 };
v1.push_back(6); // v1 = { 1, 2, 3, 4, 5, 6 }
\end{cpp}

\item
使用 \verb|pop_back()| 从 vector 的末尾移除一个元素:

\begin{cpp}
v1.pop_back();   // v1 = { 1, 2, 3, 4, 5 }
\end{cpp}

\item
使用 \verb|insert()| 在 vector 的任意位置插入:

\begin{cpp}
int arr[] = { 1, 2, 3, 4, 5 };
std::vector<int> v21;
v21.insert(v21.begin(), arr, arr + 5); // v21 = { 1, 2, 3, 4, 5 }
std::vector<int> v22;
v22.insert(v22.begin(), arr, arr + 3); // v22 = { 1, 2, 3 }
\end{cpp}

\item
使用 \verb|emplace_back()| 创建并添加一个元素到 vector 的末尾:

\begin{cpp}
struct foo
{
    int a;
    double b;
    std::string c;
    foo(int a, double b, std::string const & c) :
    a(a), b(b), c(c) {}
};
std::vector<foo> v3;
v3.emplace_back(1, 1.0, "one"s);
// v3 = { foo{1, 1.0, "one"} }
\end{cpp}

\item
使用 \verb|emplace()| 创建并插入一个元素到 vector 的任意位置:

\begin{cpp}
v3.emplace(v3.begin(), 2, 2.0, "two"s);
// v3 = { foo{2, 2.0, "two"}, foo{1, 1.0, "one"} }
\end{cpp}
\end{itemize}

要修改 vector 的全部内容,可以使用以下任意方法:

\begin{itemize}
\item
使用赋值operator=从另一个 vector 赋值;这将替换容器的内容:

\begin{cpp}
std::vector<int> v1{ 1, 2, 3, 4, 5 };
std::vector<int> v2{ 10, 20, 30 };
v2 = v1; // v2 = { 1, 2, 3, 4, 5 }
\end{cpp}

\item
使用 assign() 方法从由开始和结束迭代器定义的另一个序列赋值;这将替换容器的内容:

\begin{cpp}
int arr[] = { 1, 2, 3, 4, 5 };
std::vector<int> v31;
v31.assign(arr, arr + 5);     // v31 = { 1, 2, 3, 4, 5 }
std::vector<int> v32;
v32.assign(arr + 1, arr + 4); // v32 = { 2, 3, 4 }
\end{cpp}

\item
使用 swap() 方法交换两个 vector 的内容:

\begin{cpp}
std::vector<int> v4{ 1, 2, 3, 4, 5 };
std::vector<int> v5{ 10, 20, 30 };
v4.swap(v5); // v4 = { 10, 20, 30 }, v5 = { 1, 2, 3, 4, 5 }
\end{cpp}

\item
使用 clear() 方法移除所有元素:

\begin{cpp}
std::vector<int> v6{ 1, 2, 3, 4, 5 };
v6.clear(); // v6 = { }
\end{cpp}

\item
使用 erase() 方法(需要一个迭代器或定义了要从 vector 中移除的元素范围的一对迭代器)移除一个或多个元素:

\begin{cpp}
std::vector<int> v7{ 1, 2, 3, 4, 5 };
v7.erase(v7.begin() + 2, v7.begin() + 4); // v7 = { 1, 2, 5 }
\end{cpp}

\item
使用 \verb|std::remove_if()| 函数和 erase() 方法移除满足某个谓词的一个或多个元素:

\begin{cpp}
std::vector<int> v8{ 1, 2, 3, 4, 5 };
auto iterToNext = v8.erase(
    std::remove_if(v8.begin(), v8.end(),
        [](const int n) {return n % 2 == 0; }),
    v8.end());                            // v8 = { 1, 3, 5 }
\end{cpp}

\item
使用 C++20 引入的 \verb|std::erase_if()| 函数移除满足某个谓词的一个或多个元素。类似的 std::erase() 函数也存在:

\begin{cpp}
std::vector<int> v9{ 1, 2, 3, 4, 5 };
auto erasedCount = std::erase_if(v9, [](const int n) {
    return n % 2 == 0; });                // v9 = { 1, 3, 5 }
\end{cpp}
\end{itemize}

获取 vector 中第一个元素的地址,通常是为了将 vector 的内容传递给类似 C 的 API:

\begin{itemize}
\item
使用 data() 方法,返回指向第一个元素的指针,提供对 vector 元素存储的底层连续内存序列的直接访问;此方法自 C++11 起可用:

\begin{cpp}
void process(int const * const arr, size_t const size)
{ /* do something */ }
std::vector<int> v{ 1, 2, 3, 4, 5 };
process(v.data(), v.size());
\end{cpp}

\item
获取第一个元素的地址:

\begin{cpp}
process(&v[0], v.size());
\end{cpp}

\item
获取 front() 方法引用的元素的地址(空 vector 上调用此方法的行为未定义):

\begin{cpp}
process(&v.front(), v.size());
\end{cpp}

\item
获取 begin() 返回的迭代器所指向的元素的地址:

\begin{cpp}
process(&*v.begin(), v.size());
\end{cpp}
\end{itemize}

C++23 中,要修改 vector 的内容,还可以使用以下范围感知的成员函数:

要使用给定范围的元素副本替换 vector 的元素,使用 \verb|assign_range()|:

\begin{cpp}
std::list<int>   l{ 1, 2, 3, 4, 5 };
std::vector<int> v;
v.assign_range(l); // v = {1, 2, 3, 4, 5}
\end{cpp}

要在 vector 的末尾(在结束迭代器之前)追加范围的元素副本,使用 \verb|append_range()|:

\begin{cpp}
std::list<int>   l{ 3, 4, 5 };
std::vector<int> v{ 1, 2 };
v.append_range(l);  // v = {1, 2, 3, 4, 5}
\end{cpp}

要在 vector 的给定迭代器之前插入范围的元素副本,使用 \verb|insert_range()|:

\begin{cpp}
std::list<int>   l{ 2, 3, 4 };
std::vector<int> v{ 1, 5 };
v.insert_range(v.begin() + 1, l); // v = {1, 2, 3, 4, 5}
\end{cpp}

\mySubsubsection{}{How it works...}

std::vector 类设计为最接近数组且与数组兼容的 C++ 容器。vector 是一个元素数量可变的序列,保证在内存中连续存储,这使得 vector 的内容可以轻松传递给接受数组元素指针及其大小的 C 风格函数。

使用 vector的好处包括:

\begin{itemize}
\item
开发人员不需要直接管理内存,容器内部处理了内存的分配、重新分配和释放。

\begin{myNotic}
vector 旨在存储对象实例。如果需要存储指针,请不要存储原始指针而是智能指针。否则,需要处理指向对象的生命周期管理。
\end{myNotic}

\item
能够修改 vector 的大小。

\item
两个 vector 之间可以简单地赋值或连接。

\item
两个 vector 可以直接比较。
\end{itemize}

vector 类是一个非常高效的容器,实现提供了许多大多数开发者无法通过数组实现的优化。对 vector 元素的随机访问以及在 vector 末尾的插入和删除操作是常数时间 O(1) 操作(前提是不需要重新分配内存),而在其他位置的插入和删除则是线性时间 O(n) 操作。

与其他标准容器相比,vector 具有多种优势:

\begin{itemize}
\item
与数组和 C 风格的 API 兼容。如果函数接受数组作为参数,则在将其他容器(除了 std::array 之外)的内容作为参数传递给函数之前,需要将其复制到 vector 中。

\item
对所有容器而言,对元素的访问速度最快(与 std::array 相同)。

\item
存储元素时没有每个元素的内存开销。因为元素像数组一样存储在连续的空间中,所以vector 的内存占用较小,不像其他容器,比如:list 需要额外的指针指向其他元素,或者关联容器需要哈希值。
\end{itemize}

std::vector 在语义上与数组非常相似,但其大小可变。vector 的大小可以增加也可以减少。有两个属性定义了 vector 的大小:

\begin{itemize}
\item
容量是指在不进行额外内存分配的情况下 vector 可以容纳的元素数量;这由 capacity() 方法表示。

\item
大小是指 vector 中实际存在的元素数量;这由 size() 方法表示。
\end{itemize}

大小总是小于或等于容量。当大小等于容量并且需要添加新元素时,必须修改容量以使 vector 有足够的空间容纳更多元素。这时,vector 将分配一块新的内存,并在释放先前分配的内存之前将内容移动到新的位置。尽管听起来这很耗时——确实如此——但实现方式是每次需要更改容量时都将其指数级增加,通常是翻倍。结果是,平均来说,vector 中的每个元素只需要移动一次(因为容量增加时,所有 vector 的元素都会移动,但那之后可以添加相同数量的元素,而不会产生更多的移动,插入在 vector 末尾进行)。

如果事先知道将有多少个元素会插入到 vector 中,可以首先调用 reserve() 方法来增加至少指定数量的容量(如果指定的大小小于当前容量,这个方法什么也不会做),然后再插入元素。

另一方面,如果想释放多余的预留内存,可以使用 \verb|shrink_to_fit()| 方法请求这样做,但这是否释放内存是由实现决定的。从 C++11 开始,还有一个替代这种非强制方法的方式,即与一个临时的空 vector 进行交换:

\begin{cpp}
std::vector<int> v{ 1, 2, 3, 4, 5 };
std::vector<int>().swap(v); // v.size = 0, v.capacity = 0
\end{cpp}

调用 clear() 方法只会从 vector 中移除所有元素,但不会释放内存。

需要注意的是,vector 类实现了一些特定于其他类型容器的操作:

\begin{itemize}
\item
栈(stack):使用 \verb|push_back()| 和 \verb|emplace_back()| 在末尾添加元素,使用 \verb|pop_back()| 从末尾移除元素。注意,\verb|pop_back()| 不返回移除的最后一个元素。如果需要的话,可以在移除元素之前显式地访问,例如:使用 back() 方法。

\item
列表(list):使用 insert() 和 emplace() 在序列中间添加元素,使用 erase() 从序列中的任何位置移除元素。
\end{itemize}

\begin{myTip}
对于 C++ 容器的一个经验法则是,除非有充分的理由使用其他容器,否则应将 std::vector 作为默认容器。
\end{myTip}
