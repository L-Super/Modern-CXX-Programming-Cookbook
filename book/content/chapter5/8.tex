标准库提供了几种用于集合操作的算法,们能够对已排序的范围执行并集、交集或差集操作。本示例中,将介绍这些算法是什么,以及其工作原理。

\mySubsubsection{}{Getting ready}

集合操作的算法使用迭代器工作,可以用于标准容器、数组或具有输入迭代器可用的自定义类型的序列。本示例中的所有例子都将使用 std::vector。

接下来部分的所有例子中,将使用以下范围:

\begin{cpp}
std::vector<int> v1{ 1, 2, 3, 4, 4, 5 };
std::vector<int> v2{ 2, 3, 3, 4, 6, 8 };
std::vector<int> v3;
\end{cpp}

下面的部分中,将探索标准集合操作算法的使用。

\mySubsubsection{}{How to do it...}

使用以下通用算法进行集合操作:

\begin{itemize}
\item
使用\verb|std::set_union()|计算两个范围的并集到第三个范围:

\begin{cpp}
std::set_union(v1.cbegin(), v1.cend(),
               v2.cbegin(), v2.cend(),
               std::back_inserter(v3));
// v3 = {1, 2, 3, 3, 4, 4, 5, 6, 8}
\end{cpp}

\item
使用 \verb|std::merge()| 合并两个范围的内容到第三个范围;与 \verb|std::set_union()| 类似,不同之处在于将输入范围的全部内容复制到输出范围,而不仅仅是其并集:

\begin{cpp}
std::merge(v1.cbegin(), v1.cend(),
           v2.cbegin(), v2.cend(),
           std::back_inserter(v3));
// v3 = {1, 2, 2, 3, 3, 3, 4, 4, 4, 5, 6, 8}
\end{cpp}

\item
使用 \verb|std::set_intersection()| 计算两个范围的交集到第三个范围:

\begin{cpp}
std::set_intersection(v1.cbegin(), v1.cend(),
                      v2.cbegin(), v2.cend(),
                      std::back_inserter(v3));
// v3 = {2, 3, 4}
\end{cpp}

\item
使用 \verb|std::set_difference()| 计算两个范围的差集到第三个范围;输出范围将包含来自第一个范围但不在第二个范围中的元素:

\begin{cpp}
std::set_difference(v1.cbegin(), v1.cend(),
                    v2.cbegin(), v2.cend(),
                    std::back_inserter(v3));
// v3 = {1, 4, 5}
\end{cpp}

\item
使用 \verb|std::set_symmetric_difference()| 计算两个范围的对称差集到第三个范围;输出范围将包含出现在输入范围中,但只出现在一个范围中的元素:

\begin{cpp}
std::set_symmetric_difference(v1.cbegin(), v1.cend(),
                              v2.cbegin(), v2.cend(),
                              std::back_inserter(v3));
// v3 = {1, 3, 4, 5, 6, 8}
\end{cpp}

\item
使用 \verb|std::includes()| 检查一个范围是否是另一个范围的子集(所有元素也都在另一个范围中出现):

\begin{cpp}
std::vector<int> v1{ 1, 2, 3, 4, 4, 5 };
std::vector<int> v2{ 2, 3, 3, 4, 6, 8 };
std::vector<int> v3{ 1, 2, 4 };
std::vector<int> v4{ };
auto i1 = std::includes(v1.cbegin(), v1.cend(),
                        v2.cbegin(), v2.cend()); // i1 = false
auto i2 = std::includes(v1.cbegin(), v1.cend(),
                        v3.cbegin(), v3.cend()); // i2 = true
auto i3 = std::includes(v1.cbegin(), v1.cend(),
                        v4.cbegin(), v4.cend()); // i3 = true
\end{cpp}
\end{itemize}

\mySubsubsection{}{How it works...}

所有从两个输入范围生成新范围的集合操作都有相同的接口,并以相似的方式工作:

\begin{itemize}
\item
接受两个输入范围,每个范围由第一个和最后一个输入迭代器定义。

\item
接受一个输出迭代器,用于指向将插入元素的输出范围。

\item
有一个重载版本,接受一个参数,该参数表示一个比较二元函数对象,该对象必须在第一个参数小于第二个参数时返回 true。如果没有指定比较函数对象,则使用 operator<。

\item
返回一个指向构造后的输出范围末尾之后的迭代器。

\item
输入范围必须使用 operator< 或提供的比较函数排序,具体取决于使用的重载版本。

\item
输出范围不得与两个输入范围中的任何一个重叠。
\end{itemize}

可通过 Task 的 POD 类型的vector示例,来演示它们的工作方式:

\begin{cpp}
struct Task
{
    int         priority;
    std::string name;
};
bool operator<(Task const & lhs, Task const & rhs) {
    return lhs.priority < rhs.priority;
}
bool operator>(Task const & lhs, Task const & rhs) {
    return lhs.priority > rhs.priority;
}
std::vector<Task> v1{
    { 10, "Task 1.1"s },
    { 20, "Task 1.2"s },
    { 20, "Task 1.3"s },
    { 20, "Task 1.4"s },
    { 30, "Task 1.5"s },
    { 50, "Task 1.6"s },
};
std::vector<Task> v2{
    { 20, "Task 2.1"s },
    { 30, "Task 2.2"s },
    { 30, "Task 2.3"s },
    { 30, "Task 2.4"s },
    { 40, "Task 2.5"s },
    { 50, "Task 2.6"s },
};
\end{cpp}

每个算法生成输出范围的具体方式如下:

\begin{itemize}
\item
\verb|std::set_union()| 将存在于一个或两个输入范围中的所有元素复制到输出范围,生成一个新的已排序范围。如果一个元素在第一个范围中出现了 $M$ 次,在第二个范围中出现了 $N$ 次,所有 $M$ 个元素将按照现有的顺序从第一个范围复制到输出范围,然后如果有 $N > M$,则从第二个范围复制 $N – M$ 个元素到输出范围;否则不复制任何元素:

\begin{cpp}
std::vector<Task> v3;
std::set_union(v1.cbegin(), v1.cend(),
v2.cbegin(), v2.cend(),
std::back_inserter(v3));
// v3 = {{10, "Task 1.1"},{20, "Task 1.2"},{20, "Task 1.3"},
//       {20, "Task 1.4"},{30, "Task 1.5"},{30, "Task 2.3"},
//       {30, "Task 2.4"},{40, "Task 2.5"},{50, "Task 1.6"}}
\end{cpp}

\item
\verb|std::merge()| 将两个输入范围中的所有元素复制到输出范围,生成一个根据比较函数排序的新范围:

\begin{cpp}
std::vector<Task> v4;
std::merge(v1.cbegin(), v1.cend(),
           v2.cbegin(), v2.cend(),
           std::back_inserter(v4));
// v4 = {{10, "Task 1.1"},{20, "Task 1.2"},{20, "Task 1.3"},
//       {20, "Task 1.4"},{20, "Task 2.1"},{30, "Task 1.5"},
//       {30, "Task 2.2"},{30, "Task 2.3"},{30, "Task 2.4"},
//       {40, "Task 2.5"},{50, "Task 1.6"},{50, "Task 2.6"}}
\end{cpp}

\item
\verb|std::set_intersection()| 将同时存在于两个输入范围中的所有元素复制到输出范围,生成一个根据比较函数排序的新范围:

\begin{cpp}
std::vector<Task> v5;
std::set_intersection(v1.cbegin(), v1.cend(),
                      v2.cbegin(), v2.cend(),
                      std::back_inserter(v5));
// v5 = {{20, "Task 1.2"},{30, "Task 1.5"},{50, "Task 1.6"}}
\end{cpp}

\item
\verb|std::set_difference()| 将仅存在于第一个输入范围中的所有元素复制到输出范围。对于两个范围中都存在的等效元素,适用以下规则:如果某个元素在第一个范围中出现了 $M$ 次,第二个范围中出现了 $N$ 次,且 $M > N$,则复制 $M – N$ 次;否则不进行复制:

\begin{cpp}
std::vector<Task> v6;
std::set_difference(v1.cbegin(), v1.cend(),
                    v2.cbegin(), v2.cend(),
                    std::back_inserter(v6));
// v6 = {{10, "Task 1.1"},{20, "Task 1.3"},{20, "Task 1.4"}}
\end{cpp}

\item
\verb|std::set_symmetric_difference()| 将仅存在于两个输入范围之一中的所有元素复制到输出范围。如果一个元素在第一个范围中出现了 $M$ 次,在第二个范围中出现了 $N$ 次;如果 $M > N$,则将第一个范围中的最后 $M – N$ 个这样的元素复制到输出范围;否则,将第二个范围中的最后 $N – M$ 个这样的元素复制到输出范围:

\begin{cpp}
std::vector<Task> v7;
std::set_symmetric_difference(v1.cbegin(), v1.cend(),
                              v2.cbegin(), v2.cend(),
                              std::back_inserter(v7));
// v7 = {{10, "Task 1.1"},{20, "Task 1.3"},{20, "Task 1.4"}
//       {30, "Task 2.3"},{30, "Task 2.4"},{40, "Task 2.5"}}
\end{cpp}
\end{itemize}

另一方面,std::includes() 不生成输出范围;仅检查第二个范围是否包含在第一个范围中。返回一个布尔值,如果第二个范围为空或其所有元素都包含在第一个范围中,则返回 true;否则返回 false。还有两个重载版本,其中一个指定了一个比较二元函数实例。

