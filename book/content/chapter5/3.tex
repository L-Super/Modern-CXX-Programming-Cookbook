前一个示例中,介绍了使用 std::bitset 处理固定大小的位序列。有时 std::bitset 并不是一个好的选择,因为在编译时并不知道位的数量,而简单地定义一个足够大的位集合也不是一个好主意,因为可能会遇到实际位数并不足够大的情况。对此的标准替代方案是使用 std::vector<bool> 容器,std::vector 的一个特化版本,实现了空间和速度上的优化。因为其实现并不是存储布尔值,而是为每个元素存储单独的位。

\begin{myNotic}
正因为如此,std::vector<bool> 不符合标准容器或顺序容器的要求,std::vector<bool>::iterator 也不符合前向迭代器的要求。因此,这种特化不能用于期望vector的泛型代码中。作为一个vector,具有与 std::bitset 不同的接口,不能视为数字的二进制表示。没有直接的方法可以从数字或字符串构造 std::vector<bool>,也不能将其转换为数字或字符串。
\end{myNotic}

\mySubsubsection{}{Getting ready}

此示例假设已经熟悉 std::vector 和 std::bitset。

vector<bool> 类在 <vector> 头文件中的 std 命名空间中可用。

\mySubsubsection{}{How to do it...}

要操作 std::vector<bool>,请使用与操作 std::vector<T> 相同的方法:

\begin{itemize}
\item
创建一个空vector:

\begin{cpp}
std::vector<bool> bv; // []
\end{cpp}

\item
向vector中添加位:

\begin{cpp}
bv.push_back(true);  // [1]
bv.push_back(true);  // [1, 1]
bv.push_back(false); // [1, 1, 0]
bv.push_back(false); // [1, 1, 0, 0]
bv.push_back(true);  // [1, 1, 0, 0, 1]
\end{cpp}

\item
设置单个位的值:

\begin{cpp}
bv[3] = true;        // [1, 1, 0, 1, 1]
\end{cpp}

\item
使用泛型算法:

\begin{cpp}
auto count_of_ones = std::count(bv.cbegin(), bv.cend(), true);
\end{cpp}

\item
从vector中移除位:

\begin{cpp}
bv.erase(bv.begin() + 2); // [1, 1, 1, 1]
\end{cpp}
\end{itemize}

\mySubsubsection{}{How it works...}

std::vector<bool> 并不是标准vector,其设计为通过为每个元素存储单个位,而非布尔值来提供空间优化。因此,元素不是存储在一个连续的序列中,不能代替布尔值数组:

\begin{itemize}
索引操作符不能返回对特定元素的引用,元素不是单独存储的:

\begin{cpp}
std::vector<bool> bv;
bv.resize(10);
auto& bit = bv[0];      // error
\end{cpp}

\item
解引用迭代器不能产生对 bool 的引用,原因与上述相同:

\begin{cpp}
auto& bit = *bv.begin(); // error
\end{cpp}

\item
不能保证可以同时从不同线程独立地操作各个位。

\item
vector不能与要求前向迭代器的算法一起使用,如 std::search()。

\item
如果代码期望的是 std::vector<T> 并且需要本列表中提到的操作,则该vector不能用于某些泛型代码中。
\end{itemize}

std::vector<bool> 的一个替代方案是 std::deque<bool>,这是一个标准容器(双端队列),满足所有容器和迭代器的要求,并可以与所有标准算法一起使用,但这不会提供 std::vector<bool> 提供的空间优化。

\mySubsubsection{}{There's more...}

std::vector<bool> 的接口与 std::bitset 非常不同。如果能够以相似的方式编写代码,可以创建一个封装 std::vector<bool> 的包装器,使其看起来像 std::bitset(尽可能)。

以下实现提供了类似于 std::bitset 中可用的成员:

\begin{cpp}
class bitvector
{
    std::vector<bool> bv;
    public:
    bitvector(std::vector<bool> const & bv) : bv(bv) {}
    bool operator[](size_t const i) { return bv[i]; }
    inline bool any() const {
        for (auto b : bv) if (b) return true;
        return false;
    }
    inline bool all() const {
        for (auto b : bv) if (!b) return false;
        return true;
    }
    inline bool none() const { return !any(); }
    inline size_t count() const {
        return std::count(bv.cbegin(), bv.cend(), true);
    }
    inline size_t size() const { return bv.size(); }
    inline bitvector & add(bool const value) {
        bv.push_back(value);
        return *this;
    }
    inline bitvector & remove(size_t const index) {
        if (index >= bv.size())
        throw std::out_of_range("Index out of range");
        bv.erase(bv.begin() + index);
        return *this;
    }
    inline bitvector & set(bool const value = true) {
        for (size_t i = 0; i < bv.size(); ++i)
        bv[i] = value;
        return *this;
    }
    inline bitvector& set(size_t const index, bool const value = true) {
        if (index >= bv.size())
        throw std::out_of_range("Index out of range");
        bv[index] = value;
        return *this;
    }
    inline bitvector & reset() {
        for (size_t i = 0; i < bv.size(); ++i) bv[i] = false;
        return *this;
    }
    inline bitvector & reset(size_t const index) {
        if (index >= bv.size())
        throw std::out_of_range("Index out of range");
        bv[index] = false;
        return *this;
    }
    inline bitvector & flip() {
        bv.flip();
        return *this;
    }
    std::vector<bool>& data() { return bv; }
};
\end{cpp}

这只是一个基本的实现,如果想使用这样的包装器,应该添加其他方法,比如:位逻辑操作、移位、也许是从流中读取和写入等。然而,有了前面的代码,就可以写出以下示例:

\begin{cpp}
bitvector bv;
bv.add(true).add(true).add(false); // [1, 1, 0]
bv.add(false);                     // [1, 1, 0, 0]
bv.add(true);                      // [1, 1, 0, 0, 1]
if (bv.any()) std::cout << "has some 1s" << '\n';
if (bv.all()) std::cout << "has only 1s" << '\n';
if (bv.none()) std::cout << "has no 1s" << '\n';
std::cout << "has " << bv.count() << " 1s" << '\n';
bv.set(2, true);                   // [1, 1, 1, 0, 1]
bv.set();                          // [1, 1, 1, 1, 1]
bv.reset(0);                       // [0, 1, 1, 1, 1]
bv.reset();                        // [0, 0, 0, 0, 0]
bv.flip();                         // [1, 1, 1, 1, 1]
\end{cpp}

这些示例非常类似于使用 std::bitset 的示例。这个 bitvector 类拥有与 std::bitset 兼容的 API,但对于处理可变长度的位序列来说很有用。

