当这些对象或数组需要共享时,不可能使用 \verb|std::unique_ptr| 管理动态分配的对象或数组。因为 \verb|std::unique_ptr| 保持其唯一所有权。C++ 标准提供了另一种智能指针,称为 \verb|std::shared_ptr|;很多方面与 \verb|std::unique_ptr| 相似,但不同之处在于可以与其他 \verb|std::shared_ptr| 实例共享对象或数组的所有权。本示例中,将介绍 \verb|std::shared_ptr| 如何工作,以及它与 \verb|std::unique_ptr| 的区别。还将介绍 \verb|std::weak_ptr|,这是一种非资源拥有的智能指针,其持有一个由 \verb|std::shared_ptr| 管理实例的引用。

\mySubsubsection{}{Getting ready}

请阅读前一个示例以熟悉 \verb|unique_ptr| 和 \verb|make_unique()| 的工作原理。将使用其中定义的\verb|foo|、\verb|foo_deleter|、\verb|Base| 和 \verb|Derived| 类型,并且会多次引用它们。

\verb|shared_ptr| 和 \verb|weak_ptr| 类型,以及 \verb|make_shared()| 函数模板均位于 <memory> 头文件中的 std 命名空间下。

\begin{myNotic}
为了简单和可读性,不会使用完全限定名 \verb|std::unique_ptr|、\verb|std::shared_ptr| 和 \verb|std::weak_ptr|,而是使用 \verb|unique_ptr|、\verb|shared_ptr| 和 \verb|weak_ptr|。
\end{myNotic}

\mySubsubsection{}{How to do it...}

以下是使用 \verb|shared_ptr| 和 \verb|weak_ptr| 时需要了解的一些典型操作:

\begin{itemize}
\item
使用可用的重载构造函数之一,创建通过指针管理对象的 \verb|shared_ptr|。默认构造函数创建一个空的 \verb|shared_ptr|,即不管理任何对象:

\begin{cpp}
std::shared_ptr<int> pnull1;
std::shared_ptr<int> pnull2(nullptr);
std::shared_ptr<int> pi1(new int(42));
std::shared_ptr<int> pi2 = pi1;
std::shared_ptr<foo> pf1(new foo());
std::shared_ptr<foo> pf2(new foo(42, 42.0, "42"));
\end{cpp}

\item
或者,从 C++11 开始,可以使用 \verb|std::make_shared()| 函数模板来创建 \verb|shared_ptr| 对象:

\begin{cpp}
std::shared_ptr<int> pi  = std::make_shared<int>(42);
std::shared_ptr<foo> pf1 = std::make_shared<foo>();
std::shared_ptr<foo> pf2 = std::make_shared<foo>(42, 42.0, "42");
\end{cpp}

\item
C++20可以使用 \verb|std::make_shared_for_overwrite()| 函数模板创建默认初始化的对象或对象数组的 \verb|shared_ptr|。这些对象之后应赋予确定的值:

\begin{cpp}
std::shared_ptr<int> pi = std::make_shared_for_overwrite<int>();
std::shared_ptr<foo[]> pa = std::make_shared_for_overwrite<foo[]>(3);
\end{cpp}

\item
如果默认删除操作不适合销毁托管对象,可以使用带有自定义删除器的重载构造函数:

\begin{cpp}
std::shared_ptr<foo> pf1(new foo(42, 42.0, "42"),
                         foo_deleter());
std::shared_ptr<foo> pf2(
new foo(42, 42.0, "42"),
    [](foo* p) {
        std::cout << "deleting foo from lambda..." << '\n';
        delete p;});
\end{cpp}

\item
当管理对象数组时,总是要指定删除器。删除器可以是 \verb|std::default_delete| 的偏特化版本,也可以是接受模板类型指针的函数:

\begin{cpp}
std::shared_ptr<int> pa1(
    new int[3]{ 1, 2, 3 },
        std::default_delete<int[]>());
std::shared_ptr<int> pa2(
    new int[3]{ 1, 2, 3 },
        [](auto p) {delete[] p; });

\end{cpp}

\item
要访问托管对象的原始指针,可使用 get() 函数:

\begin{cpp}
void func(int* ptr)
{
    if (ptr != nullptr)
        std::cout << *ptr << '\n';
    else
        std::cout << "null" << '\n';
}
std::shared_ptr<int> pi;
func(pi.get());
pi = std::make_shared<int>(42);
func(pi.get());
\end{cpp}

\item
使用操作符 * 和 -> 解引用指向托管对象的指针:

\begin{cpp}
std::shared_ptr<int> pi = std::make_shared<int>(42);
*pi = 21;
std::shared_ptr<foo> pf = std::make_shared<foo>(42, 42.0, "42");
pf->print();
\end{cpp}

\item
如果 \verb|shared_ptr| 管理的是对象数组,可以使用操作符 [] 访问数组中的各个元素。这是 C++17 中引入的功能:

\begin{cpp}
std::shared_ptr<int[]> pa1(
    new int[3]{ 1, 2, 3 },
        std::default_delete<int[]>());
for (int i = 0; i < 3; ++i)
    pa1[i] *= 2;
\end{cpp}

\item
要检查 \verb|shared_ptr| 是否管理了某个对象,可以使用显式的布尔操作符或检查 \verb|get() != nullptr|:

\begin{cpp}
std::shared_ptr<int> pnull;
if (pnull) std::cout << "not null" << '\n';
std::shared_ptr<int> pi(new int(42));
if (pi) std::cout << "not null" << '\n';
\end{cpp}

\item
\verb|shared_ptr| 实例可以存储在容器中,如 std::vector:

\begin{cpp}
std::vector<std::shared_ptr<foo>> data;
for (int i = 0; i < 5; i++)
    data.push_back(
        std::make_shared<foo>(i, i, std::to_string(i)));
auto pf = std::make_shared<foo>(42, 42.0, "42");
data.push_back(std::move(pf));
assert(!pf);
\end{cpp}

\item
使用 \verb|weak_ptr| 维护对共享对象的非拥有引用,这些引用可以在以后通过从 \verb|weak_ptr| 对象构造的 \verb|shared_ptr| 访问:

\begin{cpp}
auto sp1 = std::make_shared<int>(42);
assert(sp1.use_count() == 1);
std::weak_ptr<int> wpi = sp1;
assert(sp1.use_count() == 1);
auto sp2 = wpi.lock(); // sp2 type is std::shared_ptr<int>
assert(sp1.use_count() == 2);
assert(sp2.use_count() == 2);
sp1.reset();
assert(sp1.use_count() == 0);
assert(sp2.use_count() == 1);
\end{cpp}

\item
当需要为另一个 \verb|shared_ptr| 管理的实例创建 \verb|shared_ptr| 时,可以使用 \verb|std::enable_shared_from_this| 类模板作为类型的基类:

\begin{cpp}
struct Apprentice;
struct Master : std::enable_shared_from_this<Master>
{
    ~Master() { std::cout << "~Master" << '\n'; }
    void take_apprentice(std::shared_ptr<Apprentice> a);
    private:
    std::shared_ptr<Apprentice> apprentice;
};
struct Apprentice
{
    ~Apprentice() { std::cout << "~Apprentice" << '\n'; }
    void take_master(std::weak_ptr<Master> m);
    private:
    std::weak_ptr<Master> master;
};
void Master::take_apprentice(std::shared_ptr<Apprentice> a)
{
    apprentice = a;
    apprentice->take_master(shared_from_this());
}
void Apprentice::take_master(std::weak_ptr<Master> m)
{
    master = m;
}
auto m = std::make_shared<Master>();
auto a = std::make_shared<Apprentice>();
m->take_apprentice(a);
\end{cpp}
\end{itemize}

\mySubsubsection{}{How it works...}

\verb|shared_ptr| 在许多方面都非常类似于 \verb|unique_ptr|,但用于不同的目的:共享对象或数组的所有权。两个或更多的 \verb|shared_ptr| 智能指针可以管理同一个动态分配的对象或数组,当最后一个智能指针超出作用域、使用赋值操作符 = 赋予新指针或使用方法 reset() 重置时,对象会自动销毁。默认情况下,对象是通过删除操作符 delete 销毁;但是用户可以向构造函数提供自定义删除器,这是使用 \verb|std::make_shared()| 无法做到的。如果 \verb|shared_ptr| 用于管理对象数组,则必须提供自定义删除器。可以使用 \verb|std::default_delete<T[]>|,其 \verb|std::default_delete| 类模板的偏特化版本,使用删除操作符 delete[] 删除动态分配的数组。

实用函数 \verb|std::make_shared()|(自 C++11 起可用),不像自 C++14 起才可用的 \verb|std::make_unique()|,除非需要提供自定义删除器,否则应该用来创建智能指针。这样做的主要原因是与 \verb|make_unique()| 相同:避免在某些上下文中抛出异常时可能出现的潜在内存泄漏。更多信息,请参阅前一个示例中提供的关于 \verb|std::make_unique()| 的解释。

C++20增加了一个新的函数,叫做 \verb|std::make_shared_for_overwrite()|。与 \verb|make_shared()| 类似,不同之处在于默认初始化对象或对象数组。此函数可以在通用代码中使用,当未知类型模板参数是否是平凡可复制时。此函数表达了创建一个可能未初始化的对象指针的意图,以便稍后对其进行覆盖。

同样地,就像 \verb|unique_ptr| 的情况一样,管理派生类对象的 \verb|shared_ptr| 可以隐式转换为管理基类对象的 \verb|shared_ptr|。如果派生类 Derived 继承自 Base,则这是可能的。这种隐式转换只有在 Base 具有虚析构函数时才是安全的(当对象应该通过指向基类的指针或引用多态地删除时,所有的基类都应该是这样的);否则,将产生未定义行为。C++17添加了几个新的非成员函数:\verb|std::static_pointer_cast()|、\verb|std::dynamic_pointer_cast()|、\verb|std::const_pointer_cast()| 和 \verb|std::reinterpret_pointer_cast()|。这些函数分别应用 \verb|static_cast|、\verb|dynamic_cast|、\verb|const_cast| 和 \verb|reinterpret_cast| 到存储的指针,返回一个新的指定类型的 \verb|shared_ptr|。

下面的例子中,Base 和 Derived 是前一个示例中使用的类型:

\begin{cpp}
std::shared_ptr<Derived> pd = std::make_shared<Derived>();
std::shared_ptr<Base> pb = pd;
std::static_pointer_cast<Derived>(pb)->print();
\end{cpp}

当需要一个智能指针来引用一个共享对象,但又不希望它参与到共享所有权中。假设构建了一个树形结构,其中节点对它的子节点有引用,这些子节点由 \verb|shared_ptr| 表示。另一方面,假设一个节点需要保持对其父节点的引用。如果这个引用也是 \verb|shared_ptr|,将创建循环引用,导致没有对象能够自动销毁。

\verb|weak_ptr| 是一种用于打破此类循环依赖关系的智能指针,其持有由 \verb|shared_ptr| 管理的对象或数组的非拥有引用。可以由 \verb|shared_ptr| 创建 \verb|weak_ptr|。为了访问托管对象,需要获取一个临时的 \verb|shared_ptr|。为此,需要使用 \verb|lock()| 方法。该方法原子地检查所引用的对象是否仍然存在,并返回一个空的 \verb|shared_ptr|(如果对象已不存在)或者一个拥有该对象的 \verb|shared_ptr|(如果对象仍然存在)。由于 \verb|weak_ptr| 是一个非拥有智能指针,在 \verb|weak_ptr| 超出作用域之前,或在所有拥有该对象的 \verb|shared_ptr| 对象已销毁、重置或分配给其他指针之前,所引用的对象可能会销毁。可以使用 \verb|expired()| 方法来检查引用的对象是否已销毁或仍然可用。

前面的例子模拟了师徒关系。有一个 Master 类型和一个 Apprentice 类型。Master 类型有一个对 Apprentice 类型的引用和一个名为 \verb|take_apprentice()| 的方法来设置 Apprentice 实例。Apprentice 类型有一个对 Master 类型的引用和一个名为 \verb|take_master()| 的方法来设置 Master 实例。为了避免循环依赖,这两个引用之一必须由 \verb|weak_ptr| 表示。例子中,Master 类型拥有一个 \verb|shared_ptr| 来管理 Apprentice 实例,而 Apprentice 类型则拥有一个 \verb|weak_ptr| 来跟踪 Master 实例的引用。这个例子稍微复杂一些,因为\verb|Apprentice::take_master()| 是在 \verb|Master::take_apprentice()| 方法中调用的,需要一个 \verb|weak_ptr<Master>|。为了从 Master 类型内部调用的,必须能够在 Master 类型中使用 this 指针创建一个 \verb|shared_ptr<Master>|。唯一安全地实现是使用 \verb|std::enable_shared_from_this|。

\verb|std::enable_shared_from_this| 是一个类型模板,必须作为需要为当前对象(即 this 指针)创建一个 \verb|shared_ptr| 的类型的基类,前提是这个对象已经由另一个 \verb|shared_ptr| 管理。其类型模板参数必须是继承自它的那个类型,这遵循了奇特递归模板模式(Curiously Recurring Template Pattern)。它有两个方法:\verb|shared_from_this()| 返回一个共享当前对象所有权的 \verb|shared_ptr|,而 \verb|weak_from_this()| 返回一个持有当前对象非拥有引用的 \verb|weak_ptr|。后者方法仅在 C++17 中可用。这些方法只能在一个由现有的 \verb|shared_ptr| 管理的对象上调用;否则,从 C++17 开始,则会抛出一个 \verb|std::bad_weak_ptr| 异常。在 C++17 之前,这种行为未定义。

如果不使用 \verb|std::enable_shared_from_this| 并直接创建 \verb|shared_ptr<T>(this)|,这将导致有多个 \verb|shared_ptr| 对象独立地管理同一个对象,彼此之间互不知情。当这种情况发生时,该对象最终会被不同的 \verb|shared_ptr| 对象多次销毁。

