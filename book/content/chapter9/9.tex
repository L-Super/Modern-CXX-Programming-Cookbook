移动语义是推动现代 C++ 性能提升的关键特性,使得资源或通常昂贵复制的对象能够移动而不是复制。然而,这需要类实现移动构造函数和移动赋值操作符。某些情况下,编译器会提供这些函数。但实践中,往往需要显式地编写。本示例中,将介绍如何实现移动构造函数和移动赋值操作符。

\mySubsubsection{}{Getting ready}

应该具备关于右值引用和特殊类函数(构造函数、赋值操作符和析构函数)的基本知识。将通过以下 Buffer 类型来演示如何实现移动构造函数和赋值操作符:

\begin{cpp}
class Buffer
{
    unsigned char* ptr;
    size_t length;
public:
    Buffer(): ptr(nullptr), length(0)
    {}
    explicit Buffer(size_t const size):
        ptr(new unsigned char[size] {0}), length(size)
    {}
    ~Buffer()
    {
        delete[] ptr;
    }
    Buffer(Buffer const& other):
        ptr(new unsigned char[other.length]),
    length(other.length)
    {
        std::copy(other.ptr, other.ptr + other.length, ptr);
    }
    Buffer& operator=(Buffer const& other)
    {
        if (this != &other)
        {
            delete[] ptr;
            ptr = new unsigned char[other.length];
            length = other.length;
            std::copy(other.ptr, other.ptr + other.length, ptr);
        }
        return *this;
    }
    size_t size() const { return length;}
    unsigned char* data() const { return ptr; }
};
\end{cpp}

\mySubsubsection{}{How to do it...}

要为一个类型实现移动构造函数,请按照以下步骤操作:

\begin{enumerate}
\item
编写一个接受该类类型右值引用的构造函数:

\begin{cpp}
Buffer(Buffer&& other)
{
}
\end{cpp}

\item
将来自右值引用的所有数据成员分配给当前实例。这可以在构造函数体内完成,或者在初始化列表中完成,后者是首选方式:

\begin{cpp}
ptr = other.ptr;
length = other.length;
\end{cpp}

\item
可选地,将右值引用的数据成员分配给默认值(确保移动的实例处于可销毁状态):

\begin{cpp}
other.ptr = nullptr;
other.length = 0;
\end{cpp}
\end{enumerate}

综合以上,Buffer 类型的移动构造函数:

\begin{cpp}
Buffer(Buffer&& other)
{
    ptr = other.ptr;
    length = other.length;
    other.ptr = nullptr;
    other.length = 0;
}
\end{cpp}

要为一个类实现移动赋值操作符,请按照以下步骤操作:

\begin{enumerate}
\item
编写一个接受该类类型右值引用,并返回其引用的赋值操作符:

\begin{cpp}
Buffer& operator=(Buffer&& other)
{
}
\end{cpp}

\item
检查右值引用是否指向同一个对象,如果不是,则执行第 3 至第 5 步:

\begin{cpp}
if (this != &other)
{
}
\end{cpp}

\item
释放当前对象的所有资源(如内存、句柄等):

\begin{cpp}
delete[] ptr;
\end{cpp}

\item
将来自右值引用的所有数据成员分配给当前实例:

\begin{cpp}
ptr = other.ptr;
length = other.length;
\end{cpp}

\item
将右值引用的数据成员分配给默认值:

\begin{cpp}
other.ptr = nullptr;
other.length = 0;
\end{cpp}

\item
不管第 3 至第 5 步是否执行,都返回当前实例的引用:

\begin{cpp}
return *this;
\end{cpp}
\end{enumerate}

综合以上,Buffer 类型的移动赋值操作符:

\begin{cpp}
Buffer& operator=(Buffer&& other)
{
    if (this != &other)
    {
        delete[] ptr;
        ptr = other.ptr;
        length = other.length;
        other.ptr = nullptr;
        other.length = 0;
    }
    return *this;
}
\end{cpp}

\mySubsubsection{}{How it works...}

除非已经存在用户定义的复制构造函数、移动构造函数、复制赋值操作符、移动赋值操作符或析构函数,否则编译器会默认提供移动构造函数和移动赋值操作符。当由编译器提供时,以成员的方式执行移动操作。移动构造函数递归地调用类数据成员的移动构造函数;同样,移动赋值操作符递归地调用类数据成员的移动赋值操作符。

这种情况下,移动对于太大而无法复制的对象(如字符串或容器)或不应该复制的对象(如 \verb|unique_ptr| 智能指针)来说是一种性能上的优势。并不是所有的类型都需要同时实现复制和移动语义。一些类应该是只可移动的,而另一些类应该是既可复制又可移动的。另一方面,一个类可以是可复制但不可移动的,虽然技术上可以实现,但这并没有太大的意义。

并不是所有的类型都能从移动语义中受益。对于内置类型(如 bool、int 或 double)、数组或 POD(Plain Old Data)类型,移动实际上就是复制操作。另一方面,在右值(即临时对象)的上下文中,移动语义提供了性能上的优势。右值是没有名字的对象;在表达式求值期间暂时存在,并在遇到下一个分号时销毁:

\begin{cpp}
T a;
T b = a;
T c = a + b;
\end{cpp}

上面的例子中,a、b 和 c 是左值;是有名字的实例,可以在其生命周期中的任何时候用来引用该实例。另一方面,当计算表达式 a+b 时,编译器会创建一个临时实例(这个例子中,赋给了 c),然后当遇到分号时,该临时实例就会销毁。这些临时实例称为右值,通常出现在赋值表达式的右侧。C++11可以通过右值引用来引用这些对象,用 \verb|&&| 表示。

在右值的上下文中,移动语义很重要。可以从即将销毁的临时实例那里接管所有权,而且移动操作完成后无法再使用该实例。另一方面,左值不能移动,只能复制。这是因为在移动操作后仍然可以访问,用户期望对象保持原来的状态。例如,在上述例子中,表达式 b = a 将 a 赋值给 b。

完成这个操作之后,作为左值的 a 实例仍然可以使用,并且应该保持与之前相同的状态。另一方面,a+b 的结果是临时的,其数据可以安全地移动到 c。

移动构造函数与拷贝构造函数不同,接受的是该类类型的右值引用 \verb|T(T&&)|,而复制构造函数接受的是左值引用 \verb|T(T const&)|。同样,移动赋值接受的是右值引用 \verb|T& operator=(T&&)|,而复制赋值接受的是左值引用 \verb|T& operator=(T const &)|。即使两者都返回对 \verb|T&| 类的引用,编译器也会根据参数的类型(右值或左值)选择合适的构造函数或赋值操作符。

当存在移动构造函数/赋值操作符时,右值会自动移动。左值也可以移动,但这需要显式地转换为右值引用。这可以通过 std::move() 函数来完成(执行的是 \verb|static_cast<T&&>|):

\begin{cpp}
std::vector<Buffer> c;
c.push_back(Buffer(100));  // move
Buffer b(200);
c.push_back(b);            // copy
c.push_back(std::move(b)); // move
\end{cpp}

对象移动后,必须保持在一个有效的状态,对于这个状态的具体形式没有具体要求。为了保持一致性,应该将所有成员字段设置为其默认值(数值类型设为 0,指针设为 nullptr,布尔值设为 false 等)。

下面的例子展示了 Buffer 实例可以构造和赋值的不同方式:

\begin{cpp}
Buffer b1;                // default constructor
Buffer b2(100);           // explicit constructor
Buffer b3(b2);            // copy constructor
b1 = b3;                  // assignment operator
Buffer b4(std::move(b1)); // move constructor
b3 = std::move(b4);       // move assignment
\end{cpp}

每个实例的构造或赋值过程中,涉及的构造函数或赋值操作符已在注释中提到。

\mySubsubsection{}{There's more...}

正如 Buffer 示例中所见,实现移动构造函数和移动赋值操作符涉及到编写相似的代码(移动构造函数中的整个代码也在移动赋值操作符中出现)。实际上,这可以通过在移动构造函数中调用移动赋值操作符来避免(或者,另寻途径将赋值代码提炼到一个私有函数中,该函数由移动构造函数和移动赋值操作符调用):

\begin{cpp}
Buffer(Buffer&& other) : ptr(nullptr), length(0)
{
    *this = std::move(other);
}
\end{cpp}

例子中有两点需要注意:

\begin{itemize}
\item
构造函数初始化列表中的成员初始化是必要的,这些成员可能会在稍后的移动赋值操作符中使用(例如,这里的 ptr 成员)。

\item
其他对象被静态转换为右值引用。如果没有这个显式的转换,将会调用复制赋值操作符。因为即使一个对象是右值,如果没有显式地转换为右值引用,编译器也会认为它是左值。
\end{itemize}




