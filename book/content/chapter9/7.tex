作为一种多范式编程语言,C++支持面向对象编程。多态性,作为面向对象编程的核心原则,在C++中有两种形式:编译时多态,通过函数和操作符重载实现,以及运行时多态,通过虚函数实现。虚函数允许派生类覆盖基类中函数的实现。然而,C++20中虚函数允许出现在常量表达式中,所以可以在编译时调用。

\mySubsubsection{}{Getting ready}

本示例中,将在随后的例子中使用以下结构来表示文档和信封的尺寸:

\begin{cpp}
struct dimension
{
    double width;
    double height;
};
\end{cpp}

\mySubsubsection{}{How to do it...}

可以通过以下步骤将运行时多态性移动到编译时:

\begin{itemize}
\item
将要移动到编译时调用的虚函数声明为constexpr。

\item
将继承层次结构中基类的析构函数声明为constexpr。

\item
将重写的虚函数声明为constexpr。

\item
在常量表达式中调用constexpr虚函数。
\end{itemize}

以下代码展示了这一过程:

\begin{cpp}
struct document_type
{
    constexpr virtual ~document_type() {};
    constexpr virtual dimension size() const = 0;
};
struct document_a5 : document_type
{
    constexpr dimension size() const override { return { 148.5, 210 }; }
};
struct envelope_type
{
    constexpr virtual ~envelope_type() {}
    constexpr virtual dimension size() const = 0;
    constexpr virtual dimension max_enclosure_size() const = 0;
};
struct envelop_commercial_8 : envelope_type
{
    constexpr dimension size() const override { return { 219, 92 }; }
    constexpr dimension max_enclosure_size() const override
    { return { 213, 86 }; }
};
constexpr bool document_fits_envelope(document_type const& d,
envelope_type const& e)
{
    return e.max_enclosure_size().width >= d.size().width;
}
int main()
{
    constexpr envelop_commercial_8 e1;
    constexpr document_a5          d1;
    static_assert(document_fits_envelope(d1, e1));
}
\end{cpp}

\mySubsubsection{}{How it works...}

C++20之前,虚函数不能声明为constexpr。然而,用于常量表达式实例的动态类型必须在编译时已知。因此,C++20中取消了使虚函数成为constexpr的限制。

拥有constexpr虚函数的优势在于,某些计算可以从运行时移至编译时。虽然实践中对许多使用场景没有实际影响,但在上一节中给出了一个例子。

有一系列不同的文档纸张尺寸,例如:包括A3、A4、A5、法律、信纸和半信纸。尺寸不同的,例如:A5是148.5毫米 x 210毫米,而信纸是215.9毫米 x 279.4毫米。

另一方面,有各种类型和大小的信封。例如,有一个信封尺寸为92毫米 x 219毫米,最大装入尺寸为86毫米 x 213毫米。要编写一个函数来确定某种类型的折叠纸能否放入信封中。由于尺寸是标准的,在编译时已知。所以可以在此时进行检查,而不是在运行时。

为此,在“How to do it...”部分中,已经看到了:

\begin{itemize}
\item
文档的继承层次结构,基类称为\verb|document_type|。有两个成员:一个虚析构函数和一个虚函数size(),该函数返回纸张的尺寸。这两个函数都是constexpr。

\item
信封的继承层次结构,基类称为\verb|envelope_type|。有三个成员:一个虚析构函数、一个虚函数size(),该函数返回信封的尺寸,以及一个虚函数\verb|max_enclosure_size()|,该函数返回可以放入信封的最大(折叠)纸张尺寸。所有这些都是constexpr。

\item
一个名为\verb|document_fits_envelope()|的独立函数,通过比较两者的宽度,来确定给定文档类型是否适合特定的信封类型。这也是一个constexpr函数。
\end{itemize}

因为提到的所有函数都是constexpr,所以在常量表达式中可以调用\verb|document_fits_envelope()|函数,比如\verb|static_assert|,前提是调用其实例也是constexpr。在随书附带的代码文件中,会发现一个关于各种纸张和信封尺寸的详细例子。

你应该记住:

\begin{itemize}
\item
即使基类中被覆盖的函数未定义为constexpr,也可以将覆写的虚函数声明为constexpr。

\item
相反的情况也是可能的,派生类中覆写的虚函数可以是非constexpr的,即使该函数在基类中定义为constexpr。

\item
如果在一个多层次的继承结构中,一个虚函数的一些覆写版本定义为constexpr,而一些则不是,则最终决定该虚函数是否为constexpr的,是针对该函数调用实例适用的最终覆写版本。
\end{itemize}



