
异常规格是C++语言的一个特性,可以带来性能上的提升,但如果使用不当,也可能导致程序异常终止。C++03引入异常规格,允许指出函数可能抛出哪些类型的异常,这一特性在C++11中弃用,并在C++17中移除,被C++11中的noexcept修饰符所取代。此外,使用throw()修饰符来表明函数会抛出异常,但不指定异常类型的做法也在C++17中弃用,并在C++20中完全移除。noexcept修饰符只允许声明一个函数不会抛出异常(与旧的throw修饰符不同,后者可以指定函数可能会抛出的异常类型列表)。本节提供了关于现代C++异常规格的信息,以及何时使用它们。

\mySubsubsection{}{How to do it...}

使用以下结构来指定或查询异常规格:

\begin{itemize}
\item
函数声明中使用noexcept来表示该函数不会抛出异常:

\begin{cpp}
void func_no_throw() noexcept
{
}
\end{cpp}

\item
在函数声明中使用noexcept(expr),如模板元编程中,根据布尔值的条件来表示函数可能抛出异常:

\begin{cpp}
template <typename T>
T generic_func_1()
noexcept(std::is_nothrow_constructible_v<T>)
{
    return T{};
}
\end{cpp}

\item
在编译时使用noexcept操作符检查表达式,是否声明为不抛出异常:

\begin{cpp}
template <typename T>
T generic_func_2() noexcept(noexcept(T{}))
{
    return T{};
}
template <typename F, typename A>
auto func(F&& f, A&& arg) noexcept
{
    static_assert(noexcept(f(arg)), "F is throwing!");
    return f(arg);
}
std::cout << noexcept(generic_func_2<int>) << '\n';
\end{cpp}
\end{itemize}

\mySubsubsection{}{How it works...}

自C++17起,异常规格成为函数类型的一部分,但不是函数签名的一部分,可以出现在函数声明符中。异常规格不是函数签名的一部分,所以两个函数签名不能仅在异常规格上有所不同。

C++17之前,异常规格不是函数类型的一部分,只能作为lambda声明符或顶层函数声明符的一部分,甚至不能在typedef或类型别名声明中出现。关于异常规格的进一步讨论仅涉及C++17。

异常抛出过程可以有几种不同的方式:

\begin{itemize}
\item
如果没有异常规格,则该函数可能会抛出异常。

\item
noexcept(false)等同于没有异常。

\item
noexcept(true)和noexcept表示函数不会抛出异常。

\item
throw()等同于noexcept(true),但在C++17中弃用,并在C++20中彻底移除。
\end{itemize}

\begin{myNotic}
使用异常规格必须谨慎,如果异常(无论是直接抛出还是从另一个调用的函数中抛出)离开了一个标记为不抛出异常的函数,程序会立即且异常地终止(调用std::terminate())。
\end{myNotic}

不抛出异常的函数指针,可以隐式转换为可能抛出异常的函数指针,反之则不行。另一方面,如果虚函数有一个不抛出异常的异常规格,这表明所有重写都必须保持这个规格,除非重写的函数声明为已删除。

在编译时,可以使用noexcept操作符检查函数是否声明为不抛出异常。此操作符接受一个表达式,并在表达式声明为不抛出异常时返回true,否则返回false。其不会计算其检查的表达式。

noexcept操作符与noexcept修饰符一起特别适用于模板元编程,以表明对于某些类型函数是否可能抛出异常。也用于\verb|static_assert|声明中,以检查表达式是否破坏了函数的不抛出保证。

下面提供了更多关于noexcept操作符工作原理的例子:

\begin{cpp}
int double_it(int const i) noexcept
{
    return i + i;
}
int half_it(int const i)
{
    throw std::runtime_error("not implemented!");
}
struct foo
{
    foo() {}
};
std::cout << std::boolalpha
    << noexcept(func_no_throw()) <<  '\n' // true
    << noexcept(generic_func_1<int>()) <<  '\n' // true
    << noexcept(generic_func_1<std::string>()) <<  '\n'// true
    << noexcept(generic_func_2<int>()) << '\n' // true
    << noexcept(generic_func_2<std::string>()) <<  '\n'// true
    << noexcept(generic_func_2<foo>()) <<  '\n' // false
    << noexcept(double_it(42)) <<  '\n' // true
    << noexcept(half_it(42)) <<  '\n' // false
    << noexcept(func(double_it, 42)) <<  '\n' // true
    << noexcept(func(half_it, 42)) << '\n';            // true
\end{cpp}

noexcept修饰符并不提供编译时对异常的检查,只是用户告知编译器一个函数预期不会抛出异常的一种方式。编译器可以利用这一点启用某些优化。例如,std::vector如果元素的移动构造函数是noexcept的话,就会移动元素,否则复制。

\mySubsubsection{}{There's more...}

如前所述,声明为noexcept修饰符的函数如果因异常退出会导致程序异常终止。因此,应谨慎使用noexcept修饰符。其存在可以启用代码优化,帮助提高性能同时保持强异常保证。例如,库容器就是这种情况。

\begin{myNotic}
C++语言提供了几个级别的异常保证:

\begin{itemize}
\item
第一级,无异常保证,不提供保证。如果发生异常,没有可以表明程序是否处于有效状态。资源可能泄露,内存可能损坏,可能会破坏实例的不变性。

\item
基本异常保证是最简单的保证级别,确保在抛出异常后,实例处于一致且可用的状态,不会发生资源泄漏,并且保持不变性。

\item
强异常保证规定要么操作成功完成,因异常完成操作,但程序恢复到操作开始前的状态。这确保了提交或回滚语义。

\item
无异常保证实际上是所有保证中最强大的,其规定操作保证不会抛出异常并成功完成。
\end{itemize}
\end{myNotic}

许多标准容器为其一些操作提供了强异常保证。例如,vector的\verb|push_back()|方法。为了保持其强异常保证,只有当元素类型的移动构造函数或移动赋值操作符不抛出异常时,才能代替复制构造函数或复制赋值操作符来进行优化。

\verb|std::move_if_noexcept()|工具函数如果其类型参数的移动构造函数标记为noexcept,则会执行这样的操作。能够指出移动构造函数或移动赋值操作符,不会抛出异常可能是使用noexcept最重要的场景。

考虑以下异常规则:

\begin{itemize}
\item
如果函数有可能抛出异常,则不要使用异常修饰符。

\item
只将那些保证不会抛出异常的函数标记为noexcept。

\item
只将那些基于条件抛出异常的函数标记为noexcept(expression)。
\end{itemize}

这些规则很重要,从noexcept函数中抛出异常会立即导致程序通过调用std::terminate()终止。

