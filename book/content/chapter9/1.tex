
异常是对程序运行时可能出现的特殊情况的响应,使得控制流可以转移到程序的另一部分。相比于返回错误码(这会大大增加代码的复杂性和混乱度),异常是一种更简单且强大的错误处理机制。本示例中,将介绍一些与抛出和处理异常相关的要点。

\mySubsubsection{}{Getting ready}

本示例要求具备抛出异常(使用 throw 语句)和捕获异常(使用 try...catch 块)的基本机制的知识。本示例专注于围绕异常的良好实践,而不是 C++ 语言中异常机制的具体细节。

\mySubsubsection{}{How to do it...}

使用以下方式来处理异常:

\begin{itemize}
\item
按值抛出异常:

\begin{cpp}
void throwing_func()
{
    throw std::runtime_error("timed out");
}
void another_throwing_func()
{
    throw std::system_error(
    std::make_error_code(std::errc::timed_out));
}
\end{cpp}

\item
按引用或常量引用捕获异常:

\begin{cpp}
try
{
    throwing_func(); // throws std::runtime_error
}
catch (std::exception const & e)
{
    std::cout << e.what() << '\n';
}
\end{cpp}

\item
当从类层次结构中捕获多个异常时,按照从派生类到基类的顺序排列捕获语句:

\begin{cpp}
auto exprint = [](std::exception const & e)
{
    std::cout << e.what() << '\n';
};
try
{
    another_throwing_func(); // throws std::system_error
    // 1st catch statements catches it
}
catch (std::system_error const & e)
{
    exprint(e);
}
catch (std::runtime_error const & e)
{
    exprint(e);
}
catch (std::exception const & e)
{
    exprint(e);
}
\end{cpp}

\item
使用 catch(...) 捕获所有类型的异常:

\begin{cpp}
try
{
    throwing_func();
}
catch (std::exception const & e)
{
    std::cout << e.what() << '\n';
}
catch (...)
{
    std::cout << "unknown exception" << '\n';
}
\end{cpp}

\item
使用 throw; 重新抛出现有异常。这可以用来为多种异常创建单一的异常处理函数。

\item
想要隐藏异常原始位置时,可以抛出异常对象(例如 throw e;):

\begin{cpp}
void handle_exception()
{
    try
    {
        throw; // throw current exception
    }
    catch (const std::logic_error & e)
    { /* ... */ }
    catch (const std::runtime_error & e)
    { /* ... */ }
    catch (const std::exception & e)
    { /* ... */ }
}
try
{
    throwing_func();
}
catch (...)
{
    handle_exception();
}
\end{cpp}
\end{itemize}

\mySubsubsection{}{How it works...}

大多数函数需要指示其执行的成功或失败,可以通过不同的方式实现:

\begin{itemize}
\item
返回错误码(成功时返回特殊值)来指示失败的具体原因:

\begin{cpp}
int f1(int& result)
{
    if (...) return 1;
    // do something
    if (...) return 2;
    // do something more
    result = 42;
    return 0;
}
enum class error_codes {success, error_1, error_2};
error_codes f2(int& result)
{
    if (...) return error_codes::error_1;
    // do something
    if (...) return error_codes::error_2;
    // do something more
    result = 42;
    return error_codes::success;
}
\end{cpp}

\item
返回布尔值的版本,仅指示成功或失败:

\begin{cpp}
bool g(int& result)
{
    if (...) return false;
    // do something
    if (...) return false;
    // do something more
    result = 42;
    return true;
}
\end{cpp}

\item
另一种凡是是返回无效的实例、空指针或空的 std::optional<T> 实例:

\begin{cpp}
std::optional<int> h()
{
    if (...) return {};
    // do something
    if (...) return {};
    // do something more
    return 42;
}
\end{cpp}
\end{itemize}

无论哪种情况,都应检查函数的返回值。这可能导致代码复杂、难以阅读和维护。此外,检查函数返回值的过程总会执行,无论该函数是否成功或失败。相比之下,只有在函数失败时才会抛出和处理异常,这种情况比成功执行要少得多,这可能比返回并测试错误码的代码更快。

\begin{myTip}
异常和错误码不互斥。异常应该只用于在特殊情况中转移控制流,而不是用于控制程序中的数据流。
\end{myTip}

构造函数是不返回值的函数。任务是构建实例,但在失败的情况下,无法通过返回值来表示。异常应该是构造函数用来表示失败的机制。与 RAII(Resource Acquisition Is Initialization,资源获取即初始化)惯用法一起使用时,确保了在所有情况下都能安全地获取和释放资源。另一方面,析构函数中不允许抛出异常。当这种情况发生时,程序会通过调用 std::terminate() 异常终止。这是在栈展开过程中由于另一个异常的发生而调用析构函数的情况。当异常发生时,栈从异常被抛出的地方展开到异常处理块。此过程涉及销毁所有相关栈帧中的局部对象。

如果在此过程中销毁实例的析构函数抛出了异常,则应开始另一个栈展开过程,这与正在进行的栈展开冲突。因此,程序异常终止。

\begin{myTip}
处理构造函数和析构函数中异常的经验法则:

\begin{itemize}
\item
使用异常来表示构造函数中发生的错误。

\item
不要在析构函数中抛出异常。
\end{itemize}
\end{myTip}

代码可以抛出任何类型的异常,大多数情况下,应该抛出临时对象并按常量引用捕获异常。按引用(常量)捕获,原因是为了避免异常类型的切片:

\begin{cpp}
class simple_error : public std::exception
{
public:
    virtual const char* what() const noexcept override
    {
        return "simple exception";
    }
};
try
{
    throw simple_error{};
}
catch (std::exception e)
{
    std::cout << e.what() << '\n'; // prints "Unknown exception"
}
\end{cpp}

抛出了一个 simple\_error 实例,按值捕获了一个 std::exception 实例。simple\_error 是 std::exception 的派生类型。发生了切片过程,派生类型信息丢失,只有实例的 std::exception 部分保留。因此,输出的消息是“未知异常”,而不是预期的“simple exception”。使用引用可以避免对象切片。

以下是关于异常抛出的一些指导原则:

\begin{itemize}
\item
尽量抛出标准异常,或从 std::exception 等其他标准异常派生的自定义异常。因为标准库提供的异常,旨在成为表示异常的首选。应该使用已经可用的标准异常,当这些不够好时,可以基于标准异常构建自己的异常。这样做的好处是一致性和帮助用户通过基类 std::exception 捕获异常。

\item
避免抛出如整数这样的内置类型的异常。这是因为数字对用户的指示很少,用户必须知道其代表什么,而实例可以提供上下文信息。例如,\verb|throw 13;| 对用户没有任何意义,但 \verb|throw access_denied_exception{};| 仅凭类名就携带了更多的隐含信息,加上数据成员的帮助,可以携带有关异常情况的所有有用或必要的信息。

\item
当使用提供了自己异常层次结构的库或框架时,至少在与之紧密相关的代码部分中,优先抛出来自该层次结构的异常或从它派生的自定义异常。这样做的原因是保持库 API 的代码一致性。
\end{itemize}

\mySubsubsection{}{There's more...}

如上一节所述,需要创建自己的异常类型时,除非正在使用具有自己异常层次结构的库或框架,否则应从可用的标准异常之一派生。C++ 标准定义了几种类别的异常,这些类别需要在为此目的选择异常时予以考虑:

\begin{itemize}
\item
\verb|std::logic_error| 表示一个异常,该异常指示程序逻辑中的错误,如无效参数和索引超出范围。有许多标准派生类,例如 \verb|std::invalid_argument|、\verb|std::out_of_range| 和 \verb|std::length_error|。

\item
\verb|std::runtime_error| 表示一个异常,该异常指示超出程序范围或由于各种因素(包括外部因素)而无法预测的错误,如溢出和下溢或操作系统错误。C++ 标准还提供了从 \verb|std::runtime_error| 派生的几个类,包括 \verb|std::overflow_error|、\verb|std::underflow_error|、\verb|std::system_error| 以及 C++20 中的 \verb|std::format_error|。

\item
以 \verb|bad_| 开头的异常,如 \verb|std::bad_alloc|、\verb|std::bad_cast| 和 \verb|std::bad_function_call|,表示程序中的各种错误,如内存分配失败、动态转换失败或函数调用失败。
\end{itemize}

所有这些异常的基类是 std::exception,有一个不抛出异常的虚方法 what(),返回一个指向字符数组的指针,该数组表示错误的描述。

需要从标准异常派生自定义异常时,请使用适当的类别,如逻辑错误或运行时错误。如果这些类别都不合适,可以从 std::exception 直接派生。以下是从标准异常派生的一些解决方案:

\begin{itemize}
\item
需要从 std::exception 派生,则覆盖虚拟方法 what() 以提供错误描述:

\begin{cpp}
class simple_error : public std::exception
{
    public:
    virtual const char* what() const noexcept override
    {
        return "simple exception";
    }
};
\end{cpp}

\item
如果从 \verb|std::logic_error| 或 \verb|std::runtime_error| 派生,并且只需要提供与运行时数据无关的静态描述,则将描述文本传递给基类构造函数:

\begin{cpp}
class another_logic_error : public std::logic_error
{
public:
    another_logic_error():
    std::logic_error("simple logic exception")
    {}
};
\end{cpp}

\item
如果从 \verb|std::logic_error| 或 \verb|std::runtime_error| 派生,但描述消息取决于运行时数据,请提供带参数的构造函数,并使用它们来构建描述消息。可以将描述消息传递给基类构造函数,也可以从覆盖的 what() 方法中返回:

\begin{cpp}
class advanced_error : public std::runtime_error
{
    int error_code;
    std::string make_message(int const e)
    {
        std::stringstream ss;
        ss << "error with code " << e;
        return ss.str();
    }
public:
    advanced_error(int const e) :
    std::runtime_error(make_message(e).c_str()),error_code(e)
    {
    }
    int error() const noexcept
    {
        return error_code;
    }
};
\end{cpp}
\end{itemize}

要获取标准异常类的完整列表,可参见 \url{https://en.cppreference.com/w/cpp/error/exception}。

