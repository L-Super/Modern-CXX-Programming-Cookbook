
C++的类(class)具有特殊的成员(构造函数、析构函数和赋值操作符),这些成员可以由编译器默认实现,也可以由开发者提供。但默认实现的规则较为复杂,并可能导致一些问题。另一方面,开发者有时希望阻止对象以特定方式复制、移动或构造。

C++11标准通过允许函数被删除或默认化简化了许多操作,具体方法将在下一节中介绍。

\mySubsubsection{}{Getting ready}

对于本示例,需要熟悉以下概念:

\begin{itemize}
\item
特殊成员函数(默认构造函数、析构函数、拷贝构造函数、移动构造函数、复制赋值操作符和移动赋值操作符)

\item
复制概念(一个类型具备拷贝构造函数和复制赋值操作符,使得创建副本成为可能)

\item
移动概念(一个类型具备移动构造函数和移动赋值操作符,使得移动对象成为可能)
\end{itemize}

了解这些后,再来介绍默认和删除的特殊函数。

\mySubsubsection{}{How to do it...}

使用以下语法来指定函数应该如何处理:

\begin{itemize}
\item
要默认化一个函数,可使用=default代替函数体。只有编译器能够提供默认实现的特殊类成员函数,才能默认化:

\begin{cpp}
struct foo
{
    foo() = default;
};
\end{cpp}

\item
要删除一个函数,使用=delete代替函数体。可删除任何函数,包括非成员函数:

\begin{cpp}
struct foo
{
    foo(foo const &) = delete;
};
void func(int) = delete;
\end{cpp}
\end{itemize}

使用默认化和删除函数可以实现各种设计目标,例如:

\begin{itemize}
\item
实现一个不可复制且隐式不可移动的类型,声明拷贝构造函数和拷贝赋值操作符为删除:

\begin{cpp}
class foo_not_copyable
{
    public:
    foo_not_copyable() = default;
    foo_not_copyable(foo_not_copyable const &) = delete;
    foo_not_copyable& operator=(foo_not_copyable const&) = delete;
};
\end{cpp}

\item
实现一个不可复制但可移动的类型,声明拷贝操作为删除并显式实现移动操作(并提供构造函数):

\begin{cpp}
class data_wrapper
{
    Data* data;
public:
    data_wrapper(Data* d = nullptr) : data(d) {}
    ~data_wrapper() { delete data; }
    data_wrapper(data_wrapper const&) = delete;
    data_wrapper& operator=(data_wrapper const &) = delete;
    data_wrapper(data_wrapper&& other)
    :data(std::move(other.data))
    {
        other.data = nullptr;
    }
    data_wrapper& operator=(data_wrapper&& other)
    {
        if (data != other.data))
        {
            delete data;
            data = std::move(other.data);
            other.data = nullptr;
        }
        return *this;
    }
};
\end{cpp}

\item
为了确保函数仅与特定类型的对象一起调用,并且可能防止类型提升,提供函数的删除重载(下面的例子适用于独立函数,也适用于任何类成员函数):

\begin{cpp}
template <typename T>
void run(T val) = delete;
void run(long val) {} // can only be called with long integers
\end{cpp}
\end{itemize}

\mySubsubsection{}{How it work...}

类中可以有多个由编译器默认实现的特殊成员,包括默认构造函数、复制构造函数、移动构造函数、复制赋值、移动赋值和析构函数(关于移动语义的讨论,请参阅第9章)。如果没有实现它们,编译器会帮忙实现,以便在需要时,可以创建、移动、复制和销毁类的实例。但若显式地提供了这些特殊方法之一或多个,编译器将根据以下规则不再生成相应的函数:

\begin{itemize}
\item
如果存在用户定义的构造函数,则不会默认生成默认构造函数。

\item
如果存在用户定义的虚析构函数,则不会生成默认析构函数。

\item
如果存在用户定义的移动构造函数或移动赋值操作符,则不会默认生成复制构造函数和复制赋值操作符。

\item
如果存在用户定义的复制构造函数、移动构造函数、复制赋值操作符、移动赋值操作符或析构函数,则不会默认生成移动构造函数和移动赋值操作符。

\item
如果存在用户定义的复制构造函数或析构函数,则会默认生成复制赋值操作符。

\item
如果存在用户定义的拷贝赋值操作符或析构函数,则会默认生成复制构造函数。
\end{itemize}

\begin{myNotic}
上述列表中的最后两条规则是已弃用的规则,可能不再受主流编译器支持。
\end{myNotic}

有时候,开发者需要提供这些特殊成员的空实现或隐藏,避免类型实例以特定方式构造,比如:一个不可复制类型。传统模式是提供一个默认构造函数,并隐藏拷贝构造函数和拷贝赋值操作符,但显式定义的默认构造函数确保了该类不再视为简单类型,因此不再是普通的旧数据(POD)类型。这种情况下的一种替代方案是使用删除函数。

当编译器在函数定义中遇到=default时,将提供默认实现。前面提到的特殊成员函数的规则仍然适用。如果函数是在类体外声明=default的,则必须内联:

\begin{cpp}
class foo
{
public:
    foo() = default;
    inline foo& operator=(foo const &);
};
inline foo& foo::operator=(foo const &) = default;
\end{cpp}

默认实现有几个好处,但不限于:

\begin{itemize}
\item
可能比显式实现更高效。

\item
即使是空的非默认实现也认为是简单类型,这会影响类型的语义,使其变得不简单(因此,非POD)。

\item
帮助用户不必编写显式的默认实现,例如:存在用户定义的移动构造函数,则编译器将不再默认提供复制构造函数和复制赋值操作符。但可以显式地默认化它们,并请求编译器提供这些函数,这样就不必手动实现了。
\end{itemize}

当编译器在函数定义中遇到=delete时,将阻止调用该函数。即使函数被删除,在重载解析过程中仍会考虑该函数,只有当删除的函数是最佳匹配时,编译器才会生成错误。例如,通过为run()函数提供的先前定义的重载,只能用长整型调用。用其他类型(包括可以自动类型提升到长整型的int)作为参数的调用,将导致删除的重载称为最佳匹配的函数,所以编译器将报错:

\begin{cpp}
run(42);  // error, matches a deleted overload
run(42L); // OK, long integer arguments are allowed
\end{cpp}

注意,先前声明的函数不能删除,=delete定义必须是转换单元中的第一个声明:

\begin{cpp}
void forward_declared_function();
// ...
void forward_declared_function() = delete; // error
\end{cpp}

类型中特殊成员函数的经验法则(也称为五法则),如果显式定义了复制构造函数、移动构造函数、复制赋值操作符、移动赋值操作符或析构函数,必须显式定义或默认化所有这些函数。

用户定义的析构函数、复制构造函数和复制赋值操作符是必要的,各种情况下(如向函数传递参数)类型实例从副本构造。如果并非用户定义,编译器帮忙实现,但其默认实现可能有误。如果类型中会对资源进行管理,默认实现会进行浅拷贝,即只复制资源句柄(如指向对象的指针)的值,而非资源本身。这种情况下,用户定义的实现必须执行深拷贝,即复制资源。此时,移动构造函数和移动赋值操作符理想状态下应该存在,因为其代表了性能上的改进。缺少这两个函数并不会有什么错误,但会错过优化的机会。

与五法则相对的是所谓的零法则:除非类型涉及资源所有权的管理,否则不应有自定义的析构函数、复制和移动构造函数,以及相应的复制和移动赋值操作符。

设计类型时,应该遵循以下指导原则:

\begin{itemize}
\item
管理资源的类型应该具有处理该资源所有权的单一职责,这样的类型必须遵循五法则并实现自定义的析构函数、复制/移动构造函数和复制/移动赋值操作符。

\item
不管理资源的类型不应该定义析构函数、复制/移动构造函数和复制/移动赋值操作符(因此遵循零法则)。
\end{itemize}








