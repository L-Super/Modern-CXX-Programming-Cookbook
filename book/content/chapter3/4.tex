lambda是匿名的函数对象,应该可以递归地调用。但这样做动机不明,需要将lambda分配给一个函数包装器并通过引用捕获该包装器。虽然可以说递归lambda没有什么意义,并且通常情况下普通函数可能是一个更好的选择。本示例中,会介绍如何对lambda进行递归。

\mySubsubsection{}{Getting ready}

为了演示如何递归lambda,将以斐波那契函数为示例。这个函数通常以递归方式实现:

\begin{cpp}
constexpr int fib(int const n)
{
    return n <= 2 ? 1 : fib(n - 1) + fib(n - 2);
}
\end{cpp}

基于此实现,来看看如何使用递归lambda对其进行重写。

\mySubsubsection{}{How to do it...}

为了编写递归lambda函数,必须做以下几件事:

\begin{itemize}
\item
在函数作用域内定义lambda。

\item
将lambda分配给std::function包装器。

\item
通过引用在lambda中捕获std::function对象,以便递归调用。
\end{itemize}

C++14中,上述模式可以使用通用lambda来简化:

\begin{itemize}
\item
在函数作用域内定义lambda。

\item
声明第一个参数为auto占位符;这用于将lambda表达式本身作为参数传递给自身。

\item
通过传递lambda本身作为第一个参数来调用lambda表达式。
\end{itemize}

C++23中,这一模式可以进一步简化:

\begin{itemize}
\item
在函数作用域内定义lambda。

\item
声明第一个参数为\verb|const auto&&| self;这是一个新的C++23特性,推导this或显式对象参数。可以通过self参数递归调用lambda表达式。

\item
通过调用lambda表达式并传递显式参数(如果有)来调用,并让编译器推导第一个参数。
\end{itemize}

以下是递归lambda的例子:

\begin{itemize}
\item
在一个函数的作用域内,定义并在同一作用域内调用的递归斐波那契lambda表达式:

\begin{cpp}
void sample()
{
    std::function<int(int const)> lfib =
    [&lfib](int const n)
    {
        return n <= 2 ? 1 : lfib(n - 1) + lfib(n - 2);
    };
    auto f10 = lfib(10);
}
\end{cpp}

\item
由一个函数返回的递归斐波那契lambda表达式:

\begin{cpp}
std::function<int(int const)> fib_create()
{
    std::function<int(int const)> f = [](int const n)
    {
        std::function<int(int const)> lfib = [&lfib](int n)
        {
            return n <= 2 ? 1 : lfib(n - 1) + lfib(n - 2);
        };
        return lfib(n);
    };
    return f;
}

void sample()
{
    auto lfib = fib_create();
    auto f10 = lfib(10);
}
\end{cpp}

\item
作为类成员可递归调用的lambda表达式:

\begin{cpp}
struct fibonacci
{
    std::function<int(int const)> lfib =
    [this](int const n)
    {
        return n <= 2 ? 1 : lfib(n - 1) + lfib(n - 2);
    };
};
fibonacci f;
f.lfib(10);
\end{cpp}

\item
递归斐波那契通用lambda表达式——C++14:

\begin{cpp}
void sample()
{
    auto lfib = [](auto f, int const n)
    {
        if (n < 2) return 1;
        else return f(f, n - 1) + f(f, n - 2);
    };
    lfib(lfib, 10);
}
\end{cpp}

\item
使用C++23特性的显式对象参数(或推导this)特性编写的递归斐波那契lambda表达式,这是对上面示例的进一步简化:

\begin{cpp}
void sample()
{
    auto lfib = [](this const auto& self, int n) -> int
    {
        return n <= 2 ? 1 : self(n - 1) + self(n - 2);
    };
    lfib(5);
}

\end{cpp}
\end{itemize}

\mySubsubsection{}{How it work...}

C++11中递归lambda时,首先考虑的是lambda表达式是一个函数对象,为了能够从lambda体内部递归调用,lambda必须捕获其闭包(即lambda的实例),所以lambda必须捕获自身,而这样做有几个含义:

\begin{itemize}
\item
首先,lambda必须有一个名字;匿名lambda不能捕获以便再次调用。

\item
其次,lambda只能在函数作用域内定义。因为lambda只能捕获函数作用域内的变量,不能捕获具有静态存储的变量。命名空间作用域内定义的类型实例,或带有static或external限定符的类型实例具有静态存储特性。如果lambda是在命名空间作用域内定义的,其闭包也将具有静态存储特性,因此lambda不会捕获。

\item
第三个含义是lambda闭包的类型不能保持未指定状态,不能用auto说明符声明。无法让用auto类型说明符声明的变量出现在初始化器中。因为当初始化器正在处理时,变量的类型尚不清楚。因此,必须指定lambda闭包的类型,可以通过函数包装器std::function来实现。

\item
最后,lambda闭包必须通过引用捕获。如果通过复制(或值)捕获,则会创建一个未初始化的函数包装器副本。最终我们得到一个无法调用的类型实例。即使编译器不会对值捕获提出警告,但在调用闭包时,仍会抛出\verb|std::bad_function_call|异常。
\end{itemize}

“How to do it...”的第一个示例,递归lambda在另一个名为sample()的函数内定义。lambda表达式的签名和主体与介绍部分定义的递归函数fib()相同。lambda闭包会分配给了一个名为lfib的函数包装器,然后通过引用被lambda捕获,并从其主体中递归调用。由于闭包是通过引用捕获的,因此在lambda主体中调用时,将会进行初始化。

第二个示例中,定义了一个返回lambda表达式闭包的函数,该函数又定义并调用了使用其参数调用的递归lambda。当需要从函数返回递归lambda时,必须实现此模式。因为调用递归lambda时,lambda闭包仍然必须可用,\verb|fib_create()|会返回一个函数包装器。调用时,该包装器会创建一个捕获自身的递归lambda。外部的f lambda不捕获任何东西,因此不会有悬空引用的问题。但当调用时,会创建嵌套lambda的闭包,这是实际调用的lambda,并返回应用该递归lfib lambda到其参数的结果。

C++14中编写递归lambda更为简单,代替捕获lambda闭包,作为参数传递(通常是第一个)。可使用auto占位符声明了一个参数:

\begin{cpp}
auto lfib = [](auto f, int const n)
{
    if (n < 2) return 1;
    else return f(f, n - 1) + f(f, n - 2);
};
lfib(lfib, 10);
\end{cpp}

lambda表达式是一个具有函数调用操作符的函数对象,通用lambda是一个具有模板函数调用操作符的函数实例。编译器为上述代码段生成的代码为:

\begin{cpp}
class __lambda_name_3
{
    public:
    template<class T1>
    inline int operator()(T1 f, const int n) const
    {
        if (n < 2) {
            return 1;
        }
        else {
            return f(f, n - 1) + f(f, n - 2);
        }
    }
    template<>
    inline int operator()<__lambda_name_3> (__lambda_name_3 f,
    const int n) const
    {
        if (n < 2) {
            return 1;
        }
        else {
            return f.operator()(__lambda_name_3(f), n - 1) +
            f.operator()(__lambda_name_3(f), n - 2);
        }
    }
};
__lambda_name_3 lfib = __lambda_name_3{};
lfib.operator()(__lambda_name_3(lfib), 10);
\end{cpp}

函数调用操作符是一个模板函数,其第一个参数的类型为模板参数类型。主要模板提供了一个针对类类型本身的完整显式特化,可以将lambda本身作为参数传递来调用,从而避免在C++11中捕获std::function实例的情况。

如果编译器支持C++23,借助显式对象参数特性(也称为推导this),可以进一步简化这个问题。这个能使编译器能够从函数内部判断,所调用表达式是否为左值或右值,或者是否为cv-或ref-限定,并且是什么类型。这一特性使得以下场景成为可能:

\begin{itemize}
\item
避免使用基于重载的cv-和ref-限定符重复代码(例如,最常见的案例是没有限定符和带有const限定符的同一个函数)。

\item
通过使用简单的继承简化奇特的递归模板模式(CRTP),可从模式中去除递归。

\item
简化编写递归lambda的过程。
\end{itemize}

根据“How to do it...”给出的例子,编译器能够推断出第一个参数self的类型,不需要显式地将lambda闭包作为参数传递。

注意,C++23的例子中,可使用尾部返回类型语法定义lambda表达式:

\begin{cpp}
[](this auto const & self, int n) -> int
\end{cpp}

如果没有这行代码,会遇到如下编译错误:

\begin{shell}
error: function 'operator()<(lambda)>' with deduced return type cannot be used before it is defined
\end{shell}

通过对函数的小改动,就不再需要尾部返回类型,而推导this特性又能正常工作:

\begin{cpp}
auto lfib = [](this auto const& self, int n)
{
    if (n <= 2) return 1;
    return self(n - 1) + self(n - 2);
};
\end{cpp}

