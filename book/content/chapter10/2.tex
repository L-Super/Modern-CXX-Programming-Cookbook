
pimpl 是指实现指针(也称为柴郡猫惯用法或编译器防火墙惯用法),是一种不透明指针技术,能够使实现细节与接口分离。这具有改变实现而无需修改接口的优势,因此避免了重新编译使用该接口的代码的需求。当只有实现细节发生变化时,这有可能使得使用 pimpl 惯用法的库在其 ABI 上与旧版本向后兼容。这个示例中,将介绍如何使用现代 C++ 特性来实现 pimpl 惯用法。

\begin{myNotic}
术语 ABI 表示应用程序二进制接口,指的是两个二进制模块之间的接口。通常,这样的模块是库或操作系统,另一个是由用户执行的程序。
\end{myNotic}

\mySubsubsection{}{Getting ready}

读者应熟悉智能指针和 \verb|std::string_view|,这两者都在本书的前几章中讨论过。

为了以一种实用的方式演示 pimpl 惯用法,将使用以下类型,然后按照 pimpl 模式对其进行重构:

\begin{cpp}
class control
{
    std::string text;
    int width = 0;
    int height = 0;
    bool visible = true;
    void draw()
    {
        std::cout
        << "control " << '\n'
        << " visible: " << std::boolalpha << visible <<
            std::noboolalpha << '\n'
        << " size: " << width << ", " << height << '\n'
        << " text: " << text << '\n';
    }
    public:
    void set_text(std::string_view t)
    {
        text = t.data();
        draw();
    }
    void resize(int const w, int const h)
    {
        width = w;
        height = h;
        draw();
    }
    void show()
    {
        visible = true;
        draw();
    }
    void hide()
    {
        visible = false;
        draw();
    }
};
\end{cpp}

此类表示一个具有诸如文本、大小和可见性等属性的控件。每当这些属性发生更改时,控件就会重绘。这个模拟实现中,绘制是将属性的值输出到控制台。

\mySubsubsection{}{How to do it...}

按照以下步骤实现 pimpl 惯用法,通过重构上述 control 类型来举例说明:

\begin{enumerate}
\item
将所有私有成员,包括数据和函数,放入一个单独的类型中,将称这个类型为 pimpl 类型,原来的类型为公共类型。

\item
公共类型的头文件中,对 pimpl 类型可进行前置声明:

\begin{cpp}
// in control.h
class control_pimpl;
\end{cpp}

\item
公共类型定义中,使用 \verb|std::unique_ptr| 声明指向 pimpl 类型的指针。这应该是类型唯一的私有数据成员:

\begin{cpp}
class control
{
std::unique_ptr<control_pimpl, void(*)(control_pimpl*)> pimpl;
public:
    control();
    void set_text(std::string_view text);
    void resize(int const w, int const h);
    void show();
    void hide();
};
\end{cpp}

\item
将 pimpl 类定义放在公共类型的源文件中,pimpl 类型反映了公共类的公共接口:

\begin{cpp}
// in control.cpp
class control_pimpl
{
    std::string text;
    int width = 0;
    int height = 0;
    bool visible = true;
    void draw()
    {
        std::cout
            << "control " << '\n'
            << " visible: " << std::boolalpha << visible
            << std::noboolalpha << '\n'
            << " size: " << width << ", " << height << '\n'
            << " text: " << text << '\n';
    }
    public:
    void set_text(std::string_view t)
    {
        text = t.data();
        draw();
    }
    void resize(int const w, int const h)
    {
        width = w;
        height = h;
        draw();
    }
    void show()
    {
        visible = true;
        draw();
    }
    void hide()
    {
        visible = false;
        draw();
    }
};
\end{cpp}

\item
公共类型的构造函数中实例化 pimpl 类型:

\begin{cpp}
control::control() :
    pimpl(new control_pimpl(),
        [](control_pimpl* pimpl) {delete pimpl; })
{}
\end{cpp}

\item
公共类型成员函数调用 pimpl 类型相应的成员函数:

\begin{cpp}
void control::set_text(std::string_view text)
{
    pimpl->set_text(text);
}
void control::resize(int const w, int const h)
{
    pimpl->resize(w, h);
}
void control::show()
{
    pimpl->show();
}
void control::hide()
{
    pimpl->hide();
}
\end{cpp}
\end{enumerate}

\mySubsubsection{}{How it works...}

pimpl 惯用法允许从库或模块的用户隐藏类型的内部实现:

\begin{itemize}
\item
提供一个干净的接口。

\item
内部实现的变化不影响公共接口,这使得新版本的库(当公共接口保持不变时)能够实现二进制向后兼容。

\item
当内部实现发生变化时,使用此惯用法类型的用户不需要重新编译,无需长时间重新构建。

\item
头文件不需要包含用于私有实现的类型和函数的头文件,再次缩短了构建时间。
\end{itemize}

还有一些缺点需要注意:

\begin{itemize}
\item
需要编写和维护更多的代码。

\item
代码可能可读性差,因为存在一层间接,所有的实现细节都需要在其他文件中查找。示例中,pimpl 类型定义是在公共类型的源文件提供,实际可能位于某个单独的文件中。

\item
公共类型到 pimpl 类型中隔着一层包装,运行时会有轻微的开销,可以忽略不计。

\item
此方法不适用于私有和受保护的成员,这些成员必须对派生类可用。

\item
此方法不适用于私有虚函数,其必须出现在类型中,要么是因为覆写了基类中的函数,要么是必须在派生类中可用以供覆写。
\end{itemize}

\begin{myTip}
作为一个经验法则,当实现 pimpl 惯用法时,总是将所有的私有成员数据和函数(除了虚函数外)放入 pimpl 类型中,并将受保护的数据成员和函数以及所有私有虚函数留在公共类型中。
\end{myTip}

本示例中,\verb|control_pimpl| 类型基本上与原始的 control 类型相同。实践中,pimpl 类型可能需要一个指向公共类的指针,以便调用未移入 pimpl 类型的成员。

关于重构后的 control 类型的实现,指向 \verb|control_pimpl| 的指针由 \verb|std::unique_ptr| 管理。这个指针的声明中,使用了自定义删除器:

\begin{cpp}
std::unique_ptr<control_pimpl, void(*)(control_pimpl*)> pimpl;
\end{cpp}

control 类型在编译器隐式定义的析构函数出现时,\verb|control_pimpl| 类型仍然是不完整的(在头文件中)。这会导致 \verb|std::unique_ptr| 出现错误,它无法删除不完整的类型。这个问题可以通过两种方式解决:

\begin{itemize}
\item
为 control 类提供一个用户定义的析构函数,该析构函数在 \verb|control_pimpl| 类型的完整定义可用后明确实现(即使声明为默认)。

\item
\verb|std::unique_ptr| 提供一个自定义删除器。
\end{itemize}

\mySubsubsection{}{There's more...}

原始的 control 类同时支持复制和移动:

\begin{cpp}
control c;
c.resize(100, 20);
c.set_text("sample");
c.hide();
control c2 = c;             // copy
c2.show();
control c3 = std::move(c2); // move
c3.hide();
\end{cpp}

重构后的 control 类仅支持移动而不支持复制。下面的代码展示了一个既支持复制,又支持移动 control 类型的实现:

\begin{cpp}
class control_copyable
{
    std::unique_ptr<control_pimpl, void(*)(control_pimpl*)> pimpl;
public:
    control_copyable();
    control_copyable(control_copyable && op) noexcept;
    control_copyable& operator=(control_copyable && op) noexcept;
    control_copyable(const control_copyable& op);
    control_copyable& operator=(const control_copyable& op);
    void set_text(std::string_view text);
    void resize(int const w, int const h);
    void show();
    void hide();
};
control_copyable::control_copyable() :
    pimpl(new control_pimpl(),
        [](control_pimpl* pimpl) {delete pimpl; })
{}
control_copyable::control_copyable(control_copyable &&)
    noexcept = default;
control_copyable& control_copyable::operator=(control_copyable &&)
    noexcept = default;
control_copyable::control_copyable(const control_copyable& op)
    : pimpl(new control_pimpl(*op.pimpl),
        [](control_pimpl* pimpl) {delete pimpl; })
{}
control_copyable& control_copyable::operator=(
const control_copyable& op)
{
    if (this != &op)
    {
        pimpl = std::unique_ptr<control_pimpl,void(*)(control_pimpl*)>(
                new control_pimpl(*op.pimpl),
                [](control_pimpl* pimpl) {delete pimpl; });
    }
    return *this;
}
// the other member functions
\end{cpp}

\verb|control_copyable| 既支持复制也支持移动,这里提供了复制构造函数和复制赋值操作符,以及移动构造函数和移动赋值操作符。后者可以默认,但前者需要明确实现,使用复制的实例创建一个新的 \verb|control_pimpl| 实例。



