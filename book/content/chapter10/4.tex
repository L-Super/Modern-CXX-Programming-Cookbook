
虚函数允许派生类修改基类的实现来为类提供专业化点。当通过指向基类的指针或引用处理派生类对象时,对重写虚函数的调用最终会调用派生类中的重写实现。另一方面,定制是一个实现细节,良好的设计应将接口与实现分离。

非虚接口惯用法(Non-Virtual Interface Idiom),由 Herb Sutter 在《C/C++ Users Journal》的一篇文章中提出,通过使(公共)接口非虚且虚函数私有化,促进了接口与实现关注点的分离。

公共虚接口阻止了类型对其接口施加前置条件和后置条件,因为这些方法可以在派生类中重写,所以期望基类实例不能保证公共虚函数的预期行为。此惯用法有助于强制执行接口的契约。

\mySubsubsection{}{Getting ready}

读者应对与虚函数相关的方面有所了解,包括定义和覆盖虚函数、抽象类和纯虚函数说明符。

\mySubsubsection{}{How to do it...}

实现此惯用法需要遵循几个简单的设计指南,这些指南由 Herb Sutter 在《C/C++ Users Journal》19(9),2001 年 9 月期中提出:

\begin{itemize}
\item
将(公共)接口设为非虚。

\item
将虚函数设为私有。

\item
如果基类实现必须从派生类调用,则将虚函数设为受保护。

\item
将基类的析构函数设为公共且虚,或设为受保护且非虚。
\end{itemize}

以下是一个简单的控件层次结构的例子,遵循了上述所有四条指导原则:

\begin{cpp}
class control
{
private:
    virtual void paint() = 0;
protected:
    virtual void erase_background()
    {
        std::cout << "erasing control background..." << '\n';
    }
public:
    void draw()
    {
        erase_background();
        paint();
    }
    virtual ~control() {}
};
class button : public control
{
private:
    virtual void paint() override
    {
        std::cout << "painting button..." << '\n';
    }
protected:
    virtual void erase_background() override
    {
        control::erase_background();
        std::cout << "erasing button background..." << '\n';
    }
};
class checkbox : public button
{
private:
    virtual void paint() override
    {
        std::cout << "painting checkbox..." << '\n';
    }
protected:
    virtual void erase_background() override
    {
        button::erase_background();
        std::cout << "erasing checkbox background..." << '\n';
    }
};
\end{cpp}

\mySubsubsection{}{How it works...}

NVI 惯用法使用了模板方法设计模式,允许派生类定制基类功能(即算法)的部分(即步骤)。通过将算法分解成更小的部分来完成,每部分都由一个虚函数实现。基类可以提供或不提供默认实现,而派生类可以对其进行覆盖,同时保持算法的整体结构和意义。

NVI 惯用法的核心原则是虚函数不应公开;应该是私有的,或者在基类中为受保护时,可以使用派生类调用。类型的接口,即用户可访问的公共部分,应完全由非虚函数组成。这有几个优点:

\begin{itemize}
\item
接口与不再暴露给客户的实现细节分开。

\item
在不改变公共接口的情况下更改实现细节成为可能,而无需更改用户代码,从而使基类更加健壮。

\item
允许一个类型对其接口拥有唯一的控制权。如果公共接口包含虚方法,派生类可以改变承诺的功能,所以无法确保其前置条件和后置条件。当除析构函数外的虚方法都不为用户所访问时,类型可以强制执行其接口的前置条件和后置条件。
\end{itemize}

\begin{myNotic}
对于这个惯用法,需要特别提到类的析构函数。通常强调基类的析构函数应该是虚的,以便实例可以多态地删除(通过指向基类的指针或引用)。当析构函数非虚时,多态地销毁实例会导致未定义的行为。然而,并非所有的基类都是打算多态地删除。对于这些情况,基类的析构函数不应该声明为虚。但也不应该是公共的,而应该是受保护的。
\end{myNotic}

前一节中的例子定义了一个表示视觉控件的类层次结构:

\begin{itemize}
\item
control 是基类,但有派生类,如 button 和 checkbox,是按钮的一种类型,因此是从这个类派生出来的。

\item
control 类定义的唯一功能是绘制控件。\verb|draw()| 方法非虚,但调用了两个虚方法,\verb|erase_background()| 和 \verb|paint()|,以实现绘制控件的两个阶段。

\item
\verb|erase_background()| 是一个受保护的虚方法,派生类需要在自己的实现中调用。

\item
paint() 是一个私有的纯虚方法。派生类必须进行实现,但不希望调用基类的实现。

\item
control 类的析构函数是公共且虚的,预计删除实例会是多态地。
\end{itemize}

以下展示了使用这些类的例子。这些类型实例由指向基类的智能指针管理:

\begin{cpp}
std::vector<std::unique_ptr<control>> controls;
controls.emplace_back(std::make_unique<button>());
controls.emplace_back(std::make_unique<checkbox>());
for (auto& c : controls)
    c->draw();
\end{cpp}

输出如下:

\begin{shell}
erasing control background...
erasing button background...
painting button...
erasing control background...
erasing button background...
erasing checkbox background...
painting checkbox...
destroying button...
destroying control...
destroying checkbox...
destroying button...
destroying control...
\end{shell}

NVI 惯用法在公共函数调用执行,实际实现的非公共虚函数时引入了一层间接性。前面的例子中,draw() 方法调用了几个其他函数,但通常只调用一次:

\begin{cpp}
class control
{
protected:
    virtual void initialize_impl()
    {
        std::cout << "initializing control..." << '\n';
    }
public:
    void initialize()
    {
        initialize_impl();
    }
};
class button : public control
{
protected:
    virtual void initialize_impl()
    {
        control::initialize_impl();
        std::cout << "initializing button..." << '\n';
    }
};
\end{cpp}

这个例子中,control 类性有一个方法叫做 initialize()(为了简化起见,没有显示类的先前内容),调用了一个 \verb|initialize_impl()| 的单个非公共虚方法,该方法在每个派生类中实现不同。这不会带来多少开销——如果有开销的话——像这样的简单函数,很可能会被编译器内联。

