
C++ 只支持位置参数,可根据参数的位置传递给函数。其他语言还支持命名参数——调用时指定参数名称和调用参数。对于具有默认值的参数,这很有用。一个函数可能有带有默认值的参数,尽管其总是出现在所有非默认参数之后。

一种称为命名参数惯用法的技术,提供了一种方法来模拟命名参数并帮助解决“为默认参数之前的参数传参”的问题。

\mySubsubsection{}{Getting ready}

为了举例说明命名参数惯用法,将使用如下代码中的 control 类型:

\begin{cpp}
class control
{
    int id_;
    std::string text_;
    int width_;
    int height_;
    bool visible_;
public:
    control(
    int const id,
    std::string_view text = "",
    int const width = 0,
    int const height = 0,
    bool const visible = false):
        id_(id), text_(text),
        width_(width), height_(height),
        visible_(visible)
    {}
};
\end{cpp}

control 类型代表一个可视控件,如按钮或输入框,并具有数值标识符、文本、尺寸和可见性等属性。这些属性提供给构造函数,除了 ID 之外,所有其他的都有默认值。实际上,这样的类型会有许多其他属性,比如:文本画笔、背景画笔、边框样式、字体大小、字体系列等。

\mySubsubsection{}{How to do it...}

为了实现一个函数(通常有很多默认参数)的命名参数惯用法,按照以下步骤操作:

\begin{enumerate}
\item
创建一个类型来包装函数的参数:

\begin{cpp}
class control_properties
{
    int id_;
    std::string text_;
    int width_ = 0;
    int height_ = 0;
    bool visible_ = false;
};
\end{cpp}

\item
需要访问这些属性的类或函数,可以声明为友元(以避免编写多余的 getter):

\begin{cpp}
friend class control;
\end{cpp}

\item
原始函数中,每个没有默认值的位置参数,应该成为类构造函数的一个位置参数,且没有默认值:

\begin{cpp}
public:
    control_properties(int const id) :id_(id)
    {}
\end{cpp}

\item
对于原始函数中每个有默认值的位置参数,应该有一个同名的函数,其内部设置值并返回一个对该类的引用:

\begin{cpp}
public:
    control_properties& text(std::string_view t)
    { text_ = t.data(); return *this; }
    control_properties& width(int const w)
    { width_ = w; return *this; }
    control_properties& height(int const h)
    { height_ = h; return *this; }
    control_properties& visible(bool const v)
    { visible_ = v; return *this; }
\end{cpp}

\item
应该修改原始函数,或者提供一个重载版本,以接受一个新类的参数,从中读取属性值:

\begin{cpp}
control(control_properties const & cp):
    id_(cp.id_),
    text_(cp.text_),
    width_(cp.width_),
    height_(cp.height_),
    visible_(cp.visible_)
{}
\end{cpp}
\end{enumerate}

综上所述,结果如下:

\begin{cpp}
class control;
class control_properties
{
    int id_;
    std::string text_;
    int width_ = 0;
    int height_ = 0;
    bool visible_ = false;
    friend class control;
public:
    control_properties(int const id) :id_(id)
    {}
    control_properties& text(std::string_view t)
    { text_ = t.data(); return *this; }
    control_properties& width(int const w)
    { width_ = w; return *this; }
    control_properties& height(int const h)
    { height_ = h; return *this; }
    control_properties& visible(bool const v)
    { visible_ = v; return *this; }
};
class control
{
    int         id_;
    std::string text_;
    int         width_;
    int         height_;
    bool        visible_;
    public:
    control(control_properties const & cp):
        id_(cp.id_),
        text_(cp.text_),
        width_(cp.width_),
        height_(cp.height_),
        visible_(cp.visible_)
    {}
};
\end{cpp}

\mySubsubsection{}{How it works...}

最初的 control 类有一个带有多个参数的构造函数。实际中,可以找到参数数量很多的例子。实践中常遇到的一种解决方案是,将常见的布尔类型属性组合成位标志,这些位标志可以作为一个单一的整数参数一起传递(例如,控件的边框样式,定义边框应在哪些位置可见:顶部、底部、左侧、右侧,或者是这四个位置的任意组合)。使用初始实现创建一个 control 实例如下所示:

\begin{cpp}
control c(1044, "sample", 100, 20, true);
\end{cpp}

命名参数惯用法的优点在于,允许仅为想要的参数指定值,顺序任意,使用名称指定,比固定的位置顺序直观得多。

虽然实现这一惯用法并没有单一的策略,但本示例相当典型。control 类型的构造函数中提供的属性已放入一个单独的类型中,称为 \verb|control_properties|,将 control 类型声明为友元类,允许访问其私有数据成员而无需提供 getter 方法。这有一个副作用,即限制了 \verb|control_properties| 在 control 类型外部的使用。control 类型构造函数的非可选参数也是 \verb|control_properties| 构造函数的非可选参数。对于所有其他带有默认值的参数,\verb|control_properties| 类型定义了一个具有相关名称的函数,该函数只是将数据成员设置为提供的参数,然后返回一个 \verb|control_properties| 的引用。这使得用户能够以任意顺序链式调用这些函数。

control 类型的构造函数已经替换为一个新的构造函数,该构造函数只有一个参数,即对 \verb|control_properties| 实例的常量引用,其数据成员会复制到 control 实例的数据成员中。

使用这种方式实现的命名参数惯用法,创建一个 control 实例:

\begin{cpp}
control c(control_properties(1044)
        .visible(true)
        .height(20)
        .width(100));
\end{cpp}


