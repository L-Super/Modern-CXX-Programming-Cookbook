
编程中,经常发现自己编写重复的 if...else 语句(或等效的 switch 语句),这些语句执行相似的操作,通常只有细微的变化,而且往往是通过复制粘贴再做少量修改来完成的。随着替代条件的数量增加,代码变得既难以阅读又难以维护。重复的 if...else 语句可以通过多种技术来替换,比如多态性。本示例中,将介绍如何使用函数映射来避免工厂模式中的 if...else 语句(工厂是一个用于创建其他实例的函数或对象)。

\mySubsubsection{}{Getting ready}

本示例中,会考虑以下问题:构建一个能够处理各种格式图像文件的系统,如位图、PNG、JPG 等等。显然,细节超出了这个实例的范围;关注的部分是创建处理各种图像格式的对象。为此,将考虑以下类层次结构:

\begin{cpp}
class Image {};
class BitmapImage : public Image {};
class PngImage    : public Image {};
class JpgImage    : public Image {};
\end{cpp}

另一方面,定义一个工厂类的接口,可以创建上述类的实例,以及一个使用 if...else 语句的典型实现:

\begin{cpp}
struct IImageFactory
{
    virtual std::unique_ptr<Image> Create(std::string_view type) = 0;
};
struct ImageFactory : public IImageFactory
{
    std::unique_ptr<Image>
    Create(std::string_view type) override
    {
        if (type == "bmp")
            return std::make_unique<BitmapImage>();
        else if (type == "png")
            return std::make_unique<PngImage>();
        else if (type == "jpg")
            return std::make_unique<JpgImage>();
        return nullptr;
    }
};
\end{cpp}

\mySubsubsection{}{How to do it...}

按照以下步骤重构前面显示的工厂以避免使用 if...else 语句:

\begin{enumerate}
\item
实现工厂接口:

\begin{cpp}
struct ImageFactory : public IImageFactory
{
    std::unique_ptr<Image> Create(std::string_view type) override
    {
        // continued with 2. and 3.
    }
};
\end{cpp}

\item
定义一个map,其中键是创建对象的类型,值是一个创建这些实例的函数:

\begin{cpp}
static std::map<
    std::string,
    std::function<std::unique_ptr<Image>()>> mapping
{
    { "bmp", []() {return std::make_unique<BitmapImage>(); } },
    { "png", []() {return std::make_unique<PngImage>(); } },
    { "jpg", []() {return std::make_unique<JpgImage>(); } }
};
\end{cpp}

\item
要创建一个对象,可以在map中查找类型,如果找到了,就使用与之关联的函数来创建该类型的实例:

\begin{cpp}
auto it = mapping.find(type.data());
if (it != mapping.end())
    return it->second();
return nullptr;
\end{cpp}
\end{enumerate}

\mySubsubsection{}{How it works...}

最初的实现中,重复的 if...else 语句非常相似——检查类型参数的值,并创建适当图像类的实例。如果要检查的参数是整数类型(例如,枚举类型),if...else 语句序列也可以用 switch 语句的形式编写:

\begin{cpp}
auto factory = ImageFactory{};
auto image = factory.Create("png");
\end{cpp}

无论是使用 if...else 语句还是 switch,重构以避免重复检查都相对简单。重构后的代码中,使用了一个键类型为 std::string 的映射来表示类型,即图像格式的名称。值是一个 \verb|std::function<std::unique_ptr<Image>()>|。这是一个接受无参数并返回 \verb|std::unique_ptr<Image>|(派生类的 \verb|unique_ptr| 可以隐式转换为基类的 \verb|unique_ptr|)的函数的包装器。

现在有了这个创建对象的函数映射,工厂的实际实现变得更加简单;在映射中检查要创建的对象类型,如果存在,则使用映射中的相应值作为实际创建对象的函数,或者如果对象类型不在映射中则返回 nullptr。

这种重构对于用户代码来说是透明的,因为使用工厂的方式没有变化。另一方面,这种方法需要更多的内存来处理静态映射,这在某些应用类别中,如物联网(IoT),可能很重要。这里展示的例子相对简单,目的是为了说明概念。实际中,可能需要以不同的方式创建对象,比如使用不同数量和类型的参数。然而,这不是重构实现所特有的,使用 if...else/switch 语句的解决方案也需要考虑到这一点。实践中,使用 if...else 语句解决问题的方案也应适用于映射。

\mySubsubsection{}{There's more...}

前面的实现中,映射是虚拟函数的局部静态变量,但也可以是类的成员甚至全局变量。下面的实现将映射定义为类的静态成员。对象不是根据格式名创建的,而是根据 typeid 操作符返回类型的信息创建:

\begin{cpp}
struct IImageFactoryByType
{
    virtual std::unique_ptr<Image> Create(
    std::type_info const & type) = 0;
};
struct ImageFactoryByType : public IImageFactoryByType
{
    std::unique_ptr<Image> Create(std::type_info const & type)
    override
    {
        auto it = mapping.find(&type);
        if (it != mapping.end())
        return it->second();
        return nullptr;
    }
private:
    static std::map<
    std::type_info const *,
    std::function<std::unique_ptr<Image>()>> mapping;
};
std::map<
std::type_info const *,
std::function<std::unique_ptr<Image>()>> ImageFactoryByType::mapping
{
    {&typeid(BitmapImage),[](){
            return std::make_unique<BitmapImage>();}},
    {&typeid(PngImage),   [](){
            return std::make_unique<PngImage>();}},
    {&typeid(JpgImage),   [](){
            return std::make_unique<JpgImage>();}}
};
\end{cpp}

用户代码略有不同,因为传递的是由 typeid 操作符返回的值,而不是表示要创建的类型的名称,例如 PNG,传递的是 typeid(PngImage):

\begin{cpp}
auto factory = ImageFactoryByType{};
auto movie = factory.Create(typeid(PngImage));
\end{cpp}

这种替代方法更加健壮,因为映射的键不是字符串,后者更容易出错。本示例提出了一种模式作为解决常见问题的方案,而不是具体的实现。就像大多数模式一样,可以有多种实现方式,可选择最适合特定上下文的方法。


