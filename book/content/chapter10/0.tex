
设计模式是通用的可重用解决方案,可以应用于软件开发中出现的常见问题。惯用法(idioms)是指在一种或多种编程语言中结构化代码的模式、算法或方式。关于设计模式已经有许多书籍出版。本章的目的不是重复这些内容,而是展示如何在现代C++中实现几个有用的设计模式和惯用法,重点关注代码的可读性、性能和健壮性。

本章包括以下实例:

\begin{itemize}
\item
工厂模式中避免重复的if-else语句

\item
实现pimpl惯用法

\item
实现命名参数惯用法

\item
使用非虚接口惯用法分离接口和实现

\item
使用律师-客户惯用法处理友元关系

\item
使用奇异递归模板模式实现静态多态

\item
使用混入(mixins)为类型添加功能

\item
使用类型擦除惯用法以泛型方式处理不相关类型

\item
实现线程安全的单例模式
\end{itemize}
