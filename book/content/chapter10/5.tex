
授予函数和类通过友元声明访问一个类的非公共部分,通常会视为设计不佳的标志,因为友元关系会破坏了封装,并且将类和函数绑定在一起。无论是作为类还是函数的友元,都可以访问该类的所有私有成员,即使它们可能只需要访问其中的一部分。

律师-客户惯用法(Attorney-Client Idiom)提供了一种简单的机制,以限制友元对类中仅指定的私有成员的访问。

\mySubsubsection{}{Getting ready}

为了演示如何实现这一惯用法,使用以下类:Client,有一些私有成员数据和函数(此处公共接口不重要),以及 Friend,只访问私有细节的一部分,例如 data1 和 action1(),但却可以访问所有内容:

\begin{cpp}
class Client
{
    int data_1;
    int data_2;
    void action1() {}
    void action2() {}
    friend class Friend;
public:
    // public interface
};
class Friend
{
public:
    void access_client_data(Client& c)
    {
        c.action1();
        c.action2();
        auto d1 = c.data_1;
        auto d2 = c.data_1;
    }
};
\end{cpp}

要理解这一惯用法,必须熟悉 C++ 语言中友元是如何声明的以及是如何工作的。

\mySubsubsection{}{How to do it...}

按照以下步骤限制友元对类型中,仅需访问的私有成员进行访问:

\begin{enumerate}
\item
Client 类型中,向友元提供了对所有私有成员的访问权限,声明对一个中间类型的友元关系,这个中间类型称为 Attorney 类型:

\begin{cpp}
class Client
{
    int data_1;
    int data_2;
    void action1() {}
    void action2() {}
    friend class Attorney;
public:
    // public interface
};
\end{cpp}

\item
创建一个只包含访问用户私有成员的私有(内联)函数的类型。这个中间类型允许实际的友元访问其私有成员:

\begin{cpp}
class Attorney
{
    static inline void run_action1(Client& c)
    {
        c.action1();
    }
    static inline int get_data1(Client& c)
    {
        return c.data_1;
    }
    friend class Friend;
};
\end{cpp}

\item
Friend 类型中,通过 Attorney 类型间接访问 Client 类型的私有成员:

\begin{cpp}
class Friend
{
public:
    void access_client_data(Client& c)
    {
        Attorney::run_action1(c);
        auto d1 = Attorney::get_data1(c);
    }
};
\end{cpp}
\end{enumerate}

\mySubsubsection{}{How it works...}

律师-客户惯用法通过引入一个中介——律师,来布置一个简单的机制,以限制对客户私有成员的访问。客户类型不是直接向使用其内部状态的实例提供友元关系,而是向律师提供友元关系,律师反过来提供对客户私有数据或函数的受限集的访问。律师通过定义私有静态函数来实现这一点,这些也是内联函数,可以避免由于律师类引入的间接层导致的运行时开销。客户的友元通过实际使用律师的私有成员来访问客户的私有成员。这一惯用法称为律师-客户,其类似于律师-客户关系的工作方式,即律师知道客户的所有秘密,但只向其他方透露其中的一些。

实际上,如果不同的友元类或函数需要访问不同的私有成员,可能有必要为客户创建不止一个律师。

另一方面,友元关系不是可继承的,所以与类 B 成为友元的类或函数并不是与从 B 派生的类 D 成为友元。然而,D 中重写的虚函数,仍然可以通过指向 B 的指针或引用从友元类中多态地访问。下面的例子展示了这种情况,从 F 调用 run() 方法会打印出 base 和 derived:

\begin{cpp}
class B
{
    virtual void execute() { std::cout << "base" << '\n'; }
    friend class BAttorney;
};
class D : public B
{
    virtual void execute() override
    { std::cout << "derived" << '\n'; }
};
class BAttorney
{
    static inline void execute(B& b)
    {
        b.execute();
    }
    friend class F;
};
class F
{
public:
    void run()
    {
        B b;
        BAttorney::execute(b); // prints 'base'
        D d;
        BAttorney::execute(d); // prints 'derived'
    }
};
F;
f.run();
\end{cpp}

使用设计模式总是存在权衡,这一模式也不例外。某些情况下,使用该模式可能导致开发、测试和维护方面的开销过多。然而,对于某些类型的软件,如可扩展框架,该模式可能极具价值。



